\documentclass{xourseXloud}
%\documentclass[handout]{xourseXloud}

%%%% See bottom (post \end{document}) for changelog.

\newcommand{\recordvariable}[1]{}

\usepackage{longdivision}
\usepackage{polynom}
\usepackage{float}% Use `H' as the figure optional argument to force it's vertical placement to conform to source.
%\usepackage{caption}% Allows us to describe the figures without having "figure 1:" in it. :: Apparently Caption isn't supported.
%    \captionsetup{labelformat=empty}% Actually does the figure configuration stated above.
\usetikzlibrary{arrows.meta,arrows}% Allow nicer arrow heads for tikz.
\usepackage{gensymb, pgfplots}
\usepackage{tabularx}
\usepackage{arydshln}



\graphicspath{
  {./}
  {./explorePolynomials/}
  {./exploreRadicals/}
  {./graphing/}
}

%% Default style for tikZ
\pgfplotsset{my style/.append style={axis x line=middle, axis y line=
middle, xlabel={$x$}, ylabel={$y$}, axis equal }}


%% Because log being natural log is too hard for people.
\let\logOld\log% Keep the old \log definition, just in case we need it.
\renewcommand{\log}{\ln}


%%% Changes in polynom to show the zero coefficient terms
\makeatletter
\def\pld@CF@loop#1+{%
    \ifx\relax#1\else
        \begingroup
          \pld@AccuSetX11%
          \def\pld@frac{{}{}}\let\pld@symbols\@empty\let\pld@vars\@empty
          \pld@false
          #1%
          \let\pld@temp\@empty
          \pld@AccuIfOne{}{\pld@AccuGet\pld@temp
                            \edef\pld@temp{\noexpand\pld@R\pld@temp}}%
           \pld@if \pld@Extend\pld@temp{\expandafter\pld@F\pld@frac}\fi
           \expandafter\pld@CF@loop@\pld@symbols\relax\@empty
           \expandafter\pld@CF@loop@\pld@vars\relax\@empty
           \ifx\@empty\pld@temp
               \def\pld@temp{\pld@R11}%
           \fi
          \global\let\@gtempa\pld@temp
        \endgroup
        \ifx\@empty\@gtempa\else
            \pld@ExtendPoly\pld@tempoly\@gtempa
        \fi
        \expandafter\pld@CF@loop
    \fi}
\def\pld@CMAddToTempoly{%
    \pld@AccuGet\pld@temp\edef\pld@temp{\noexpand\pld@R\pld@temp}%
    \pld@CondenseMonomials\pld@false\pld@symbols
    \ifx\pld@symbols\@empty \else
        \pld@ExtendPoly\pld@temp\pld@symbols
    \fi
    \ifx\pld@temp\@empty \else
        \pld@if
            \expandafter\pld@IfSum\expandafter{\pld@temp}%
                {\expandafter\def\expandafter\pld@temp\expandafter
                    {\expandafter\pld@F\expandafter{\pld@temp}{}}}%
                {}%
        \fi
        \pld@ExtendPoly\pld@tempoly\pld@temp
        \pld@Extend\pld@tempoly{\pld@monom}%
    \fi}
\makeatother




%%%%% Code for making prime factor trees for numbers, taken from user Qrrbrbirlbel at: https://tex.stackexchange.com/questions/131689/how-to-automatically-draw-tree-diagram-of-prime-factorization-with-latex

\usepackage{forest,mathtools,siunitx}
\makeatletter
\def\ifNum#1{\ifnum#1\relax
  \expandafter\pgfutil@firstoftwo\else
  \expandafter\pgfutil@secondoftwo\fi}
\forestset{
  num content/.style={
    delay={
      content/.expanded={\noexpand\num{\forestoption{content}}}}},
  pt@prime/.style={draw, circle},
  pt@start/.style={},
  pt@normal/.style={},
  start primeTree/.style={%
    /utils/exec=%
      % \pt@start holds the current minimum factor, we'll start with 2
      \def\pt@start{2}%
      % \pt@result will hold the to-be-typeset factorization, we'll start with
      % \pgfutil@gobble since we don't want a initial \times
      \let\pt@result\pgfutil@gobble
      % \pt@start@cnt holds the number of ^factors for the current factor
      \def\pt@start@cnt{0}%
      % \pt@lStart will later hold "l"ast factor used
      \let\pt@lStart\pgfutil@empty,
    alias=pt-start,
    pt@start/.try,
    delay={content/.expanded={$\noexpand\num{\forestove{content}}
                            \noexpand\mathrlap{{}= \noexpand\pt@result}$}},
    primeTree},
  primeTree/.code=%
    % take the content of the node and save it in the count
    \c@pgf@counta\forestove{content}\relax
    % if it's 2 we're already finished with the factorization
    \ifNum{\c@pgf@counta=2}{%
      % add the factor
      \pt@addfactor{2}%
      % finalize the factorization of the result
      \pt@addfactor{}%
      % and set the style to the prime style
      \forestset{pt@prime/.try}%
    }{%
      % this simply calculates content/2 and saves it in \pt@end
      % this is later used for an early break of the recursion since no factor
      % can be greater then content/2 (for integers of course)
      \edef\pt@content{\the\c@pgf@counta}%
      \divide\c@pgf@counta2\relax
      \advance\c@pgf@counta1\relax % to be on the safe side
      \edef\pt@end{\the\c@pgf@counta}%
      \pt@do}}

%%% our main "function"
\def\pt@do{%
  % let's test if the current factor is already greather then the max factor
  \ifNum{\pt@end<\pt@start}{%
    % great, we're finished, the same as above
    \expandafter\pt@addfactor\expandafter{\pt@content}%
    \pt@addfactor{}%
    \def\pt@next{\forestset{pt@prime/.try}}%
  }{%
    % this calculates int(content/factor)*factor
    % if factor is a factor of content (without remainder), the result will
    % equal content. The int(content/factor) is saved in \pgf@temp.
    \c@pgf@counta\pt@content\relax
    \divide\c@pgf@counta\pt@start\relax
    \edef\pgf@temp{\the\c@pgf@counta}%
    \multiply\c@pgf@counta\pt@start\relax
    \ifNum{\the\c@pgf@counta=\pt@content}{%
      % yeah, we found a factor, add it to the result and ...
      \expandafter\pt@addfactor\expandafter{\pt@start}%
      % ... add the factor as the first child with style pt@prime
      % and the result of int(content/factor) as another child.
      \edef\pt@next{\noexpand\forestset{%
        append={[\pt@start, pt@prime/.try]},
        append={[\pgf@temp, pt@normal/.try]},
        % forest is complex, this makes sure that for the second child, the
        % primeTree style is not executed too early (there must be a better way).
        delay={
          for descendants={
            delay={if n'=1{primeTree, num content}{}}}}}}%
    }{%
      % Alright this is not a factor, let's get the next factor
      \ifNum{\pt@start=2}{%
        % if the previous factor was 2, the next one will be 3
        \def\pt@start{3}%
      }{%
        % hmm, the previos factor was not 2,
        % let's add 2, maybe we'll hit the next prime number
        % and maybe a factor
        \c@pgf@counta\pt@start
        \advance\c@pgf@counta2\relax
        \edef\pt@start{\the\c@pgf@counta}%
      }%
      % let's do that again
      \let\pt@next\pt@do
    }%
  }%
  \pt@next
}

%%% this builds the \pt@result macro with the factors
\def\pt@addfactor#1{%
  \def\pgf@tempa{#1}%
  % is it the same factor as the previous one
  \ifx\pgf@tempa\pt@lStart
    % add 1 to the counter
    \c@pgf@counta\pt@start@cnt\relax
    \advance\c@pgf@counta1\relax
    \edef\pt@start@cnt{\the\c@pgf@counta}%
  \else
    % a new factor! Add the previous one to the product of factors
    \ifx\pt@lStart\pgfutil@empty\else
      % as long as there actually is one, the \ifnum makes sure we do not add ^1
      \edef\pgf@tempa{\noexpand\num{\pt@lStart}\ifnum\pt@start@cnt>1 
                                           ^{\noexpand\num{\pt@start@cnt}}\fi}%
      \expandafter\pt@addfactor@\expandafter{\pgf@tempa}%
    \fi
    % setup the macros for the next round
    \def\pt@lStart{#1}% <- current (new) factor
    \def\pt@start@cnt{1}% <- first time
  \fi
}
%%% This simply appends "\times #1" to \pt@result, with etoolbox this would be
%%% \appto\pt@result{\times#1}
\def\pt@addfactor@#1{%
  \expandafter\def\expandafter\pt@result\expandafter{\pt@result \times #1}}

%%% Our main macro:
%%% #1 = possible optional argument for forest (can be tikz too)
%%% #2 = the number to factorize
\newcommand*{\PrimeTree}[2][]{%
  \begin{forest}%
    % as the result is set via \mathrlap it doesn't update the bounding box
    % let's fix this:
    tikz={execute at end scope={\pgfmathparse{width("${}=\pt@result$")}%
                         \path ([xshift=\pgfmathresult pt]pt-start.east);}},
    % other optional arguments
    #1
    % And go!
    [#2, start primeTree]
  \end{forest}}
\makeatother


\providecommand\tabitem{\makebox[1em][r]{\textbullet~}}
\providecommand{\letterPlus}{\makebox[0pt][l]{$+$}}
\providecommand{\letterMinus}{\makebox[0pt][l]{$-$}}



\title{Precalculus Algebra MAC1140}
\usepackage[margin=1.5cm]{geometry}


\begin{document}
\begin{abstract}
    Xronos Interactive Text for MAC1140 - Precalculus Algebra
\end{abstract}
\maketitle

\begin{onlineOnly}
    % Part 1: Xronos Tutorial
    \part{Xronos Tutorial}
    \chapterstyle
        \activity{./xronosTutorial/goals}
    \sectionstyle
        \activity{./xronosTutorial/xronosBackground}
        \activity{./xronosTutorial/accessingXronos}
        \activity{./xronosTutorial/howToUseXronos}
        \activity{./xronosTutorial/howIsMyWorkScored}
    \newpage
\end{onlineOnly}


% Part 2: Course intro and syllabus
\part{MAC1140 Introduction and Syllabus}
    \chapterstyle
        \activity{./introductionAndSyllabus/goals}
    \sectionstyle
        \activity{./introductionAndSyllabus/goalsOfCourse}
        \activity{./introductionAndSyllabus/differenceFromOtherCourses}
        \activity{./introductionAndSyllabus/virtuesOfCBT}
        \activity{./introductionAndSyllabus/methodsToPrepare}
        \activity{./introductionAndSyllabus/syllabus}
        \activity{./introductionAndSyllabus/gradingRubric}
        \activity{./examReview/preRequisites-Fluency}

\newpage



% Part 3: Mathematical Modeling
\part{Mathematical Modeling}
    \chapterstyle
        \activity{./mathematicalModeling/goals}
    \sectionstyle
        \activity{./mathematicalModeling/terminologyAndNotation}
        \activity{./mathematicalModeling/whatIsMathematicalModeling}
        \activity{./mathematicalModeling/logicalDeductionAndThreePhaseApproach}
        \activity{./mathematicalModeling/typesOfInformation}
        \activity{./mathematicalModeling/isThisActualMath}
        \activity{./mathematicalModeling/mathAsALanguage}
        \activity{./mathematicalModeling/numericModelWalkthrough}
        \activity{./mathematicalModeling/embraceLaziness}
        \activity{./mathematicalModeling/variablesAndTheirRoles}
        \activity{./mathematicalModeling/detailedGeneralizedModelWalkthrough}
\newpage


% Part 4: Functions and universal properties including graphing, domain/range, types of variables, etc. %%
\part{Variables, Functions, Graphing, and Universal Properties}
    \chapterstyle
        \activity{./functions/goals}% Section 4: Functions
    \sectionstyle
        \activity{./functions/terminologyAndNotation}
        \activity{./functions/relationsAndInfluence}
        \activity{./functions/relationshipVsFunctions}
        \activity{./functions/functionsRequireContext}
        \practice{./functions/Practice/functionsBlackestMagic-Practice}
        \activity{./functions/domainCodomainRange}
        \activity{./functions/setNotation}
        \practice{./functions/Practice/domainCodomainRange-Practice1}
        \activity{./functions/functionNotation}
        \activity{./functions/fxNotation}
        \practice{./functions/Practice/functionNotation-Practice1}
        \activity{./functions/functionComposition}
        \practice{./functions/Practice/functionComposition-Practice1}
    
    \chapterstyle
        \activity{./graphing/graphing}% Section 5: Graphing basics
    \sectionstyle
        \activity{./graphing/frenchHistoryAndDinosaurs}
        \activity{./graphing/graphingToRelateVariables}
        \activity{./graphing/accuracyAndPrecision}
        \activity{./graphing/useForGraphs}
        \activity{./graphing/verticalLineTest}
    
    \chapterstyle
        \activity{./graphing/libraryOfFunctions}% Section 6: Library of Functions and Parent Functions
    \sectionstyle
        \activity{./graphing/parentFunctions}
        \practice{./graphing/Practice/parentFunctions-Practice1}
    /
    \chapterstyle
        \activity{./universalProperties/universalProperties}% Section 7: Universal Properties and Linear transformations of Graphs.
    \sectionstyle
        \activity{./universalProperties/geometricVsAnalyticViewpoints}
        \activity{./universalProperties/geometricView}
        \activity{./universalProperties/analyticView}
        
        \activity{./universalProperties/rigidTranslationsIntro}
        \activity{./universalProperties/rigidTranslationsGeometric}
        \practice{./universalProperties/Practice/rigidTranslationsGeometric-Practice1}
        \activity{./universalProperties/rigidTranslationsAnalytic}
        \practice{./universalProperties/Practice/rigidTranslationsAnalytic-Practice1}
        
        \activity{./universalProperties/transformations}
        \practice{./universalProperties/Practice/transformations-Practice1}
        \activity{./universalProperties/translationsAndTransformations}
        \practice{./universalProperties/Practice/translationsAndTransformations-Practice1}
        
        \activity{./universalProperties/pointsOfUniversalInterest}
        \activity{./universalProperties/algebraWithFunctions}
        \practice{./universalProperties/Practice/algebraOfFunctions-Practice1}
        \activity{./universalProperties/equalsSignsAreMagic}
        \activity{./universalProperties/oneAndZero}
        \activity{./universalProperties/invertibleFunctions}
        
        \activity{./universalProperties/Practice/sectionReview}
   
\newpage


% Part 5: Individual Function-Type exploration.
\part{Exploration of Functions}
    \chapterstyle
        \activity{./explorePolynomials/goals}
    \sectionstyle
        \activity{./explorePolynomials/terminologyAndNotation}
        \activity{./explorePolynomials/polynomialContext}
        \activity{./explorePolynomials/fundamentalTheoremOfAlgebra}
        \activity{./explorePolynomials/historicalPolynomialInterjection}
        \activity{./explorePolynomials/exponentsAndExtrema}
        \activity{./explorePolynomials/exponentsAndLocalExtrema}
        \activity{./explorePolynomials/curvature}
    \newpage
    
%    \chapterstyle
        \activity{./explorePolynomials/factoringOne}
%    \sectionstyle
        \practice{./explorePolynomials/Practice/factoringCoefficients-Practice1}
        \activity{./explorePolynomials/factoringACMethod}
        \practice{./explorePolynomials/Practice/factoringACMethod-Practice1}
        \activity{./explorePolynomials/factoringByGrouping}
        \practice{./explorePolynomials/Practice/factoringGroupingMethod-Practice1}
        \activity{./explorePolynomials/factoringSpecialForms}
        \practice{./explorePolynomials/Practice/factoringSpecialForms-Practice1}
        \activity{./explorePolynomials/completingTheSquare}
        \practice{./explorePolynomials/Practice/factoringCompletingTheSquare-Practice1}
        \activity{./explorePolynomials/polyLongDivision}
        \activity{./explorePolynomials/polySyntheticDivision}
        \practice{./explorePolynomials/Practice/factoringDivision-Practice1}
        \activity{./explorePolynomials/rationalRootTheorem}
        \practice{./explorePolynomials/Practice/factoringRationalRootTheorem-Practice1}
        \activity{./explorePolynomials/complexNumbers}
        \activity{./explorePolynomials/simplifyingComplexNumbers}
        \practice{./explorePolynomials/Practice/ComplexNumbers-Practice1}
        \practice{./explorePolynomials/Practice/ComplexNumbers-Practice2}
        \activity{./explorePolynomials/factoringQuadraticFormula}
    \newpage
    
    \chapterstyle
        \activity{./exploreRadicals/goals}
    \sectionstyle
        \activity{./exploreRadicals/terminologyAndNotation}
        \activity{./exploreRadicals/whyRadicals}
        \activity{./exploreRadicals/simplifyingNumericRadicals}
        \practice{./exploreRadicals/Practice/simplifyNumericRadicals-Practice1}
        \activity{./exploreRadicals/typesOfRadicands}
        \practice{./exploreRadicals/Practice/typesOfRadicands-Practice1}
        \activity{./exploreRadicals/squareRootAsInverseFunction}
        \activity{./exploreRadicals/solvingUnsimplifiedRadicals}
        \practice{./exploreRadicals/Practice/solvingUnsimplifiedRadicals-Practice1}
    \newpage
    
    \chapterstyle
        \activity{./exploreExponentials/goals}
    \sectionstyle
        \activity{./exploreExponentials/terminologyAndNotation}
        \activity{./exploreExponentials/exponentialReview}
        \practice{./exploreExponentials/Practice/simplifyNumericExponentials-Practice1}
        \activity{./exploreExponentials/propertiesOfExponentials}
        \practice{./exploreExponentials/Practice/propertiesOfExponentials-Practice1}
        \practice{./exploreExponentials/Practice/propertiesOfExponentials-Practice2}
        \activity{./exploreExponentials/propertiesOfExponentialFunctions}
        \practice{./exploreExponentials/Practice/solvingExponentialEquations-Practice1}
        \activity{./exploreExponentials/exponentialGrowthAndDecay}
    \newpage
    
    \chapterstyle
        \activity{./exploreLogarithms/goals}
    \sectionstyle
        \activity{./exploreLogarithms/terminologyAndNotation}
        \activity{./exploreLogarithms/notationAndIntroduction}
        \practice{./exploreLogarithms/Practice/simplifyNumericLogs-Practice1}
        \activity{./exploreLogarithms/propertiesOfLogs}
        \practice{./exploreLogarithms/Practice/propertiesOfLogs-Practice1}
        \activity{./exploreLogarithms/commonMistakes}
        \activity{./exploreLogarithms/changeOfBase}
        \activity{./exploreLogarithms/examplesOfLogs}
        \practice{./exploreLogarithms/Practice/solvingLogs-Practice1}
    \newpage
    
    \chapterstyle
        \activity{./explorePiecewise/goals}
    \sectionstyle
        \activity{./explorePiecewise/piecewiseGeometricViewpoint}
        \activity{./explorePiecewise/piecewiseAnalyticViewpoint}
        \activity{./explorePiecewise/piecewiseComputation}
    \newpage
    
    \chapterstyle
        \activity{./exploreAbsoluteValue/goals}
    \sectionstyle
        \activity{./exploreAbsoluteValue/absoluteValueGeometricViewpoint}
        \activity{./exploreAbsoluteValue/absoluteValueAnalyticViewpoint}
        \practice{./exploreAbsoluteValue/Practice/absoluteValueAnalyticViewpoint-Practice1}
        \activity{./exploreAbsoluteValue/absoluteValueSolvingEqualities}
    \newpage
    
    \chapterstyle
        \activity{./exploreRational/goals}
    \sectionstyle
%        \activity{./exploreRational/terminologyAndNotation}
%        \activity{./exploreRational/notationAndIntroduction}

\end{document}

%%%%%%%%%%%%%%%%%%%%
%%%% To-Do List %%%%
%%%%%%%%%%%%%%%%%%%%

- Add *significantly* more practice problems for students. Especially variety of problems
        I have recently requested all quiz problems from former TAs in an effort to accomplish this (1/12/2020)

- Add video lessons by skill.
        I have utilized and have easy access to the math dept lightboard to make quality videos as time permits.
        I have also posted discussion board requests for students to request specific videos to prioritize what they want most.
        
- Rewrite tiles to make them more readable.
        The tiles started as scattered class notes, and then more filled in class notes, but it was written more as a stream of consciousness than concise teaching.
        I have started rewriting the tiles as of 1/11/2020, and hope to finish rewriting (at least a major rewrite) by the end of this semester (Spring 2020).
        I hope video lessons will also soften this requirement as the videos will (hopefully) be a better teaching method than learning via the text itself.
        
- Add more interactive content
        I would like to build more interactive and elaborate Desmos lessons to embed. This seemed to work well with the Translations/Transformations tile, but could be useful elsewhere.

- Add more assessment stages
        Students seem to want more comprehensive assessment stages in the book. I am reading this as they want to take an exam-style assessment (in Xronos, not as actual large point value paper test) to assess if they have actually learned the previous material. A sort of ``Practice Exam'' but maybe not directly correlating to the actual exams (as they have those practice exams already)
        This seems to make sense to do something per section or something per ``part'' (in the LaTeX sense).
        It remains to determine if this should be an unlimited practice, or static. Ultimately I think unlimited would be ideal, but that is *much* harder to write, so it may benefit to just write up what I can statically to get it published faster for students to have *something* rather than trying to do it perfectly and not get them anything for another year.


%%%%%%%%%%%%%%%%%%%%
%%%% Known-Bugs %%%%
%%%%%%%%%%%%%%%%%%%%

- Sage has been hit or miss with server results. I believe I have solved the latest issue (1/12/2020) but I am hoping using a local sage server designed by Matt Carr will *significantly* alleviate this problem.
        Problems so far:    Updates on public servers don't mesh with MathJax and/or don't mesh with Sagetex, resulting in something that tests perfectly but fails to implement.
                            Public servers interpret huge numbers of access requests from UF as a DDOS attacks and/or as a computing farm attempt, so they throttle bandwidth or shut it down entirely making students unable to use the feature.

- Practice Tiles were meant to be not worth points.
        I wanted practice tiles to be optional so I could load a bunch of problems. As it stands, I can't really do that, which means every student has to do every problem. This causes a rift between lots of practice, and lots of busywork.
        Hopefully this issue will be resolved under the new deployment mechanism, but it is suppose to work now. Perhaps I can get Jim to help me configure it during JMM.
        
- I think Video completion is worth points?
        I have only sort of tested this; but I think you must fully watch a video to get the full credit for Xronos. 
        If this is the case, this could be a problem if I add a whole bunch of videos to Xronos as that would *drastically* increase the time it takes for students to complete the assignment, which isn't my intent. 
        I should try to talk ot Jim about this as well.



%%%%%%%%%%%%%%%%%%%
%%%% Changelog %%%%
%%%%%%%%%%%%%%%%%%%

1/11/2020           Rewrote tiles for sections 4 and 5. 
                    Broke apart many longer tiles, transitioned to less conversational style writing to make content less wordy.
                    Improved several explanations and added some graphics and new images.
                    
1/12/2020           






