\documentclass{ximeraXloud}
\input{../preamble}

\title{Piecewise Functions: The Analytic View}
\begin{document}
\begin{abstract}
    This section discusses the analytic view of piecewise functions.
\end{abstract}
\maketitle

Recall that the idea of piecewise functions is to stitch together several different functions, each one of which applies to a specific segment of the overall relation. Consider the graph from the previous section that depicted profit over time for a new company:

\begin{center}
    \begin{tikzpicture}
        \begin{axis}[
            axis x line=middle, 
            axis y line=middle, 
            minor tick num=1, 
            x label style={at={(axis description cs:1,0.1)},anchor=south},
            y label style={at={(axis description cs:0.05,1)},anchor=west},
            xlabel={Time (years)}, 
            ylabel={Profit (\$100,000)},
            xmin=-0.5, 
            xmax=10.5, 
            ymin=-0.5, 
            ymax=5
            ]
        \addplot[domain=0:2, samples=300]{1/2*x^2};
        \addplot[domain=2:4, samples=300]{2};
        \addplot[->,domain=4:10, samples=300]{(3*x-8)^(1/2)};
        \end{axis}
    \end{tikzpicture}
\end{center}

We can clearly break this graph into three pieces as so;

\begin{center}
    \begin{tikzpicture}
        \begin{axis}[
            axis x line=middle, 
            axis y line=middle, 
            minor tick num=1, 
            x label style={at={(axis description cs:1,0.1)},anchor=south},
            y label style={at={(axis description cs:0.05,1)},anchor=west},
            xlabel={Time (years)}, 
            ylabel={Profit (\$100,000)},
            xmin=-0.5, 
            xmax=10.5, 
            ymin=-0.5, 
            ymax=5
            ]
        \addplot[domain=0:2, samples=300, red]{1/2*x^2};
        \addplot[domain=2:4, samples=300, blue]{2};
        \addplot[->,domain=4:10, samples=300, green]{(3*x-8)^(1/2)};
        \end{axis}
    \end{tikzpicture}
\end{center}

Our goal is to write a (piecewise) function, $P(x)$, that is really a combination of each of these three functions. However, first we need to record the three functions themselves. In our case we have the following: 
\begin{description}
    \item[Red Segment:] {\color{red} $f(t) = \frac{1}{2}t^2$} is the segment from $t=0$ to $t=2$.
    \item[Blue Segment:] {\color{blue} $g(t) = 2$} is the segment from $t=2$ to $t=4$.
    \item[Green Segment:] {\color{green} $h(t) = \sqrt{3t-8}$} is the segment from $t=4$ to $t=10$.
\end{description}

We could simply use this descriptive method to define our piecewise function, but as usual we utilize a more compact form of notation. Consider the following piecewise function notation:

\[
    P(t) =
        \begin{cases}
            {\color{red}\frac{1}{2}t^2} & 0\leq t < 2 \\
            {\color{blue} 2} & 2 \leq t \leq 4 \\
            {\color{green} h(t) = \sqrt{3t-8}} & t > 4
        \end{cases}
\]

This notation is deceptively dense and it is worth breaking down each part of it. 
\begin{description}
    \item[P(x)] As normal we name the function as the dependent variable $P$ and the independent variable $t$ as the input.
    \item[Large Open Brace] The open brace encompasses all the lines that are all part of the piecewise definition. This way, if we have multiple horizontal lines of equations, we know that $P(t)$ is a piecewise function encompassing these three functions, and any other functions are not part of $P(x)$.
    \item[Lines with Functions] Each line with a function has a left and a right component. The left side, eg ``{\color{red}$\frac{1}{2}t^2$}", ``{\color{blue} $2$}", or ``{\color{green} $h(t) = \sqrt{3t-8}$}", each denote a function in a specific segment. The right component, eg ``$0\leq t < 2$", ``$2 \leq t \leq 4$", or ``$t > 4$" gives the domain of that segment.\\
    So, in particular, the line ``${\color{red}\frac{1}{2}t^2} \hspace{0.5cm} 0\leq t < 2$" is declaring that the function on the section of the graph for $t$ values between 0 and 2, is modeled by the function $\frac{1}{2}t^2$.
\end{description}

Something to keep in mind in the above function, is that the domains (the intervals of $t$ for each function) are disjoint, meaning that there \textit{should be no overlap between the intervals}. In particular, we have $0 \leq t < 2$ and $2 \leq t \leq 4$, rather than $0 \leq t \leq 2$ and $2 \leq t \leq 4$ which has $2$ in both intervals. In general, the chances of having a well defined function with overlapping intervals is almost nil. Having overlapping intervals in the function segments is how a piecewise relation fails the vertical line test. Consider from the geometric view tile the following piecewise relation graph:

\begin{center}
    \begin{tikzpicture}
        \begin{axis}[
            axis x line=middle, 
            axis y line=middle, 
            minor tick num=1, 
            x label style={at={(axis description cs:1,0.1)},anchor=south},
            y label style={at={(axis description cs:0.05,1)},anchor=west},
            xlabel={Time (years)}, 
            ylabel={Profit (\$100,000)},
            xmin=-0.5, 
            xmax=10.5, 
            ymin=-0.5, 
            ymax=5
            ]
        \addplot[domain=0:2, samples=300, red]{1/2*x^2};
        \addplot[domain=2:4, samples=300, blue]{2};
        \addplot[->,domain=3:10, samples=300, green]{(3*x-8)^(1/2)};
        \end{axis}
    \end{tikzpicture}
\end{center}

This graph is generated by the following piecewise relation:


\[
    P(t) =
        \begin{cases}
            {\color{red}\frac{1}{2}t^2} & 0\leq t < 2 \\
            {\color{blue} 2} & 2 \leq t \leq 4 \\
            {\color{green} h(t) = \sqrt{3t-8}} & t > 3
        \end{cases}
\]

Notice that the intervals above overlap because the second function is defined on $2 \leq t \leq 4$ and the third is defined on $t > 3$, which is why the piecewise relation fails the vertical line test on the interval $3 \leq t < 4$; ie on the overlap.


\end{document}