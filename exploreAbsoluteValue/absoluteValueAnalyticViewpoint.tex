\documentclass{ximeraXloud}
\input{../preamble}
\title{Absolute Value: Analytic View}
\begin{document}
\begin{abstract}
    This discusses the absolute value analytically, ie how to manipulate absolute values algebraically.
\end{abstract}
\maketitle

The most common algebraic description of absolute value that people know, tends to be that it is the thing that ``makes the value positive.'' Although this is not incorrect, it is a bit imprecise. In this section we discuss \textit{how} absolute values make something positive, and how one should go about algebraically manipulating absolute values.

Firstly, there is no magic wand in mathematics that simply makes anything positive (or, perhaps we should say, absolute value \textit{is} the magic wand... so if we want to dive into the mechanics, we need a different tool!) However, that doesn't mean that we don't have a mechanism at our disposal to make negative things positive. Consider, by way of a demonstration, $|x|$. Since we are working in the real numbers%
\footnote{This is actually \textit{really} important. If we were working in the complex numbers, absolute value behaves \textit{very} differently. So much so, in fact, that trying to use absolute value on a complex number will certainly go horribly astray without a much better grounding in complex numbers. For this reason, we even give it a different name, to distinguish how different it really is; we call it the \textit{modulus}.}
we know that $x$ is going to be positive, negative, or zero. If $x$ is positive or zero, the absolute value will do nothing; after all the absolute value of a positive number is just that number, and the absolute value of zero is zero. 

If, on the other hand, $x$ is a negative number, then the absolute value must change it to make it into a \textit{positive} number. So, the natural question to ask is, how do we make a negative number into a positive number? The answer: we multiply it by negative one! Using this, we can write the following piecewise definition of $|x|$.

\[
    |x| = 
        \begin{cases}
            -x & x < 0 \\
            x & x \geq 0
        \end{cases}
\]

At first glance it may seem rather counterintuitive that the absolute value appears to be placing a negative sign in front of $x$. After all, absolute value is suppose to only have positive output. But the key observation is that the negative sign is only being placed in front of $x$ \textit{when $x$ is already a negative number}. So even though $-x$ \textit{looks} like it will be negative, it will actually always give a positive value because we've restricted it to values of $x$ that are already negative, so we're taking a negative of a negative, which is positive!

Another important observation is that the negative is being added in by the absolute value, not by the $x$. This means that, if we have a more complicated expression, we can end up with a more complicated looking piecewise function. Consider, for example, $3 + |x - 1|$. We write this in the piecewise form by doing the same thing, except now, it's not whether or not $x$ is negative that the absolute value cares about, it's whether or not $x-1$ is negative that the absolute value cares about. So we get an initial piecewise function like the following:

\[
    3 + |x-1| = 
        \begin{cases}
            3 + -(x-1) & (x-1) < 0 \\
            3 + (x-1) & (x-1) \geq 0
        \end{cases}
\]

Notice that the \textit{content} of the absolute value is what appears both in front of the negative \textbf{and} in the domain description. We should simplify this by solving the inequalities for $x$ and simplifying the associated functions, which would mean we would rewrite the piecewise function into the following:

\[
    3 + |x-1| = 
        \begin{cases}
            4 - x & x < 1 \\
            2 + x & x \geq 1
        \end{cases}
\]



\end{document}