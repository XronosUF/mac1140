\documentclass{ximeraXloud}
\usepackage{longdivision}
\usepackage{polynom}
\usepackage{float}% Use `H' as the figure optional argument to force it's vertical placement to conform to source.
%\usepackage{caption}% Allows us to describe the figures without having "figure 1:" in it. :: Apparently Caption isn't supported.
%    \captionsetup{labelformat=empty}% Actually does the figure configuration stated above.
\usetikzlibrary{arrows.meta,arrows}% Allow nicer arrow heads for tikz.
\usepackage{gensymb, pgfplots}
\usepackage{tabularx}
\usepackage{arydshln}



\graphicspath{
  {./}
  {./explorePolynomials/}
  {./exploreRadicals/}
  {./graphing/}
}

%% Default style for tikZ
\pgfplotsset{my style/.append style={axis x line=middle, axis y line=
middle, xlabel={$x$}, ylabel={$y$}, axis equal }}


%% Because log being natural log is too hard for people.
\let\logOld\log% Keep the old \log definition, just in case we need it.
\renewcommand{\log}{\ln}


%%% Changes in polynom to show the zero coefficient terms
\makeatletter
\def\pld@CF@loop#1+{%
    \ifx\relax#1\else
        \begingroup
          \pld@AccuSetX11%
          \def\pld@frac{{}{}}\let\pld@symbols\@empty\let\pld@vars\@empty
          \pld@false
          #1%
          \let\pld@temp\@empty
          \pld@AccuIfOne{}{\pld@AccuGet\pld@temp
                            \edef\pld@temp{\noexpand\pld@R\pld@temp}}%
           \pld@if \pld@Extend\pld@temp{\expandafter\pld@F\pld@frac}\fi
           \expandafter\pld@CF@loop@\pld@symbols\relax\@empty
           \expandafter\pld@CF@loop@\pld@vars\relax\@empty
           \ifx\@empty\pld@temp
               \def\pld@temp{\pld@R11}%
           \fi
          \global\let\@gtempa\pld@temp
        \endgroup
        \ifx\@empty\@gtempa\else
            \pld@ExtendPoly\pld@tempoly\@gtempa
        \fi
        \expandafter\pld@CF@loop
    \fi}
\def\pld@CMAddToTempoly{%
    \pld@AccuGet\pld@temp\edef\pld@temp{\noexpand\pld@R\pld@temp}%
    \pld@CondenseMonomials\pld@false\pld@symbols
    \ifx\pld@symbols\@empty \else
        \pld@ExtendPoly\pld@temp\pld@symbols
    \fi
    \ifx\pld@temp\@empty \else
        \pld@if
            \expandafter\pld@IfSum\expandafter{\pld@temp}%
                {\expandafter\def\expandafter\pld@temp\expandafter
                    {\expandafter\pld@F\expandafter{\pld@temp}{}}}%
                {}%
        \fi
        \pld@ExtendPoly\pld@tempoly\pld@temp
        \pld@Extend\pld@tempoly{\pld@monom}%
    \fi}
\makeatother




%%%%% Code for making prime factor trees for numbers, taken from user Qrrbrbirlbel at: https://tex.stackexchange.com/questions/131689/how-to-automatically-draw-tree-diagram-of-prime-factorization-with-latex

\usepackage{forest,mathtools,siunitx}
\makeatletter
\def\ifNum#1{\ifnum#1\relax
  \expandafter\pgfutil@firstoftwo\else
  \expandafter\pgfutil@secondoftwo\fi}
\forestset{
  num content/.style={
    delay={
      content/.expanded={\noexpand\num{\forestoption{content}}}}},
  pt@prime/.style={draw, circle},
  pt@start/.style={},
  pt@normal/.style={},
  start primeTree/.style={%
    /utils/exec=%
      % \pt@start holds the current minimum factor, we'll start with 2
      \def\pt@start{2}%
      % \pt@result will hold the to-be-typeset factorization, we'll start with
      % \pgfutil@gobble since we don't want a initial \times
      \let\pt@result\pgfutil@gobble
      % \pt@start@cnt holds the number of ^factors for the current factor
      \def\pt@start@cnt{0}%
      % \pt@lStart will later hold "l"ast factor used
      \let\pt@lStart\pgfutil@empty,
    alias=pt-start,
    pt@start/.try,
    delay={content/.expanded={$\noexpand\num{\forestove{content}}
                            \noexpand\mathrlap{{}= \noexpand\pt@result}$}},
    primeTree},
  primeTree/.code=%
    % take the content of the node and save it in the count
    \c@pgf@counta\forestove{content}\relax
    % if it's 2 we're already finished with the factorization
    \ifNum{\c@pgf@counta=2}{%
      % add the factor
      \pt@addfactor{2}%
      % finalize the factorization of the result
      \pt@addfactor{}%
      % and set the style to the prime style
      \forestset{pt@prime/.try}%
    }{%
      % this simply calculates content/2 and saves it in \pt@end
      % this is later used for an early break of the recursion since no factor
      % can be greater then content/2 (for integers of course)
      \edef\pt@content{\the\c@pgf@counta}%
      \divide\c@pgf@counta2\relax
      \advance\c@pgf@counta1\relax % to be on the safe side
      \edef\pt@end{\the\c@pgf@counta}%
      \pt@do}}

%%% our main "function"
\def\pt@do{%
  % let's test if the current factor is already greather then the max factor
  \ifNum{\pt@end<\pt@start}{%
    % great, we're finished, the same as above
    \expandafter\pt@addfactor\expandafter{\pt@content}%
    \pt@addfactor{}%
    \def\pt@next{\forestset{pt@prime/.try}}%
  }{%
    % this calculates int(content/factor)*factor
    % if factor is a factor of content (without remainder), the result will
    % equal content. The int(content/factor) is saved in \pgf@temp.
    \c@pgf@counta\pt@content\relax
    \divide\c@pgf@counta\pt@start\relax
    \edef\pgf@temp{\the\c@pgf@counta}%
    \multiply\c@pgf@counta\pt@start\relax
    \ifNum{\the\c@pgf@counta=\pt@content}{%
      % yeah, we found a factor, add it to the result and ...
      \expandafter\pt@addfactor\expandafter{\pt@start}%
      % ... add the factor as the first child with style pt@prime
      % and the result of int(content/factor) as another child.
      \edef\pt@next{\noexpand\forestset{%
        append={[\pt@start, pt@prime/.try]},
        append={[\pgf@temp, pt@normal/.try]},
        % forest is complex, this makes sure that for the second child, the
        % primeTree style is not executed too early (there must be a better way).
        delay={
          for descendants={
            delay={if n'=1{primeTree, num content}{}}}}}}%
    }{%
      % Alright this is not a factor, let's get the next factor
      \ifNum{\pt@start=2}{%
        % if the previous factor was 2, the next one will be 3
        \def\pt@start{3}%
      }{%
        % hmm, the previos factor was not 2,
        % let's add 2, maybe we'll hit the next prime number
        % and maybe a factor
        \c@pgf@counta\pt@start
        \advance\c@pgf@counta2\relax
        \edef\pt@start{\the\c@pgf@counta}%
      }%
      % let's do that again
      \let\pt@next\pt@do
    }%
  }%
  \pt@next
}

%%% this builds the \pt@result macro with the factors
\def\pt@addfactor#1{%
  \def\pgf@tempa{#1}%
  % is it the same factor as the previous one
  \ifx\pgf@tempa\pt@lStart
    % add 1 to the counter
    \c@pgf@counta\pt@start@cnt\relax
    \advance\c@pgf@counta1\relax
    \edef\pt@start@cnt{\the\c@pgf@counta}%
  \else
    % a new factor! Add the previous one to the product of factors
    \ifx\pt@lStart\pgfutil@empty\else
      % as long as there actually is one, the \ifnum makes sure we do not add ^1
      \edef\pgf@tempa{\noexpand\num{\pt@lStart}\ifnum\pt@start@cnt>1 
                                           ^{\noexpand\num{\pt@start@cnt}}\fi}%
      \expandafter\pt@addfactor@\expandafter{\pgf@tempa}%
    \fi
    % setup the macros for the next round
    \def\pt@lStart{#1}% <- current (new) factor
    \def\pt@start@cnt{1}% <- first time
  \fi
}
%%% This simply appends "\times #1" to \pt@result, with etoolbox this would be
%%% \appto\pt@result{\times#1}
\def\pt@addfactor@#1{%
  \expandafter\def\expandafter\pt@result\expandafter{\pt@result \times #1}}

%%% Our main macro:
%%% #1 = possible optional argument for forest (can be tikz too)
%%% #2 = the number to factorize
\newcommand*{\PrimeTree}[2][]{%
  \begin{forest}%
    % as the result is set via \mathrlap it doesn't update the bounding box
    % let's fix this:
    tikz={execute at end scope={\pgfmathparse{width("${}=\pt@result$")}%
                         \path ([xshift=\pgfmathresult pt]pt-start.east);}},
    % other optional arguments
    #1
    % And go!
    [#2, start primeTree]
  \end{forest}}
\makeatother


\providecommand\tabitem{\makebox[1em][r]{\textbullet~}}
\providecommand{\letterPlus}{\makebox[0pt][l]{$+$}}
\providecommand{\letterMinus}{\makebox[0pt][l]{$-$}}



\title{Absolute Value: Solving Equalities}
\begin{document}
\begin{abstract}
    This section is on how to solve absolute value equalities.
\end{abstract}
\maketitle

It's one thing to understand the analytic definition of an absolute value, but it can be remarkably tricky at times to understand how to utilize that definition to solve an equality that has an absolute value in it. This is especially true given the subtleties involved in the solution. Consider the following equation:

\[
    |x-1| = 13
\]

The first step to solving any absolute value equality is to break it into cases, according to the piecewise definition. In this case, we could rewrite the absolute value part as follows:

\[
    |x-1| = 
        \begin{cases}
            -x + 1 & x < 1 \\
            x - 1 & x \geq 1
        \end{cases}
\]

We then plug these two possibilities into the original equality in, but \textbf{we must keep track of the domain restriction}. 
\begin{description}
    \item[Case 1: (Assume $x < 1$)] Under this assumption we have that \hbox{$|x-1| = -(x-1)$}, so our original equality becomes $-(x-1) = 13$ which we can solve to get $x = -12$.
    \item[Case 2: (Assume $x \geq 1$)] Under this assumption we have that $|x-1| = x-1$, so our original equality becomes $x-1 = 13$ which we can solve to get $x = 14$.
\end{description}
Finally, we want to verify that the solutions we came up with are in the correct domains. In our example, we have that $x = -12$ was a solution \textbf{under the requirement that $x < 1$}. Since $-12 < 1$, this is a valid solution. Similarly, the solution $x = 14$ is only a solution under the requirement that $x \geq 1$. Since $14 > 1$, this is also a valid solution. \textbf{If in doubt, you can always plug your solutions back into the original problem to verify that they work!}

Now let's consider another example.
\[
    |3x + 3| = -x^2 + 1
\]

Again, we want to take the content of the absolute value and replace it with the piecewise definition. In this case we have:

\[
    |3x+3| = 
        \begin{cases}
            -3x - 3 & x < - 1 \\
            3x + 3 & x \geq - 1
        \end{cases}
\]
Thus we break our original equality into two cases, just as before:

\begin{description}
    \item[Case 1: (Assume $x < -1$)] Under this assumption we have that $|3x + 3| = -3x - 3$. So our original equality becomes $-3x - 3 = -x^2 + 1$ which we can rearrange and factor to get $(x -4)(x + 1) = 0$ which gives the solutions $x = 4$ and $x = -1$. 
    \item[Case 2: (Assume $x \geq -1$)] Under this assumption we have that $|3x + 3| = 3x + 3$. So our original equality becomes $3x + 3 = -x^2 + 1$ which we can rearrange and factor to get $(x + 2)(x + 1) = 0$ which gives the solutions $x = -2$ and $x = -1$.
\end{description}

Thus we have the two solutions $x = 4$ and $x = -1$ \textbf{under the assumption $x < -1$}. Since neither of these solutions are (strictly) less than $-1$, neither of these solutions are valid.

Next we have the two solutions $x = -1$ and $x = -2$ \textbf{under the assumption $x \geq -1$}. Since $-2$ is not greater than or equal to $-1$, that solution is not valid. However, the solution $x = -1$ is valid here.

Notice that $-1$ was a solution \textit{in both cases} and, although $-1$ was not valid in the first case, it \textit{is} valid in the second case, which is why it \textit{is} a valid solution.

Again, if this is somewhat difficult to parse, remember that you can always plug in all the potential solutions into your original equation to see which ones work. For example;

\begin{tabularx}{\textwidth}{lllll}
    Solution 1: $x = 4$     & $|3(4) + 3|$  & $= |12 + 3|$  & $= 15 \neq -15$   & $= -(4^2) + 1$\\
    Solution 2: $x = -1$    & $|3(-1) + 3|$ & $= |0|$       & $= 0 $            & $= -(1^2) + 1$\\
    Solution 3: $x = -2$    & $|3(-2) + 3|$ & $= |-6 + 3|$  & $= 3 \neq -3$     & $= -((-2)^2) + 1$
\end{tabularx}

The only one that satisfied our equation is the second solution, $x = -1$, as we saw earlier.

\end{document}