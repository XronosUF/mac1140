\documentclass{ximeraXloud}
\input{../../preamble}
\title{Absolute Value: Analytic View Practice 1}
\begin{document}
\begin{abstract}
    This gives practice for understanding the absolute value analytically.
\end{abstract}
\maketitle

\input{Useful-Sage-Macros}
\begin{sagesilent}
#### Problem p1
p1c1 = NonZeroInt(-5,5)
p1c2 = RandInt(-5,5)
p1c3 = RandInt(-5,5)
p1c4 = RandInt(-5,5)

p1f1 = p1c1*x - p1c2
p1f2 = p1c3*x - p1c4

if p1c1 > 0:
    p1ans1 = -p1f1 + p1f2
    p1ans2 = p1f1 + p1f2
else:
    p1ans1 = p1f1 + p1f2
    p1ans2 = -p1f1 + p1f2
p1ans3 = p1c2/p1c1


#### Problem p2
p2c1 = NonZeroInt(-5,5)
p2c2 = RandInt(-5,5)
p2c3 = RandInt(-5,5)
p2c4 = RandInt(-5,5)

p2f1 = p2c1*x - p2c2
p2f2 = p2c3*x - p2c4

if p2c1 > 0:
    p2ans1 = -p2f1 + p2f2
    p2ans2 = p2f1 + p2f2
else:
    p2ans1 = p2f1 + p2f2
    p2ans2 = -p2f1 + p2f2
p2ans3 = p2c2/p2c1


#### Problem p3
p3c1 = NonZeroInt(-5,5)
p3c2 = RandInt(-5,5)
p3c3 = RandInt(-5,5)
p3c4 = RandInt(-5,5)

p3f1 = p3c1*x - p3c2
p3f2 = p3c3*x - p3c4

if p3c1 > 0:
    p3ans1 = -p3f1 + p3f2
    p3ans2 = p3f1 + p3f2
else:
    p3ans1 = p3f1 + p3f2
    p3ans2 = -p3f1 + p3f2
p3ans3 = p3c2/p3c1


#### Problem p4
p4c1 = NonZeroInt(-5,5)
p4c2 = RandInt(-5,5)
p4c3 = RandInt(-5,5)
p4c4 = RandInt(-5,5)

p4f1 = p4c1*x - p4c2
p4f2 = p4c3*x - p4c4

if p4c1 > 0:
    p4ans1 = -p4f1 + p4f2
    p4ans2 = p4f1 + p4f2
else:
    p4ans1 = p4f1 + p4f2
    p4ans2 = -p4f1 + p4f2
p4ans3 = p4c2/p4c1


\end{sagesilent}

\begin{problem}
    Consider the following absolute value expression:
    \[
        |\sage{p1f1}| + \sage{p1f2}
    \]
    
    Fill in the missing pieces of the following piecewise definition
    
    \[
        |\sage{p1f1}| + \sage{p1f2} = 
            \begin{cases}
                \answer{\sage{p1ans1}} & x < \answer{\sage{p1ans3}} \\
                \answer{\sage{p1ans2}} & x \geq \answer{\sage{p1ans3}}
            \end{cases}
    \]
    
\end{problem}


\begin{problem}
    Consider the following absolute value expression:
    \[
        |\sage{p2f1}| + \sage{p2f2}
    \]
    
    Fill in the missing pieces of the following piecewise definition
    
    \[
        |\sage{p2f1}| + \sage{p2f2} = 
            \begin{cases}
                \answer{\sage{p2ans1}} & x < \answer{\sage{p2ans3}} \\
                \answer{\sage{p2ans2}} & x \geq \answer{\sage{p2ans3}}
            \end{cases}
    \]
    
\end{problem}


\begin{problem}
    Consider the following absolute value expression:
    \[
        |\sage{p3f1}| + \sage{p3f2}
    \]
    
    Fill in the missing pieces of the following piecewise definition
    
    \[
        |\sage{p3f1}| + \sage{p3f2} = 
            \begin{cases}
                \answer{\sage{p3ans1}} & x < \answer{\sage{p3ans3}} \\
                \answer{\sage{p3ans2}} & x \geq \answer{\sage{p3ans3}}
            \end{cases}
    \]
    
\end{problem}


\begin{problem}
    Consider the following absolute value expression:
    \[
        |\sage{p4f1}| + \sage{p4f2}
    \]
    
    Fill in the missing pieces of the following piecewise definition
    
    \[
        |\sage{p4f1}| + \sage{p4f2} = 
            \begin{cases}
                \answer{\sage{p4ans1}} & x < \answer{\sage{p4ans3}} \\
                \answer{\sage{p4ans2}} & x \geq \answer{\sage{p4ans3}}
            \end{cases}
    \]
    
\end{problem}





\end{document}