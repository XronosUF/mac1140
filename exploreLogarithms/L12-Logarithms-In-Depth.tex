\Lecture{An in-depth investigation of Logarithms}
\injectBox{Terminology To Know}{%
    \newTerm{Exponential Form of a Logarithm}{%
        The exponential form of a logarithm is the exponential equality the corresponds to a logarithmic equality.\\
        \textbf{For Example:} The logarithmic equality $\log_4(256) = x$ has, as an exponential form, $4^x = 256$.\\
        \textbf{In general:} For a logarithmic equality $\log_b(c) = a$, the exponential form is $b^a = c$.
        }%
    \newTerm{Base (of a logarithm}{%
        The base of a logarithm is the value of the base of the exponential that the logarithm is the inverse of.\\
        \textbf{For Example:} Log with a base of $4$ is written $\log_4$ and is the function that represents the inverse of the exponential with the base 4, eg $4^x$. Thus $\log_4(4^x) = x$.
        }%
    \newTerm{Argument (of a log)}{%
        The argument of a log is the contents inside the log, which is the value of the exponential that you are trying to invert.\\
        \textbf{For Example:} The expression $\log_4(256)$ has an argument of $256$.
        }%
    }%

\injectBox{Questions To Answer}{%
    What role do logarithms play mechanically?\\
    Where do logs come from?\\
    Where do we see logs come up ``naturally"?\\
    What are the properties of logs and why do we care?
    }%

\ifcompletedNotes
\injectBox{Goal for Lecture Content}{%
    Hammer home the log properties for logarithmic differentiation.\\
    Explain a way for students to remember that it's $\log(ab) = \log(a) + \log(b)$ and not $\log(a+b)=\log(a)\log(b)$.\\
    }%
\fi

Logarithms are perhaps the most artificial feeling function in our library of functions. If we remember, exponential functions were one-to-one (passing the vertical \textit{and} horizontal line tests) and are thus invertible. Logarithms are precisely the inverse functions that correspond to the exponential functions of our last topic.

\subsection{Notation and how to avoid it!}

    It turns out that logarithms are extremely useful in many unexpected ways.%
    \footnote{Indeed, there are large portions of differential and integral calculus that exploit the properties of logs to get around computationally unpleasant, or even impossible, problems to do them in a relatively easy and straightforward manner.}
    Nonetheless, until we are comfortable with logarithmic notation, it can often be helpful to translate a logarithm from it's log notation to the corresponding exponential function that they are the inverses of. The notation $\log_b(a) = v$ would be stated as ``log base $b$ of $a$ is equal to $v$". The corresponding exponential form for this equation would then be $b^v = a$. In particular, $f^{-1}(x) = \log_b(x)$ is a function that is the inverse function of the exponential $f(x)=b^x$. Thus, utilizing the inverse function property,%
    \footnote{Recall that the inverse function property says that $f^{-1}(f(x)) = f(f^{-1}(x)) = x$}
    we have;
    \[
        \log_b(b^x) = b^{\log_b(x)} = x
    \]
    This can also be used to rewrite any log in an equivalent exponential form. In general we have;
    \[
        \log_b(a)=v \iff b^v = a
    \]
    In some sense, this motivates why we say ``log base $b$", as that ``base" is the same ``base" of the exponential that the log is the inverse of. Again, our goal is to get comfortable with the logarithmic notation itself, but until we reach that point, it will often be helpful to consider the exponential form of an expression in order to ``see" what is happening during a simplification step, or why a property works.

\subsection{Properties of Logarithms}

    There are a number of properties of logarithms. We will start by showing the ``primary" three properties, that is to say the three properties that are most often used, but we will also include many more properties that are occasionally useful and easily deduced from the primary ones.
    
    \subsubsection{Primary Properties}
        
        We begin by listing the properties and then we will address each one to show why they work.
        
        \begin{enumerate}
            \item $\log_b(xy) = \log_b(x) + \log_b(y)$
            \item $\log_b(x^y) = y\log_b(x)$
            \item $\log_b\left(\frac{x}{y}\right) = \log_b(x) - \log_b(y)$
        \end{enumerate}
        
        \subsubsubsection{Showing $\log_b(xy) = \log_b(x) + \log_b(y)$}
            
            First, let's consider the right-hand side, $\log_b(x) + \log_b(y)$. Remember that, for any $z$ we can write $\log_b(b^z) = z$, specifically (taking $z$ to be the right-hand side above) we have; $\log_b(x) + \log_b(y) = \log_b\left( b^{\log_b(x) + \log_b(y)} \right)$. Thus we can write the following:
            \[
                \log_b(x) + \log_b(y)   = \log_b\left( b^{\log_b(x) + \log_b(y)} \right) 
                                        = \log_b\left( b^{\log_b(x)}\cdot b^{\log_b(y)}\right) 
                                        = \log_b(x \cdot y).
            \]
            In essence, what is happening above is a consequence of the fact that everything inside the $\log_b$ function (the argument of the log) is occurring \textit{after} the exponent is canceled by the log, and everything that happens outside of the log function, is happening \textit{before} the exponent is canceled. Thus the sum of logs, ie $\log_b(x) + \log_b(y)$, because the addition symbol is \textit{outside} the logs argument, is happening \textit{inside} the exponent. In contrast, the product of the argument, ie $\log_b(xy)$ is happening \textit{inside} the argument, so it is happening \textit{after} the exponent. Thus the property above is telling us that addition inside an exponent is the same as the product outside the exponent... which means the this property is the logarithmic version of the \textit{exponential} property $b^{x+y} = b^xb^y$.
            
        \subsubsubsection{Showing $\log_b(x^y) = y\log_b(x)$}
            This property can be seen to easily follow from the previous property. Indeed, if you recall that exponentials are just repeated multiplication, then the left hand side is ``$y$" products of $x$ which, by the first property above, is equivalent to ``$y$" additions of $\log_b(x)$, which is simply $y\log_b(x)$. This may be easier to see in a concrete example, so let us consider the example $\log_b (x^3)$. Then;
            \[
                \log_b(x^3) = \log_b(x\cdot x\cdot x) = \log_b(x) + \log_b(x) + \log_b(x) = 3\log_b(x)
            \]
        
        \subsubsubsection{Showing $\log_b\left(\frac{x}{y}\right) = \log_b(x) - \log_b(y)$}
            This property once again follows from the previous two rather directly. Recall that we can write $\frac{1}{y}$ as $y^{-1}$. So we may rewrite the above log as $\log_b\left(\frac{x}{y}\right) = \log_b\left(x \cdot \frac{1}{y}\right) = \log_b\left(x\cdot y^{-1}\right)$. By using this and the above two properties we have;
            \[
                \log_b\left(\frac{x}{y}\right)  = \log_b\left(x\cdot y^{-1}\right) 
                                                = \log_b(x) + \log_b\left(y^{-1}\right) 
                                                = \log_b(x) + (-1)\log_b(y)
                                                = \log_b(x) - \log_b(y) 
            \]
            Again, this property can be understood in the light of the exponential form as well. The subtraction \textit{outside} the logs is happening \textit{inside} the exponent, whereas the division \textit{inside} the logs is happening \textit{outside} the exponent. So this property is equivalent to the \textit{exponential} property of $b^{x-y}=\frac{b^x}{b^y}$.
    \subsubsection{Secondary Properties}
        The properties below are occasionally useful, but much like the second and third properties above, they are easily deduced/derived from the already existing properties. For this reason I would generally not bother memorizing these properties, rather I would suggest `learning' these as they are easy to reproduce if you don't remember the exact formulas, as long as you understand where the formulas came from. Obviously this is up to the reader however.
        
        As before, we start by listing the properties, and then we will show where each comes from afterward.
        
        \begin{enumerate}
            \item $\log_b(1) = 0$
            \item $\log_b(b) = 1$
            \item $\log_b\left(x^{-1}\right) = \log_b\left(\frac{1}{x}\right) = -\log_b(x)$
        \end{enumerate}
        
        The proof for each of the above is easily seen as just a specific application of the primary property $\log_b\left(x^y\right) = y\log_b(x)$ above. Specifically for each of the above properties the ``proof" is as follows:
        
        \begin{enumerate}
            \item Use $y=0$ to see that $\log_b(1) = \log_b\left(x^0\right) = 0 \cdot \log_b(x) = 0$.%
            \footnote{Notice that the last equality here follows as $\log_b(x)$ is some finite number, so multiplying it by zero annihilates it to zero.}
            \item Use $x = b$ and $y = 1$ and recall the inverse function property to see that $\log_b\left(b^1\right) = 1$.
            \item Use $y = -1$ to immediately see that $\log_b\left(x^{-1}\right) = (-1)\cdot \log_b (x) = -\log_b(x)$.
        \end{enumerate}
        
        A curious student may ask why the above three are even considered ``properties". Technically these \textit{are} properties in the sense that they are (valid) features of logs, however they are often included in the list of properties because of how often they show up in work. Nonetheless in my experience, labeling these, along with the less obvious properties above (the so-called ``primary" properties) often causes students to mix up features of various properties which generates added confusion.
        
\subsection{Non-Properties of Logarithms... aka Common Mistakes of Logs}
    
    You may be wondering why I would bother including ``non-properties" of logs, but as it turns out the properties of logs are reversed with incredible frequency and is hands-down the most common error students make when trying to deal with logs.
    
    When simplifying logs, you would do well to consider the log as you consider roots when you try to simplify those. In fact, the processes are nearly identical, the primary difference being that with roots you need a certain multiplicity of a factor, whereas with logarithms you do not. Just like with roots, we can only ``break apart" aka ``simplify" aka ``expand" a log if \textit{all factors inside are multiplied together}, that is to say, the logarithm's argument must be in a \textit{fully factored form} before it can be simplified. Consider the following example.
    
    \example{Expand the expression $\log_b\left(\dfrac{x^3z^2(x-1)^4(x+1)}{(x-1)^2(z-1)^2}\right)$.}{%
        Using our log properties above (often referred to as the `log rules') we can immediately split this log apart into it's individual pieces. I will show an exhaustive number of steps, but showing all these steps is not necessary. However, keep in mind that the fewer steps you show, the less partial credit you can earn on graded work, and the more likely you are to make an error.
        
        \begin{center}
        \renewcommand{\arraystretch}{1.5}
            \begin{tabular}{ll}
                $\log_b\left(\dfrac{x^3z^2(x-1)^4(x+1)}{(x-1)^2(z-1)^2}\right)$ &= $\log_b\left(x^3z^2(x-1)^4(x+1)\right) - \log_b\left((x-1)^2(z-1)^2\right)$ \\
                    & $= \left[ \log_b\left(x^3\right) + \log_b\left(z^2\right) + \log_b\left((x-1)^4\right) + \log_b(x+1)\right]$ \\
                        & \hspace*{1cm}$-\left[ \log_b\left((x-1)^2\right) + \log_b\left((z-1)^2\right)\right]$ \\
                    & $= (3)\log_b(x) + (2)\log_b(z) + (4)\log_b(x-1) + \log_b(x+1)$ \\
                        & \hspace*{1cm}$ - \left[ (2)\log_b(x-1) + (2)\log_b(z-1)\right]$\\
                    & $= 3\log_b(x) + 2\log_b(z) + 2\log_b(x-1) + \log_b(x+1) - 2\log_b(z-1)$
            \end{tabular}
        \end{center}
        
        }% End Example
    The above example works well because everything was already in a fully factored form. However, if you were asked to expand the expression $\log_b(x^2-1)$ for example, you cannot do so (at least, not without factoring first). \textit{By far} the most common mistake made by students with log properties, is that they remember there is a link between addition and multiplication, and between division and subtraction, but \textit{they don't remember which direction the property goes}. Thus, it is very common to see, for example, with the example $\log_b(x^2-1)$ to be ``simplified" as $\frac{\log_b(x^2)}{\log_b(1)}$, because the student remembers the link between subtraction and division, but this is clearly not a valid simplification... after all, $\log_b(1) = 0$ so that simplification is dividing by zero! So almost as important as remembering the properties of logs \textit{is remembering the direction that the property goes}, that is to say, multiplication or division \textit{inside} the log becomes addition or subtraction \textit{outside} the log.\\
    
    Another way to know if something is simplifiable or not, is to consider the argument of the log as if it were an argument of a root. Just like roots, you can think of arguments of logs as having (the same) ``type 1" and ``type 2" arguments. If there are any addition or subtraction of terms in the argument of a log (ie the argument is not in fully factored form) then you cannot apply any of the log rules. If the argument of the log has all products or division (ie the argument is in fully factored form), \textit{then} you may apply log rules to expand it.
    
    \subsubsection{Change of Base formula}
        
        The previous examples work nicely as long as we have logs with matching bases. In the unfortunate (but likely) situation where the bases don't match, we will need a way to manipulate the bases so that we can use our above properties. This leads us to the `change of base' formula for logarithms.
        
        \example{Write the logarithm $\log_2 x^2$ as a log with a base of $5$.}{%
            We want to rewrite the log above as $\log_5($something$)$. Perhaps the easiest way to do this involves using the exponential form of the log, that is, if $\log_2(x^2) = y$, then $x^2 = 2^y$. We know we want log base 5, so we can apply this to both sides; getting $\log_5(x^2) = \log_5(2^y) = y\log_5(2)$. Remember that $y$ is actually our original expression, so if we divide both sides by $\log_5(2)$ to solve for $y$ we get;
            \[
                \log_2(x^2) = y = \frac{\log_5 \left(x^2\right)}{\log_5(2)}
            \]
            Which means we have successfully written our original expression as an expression involving only $\log_5$ logs.
        }% End of Example
        If we generalize the previous example, we can actually derive the fully generalized change of base formula. If you want to change the base from $b$ to the base $c$, then you can use the following equation;
        
        \[
            \log_b(x) = \frac{\log_c(x)}{\log_c(b)}
        \]
        
\subsection{Ok, so we have all these log rules. Now what do we do with them?}
     
    There are two primary (mechanical) uses of logs. The most common usage is to convert a large product of things into a sum of things, but this usage primarily occurs in calculus. For our course, the primary usage will be to expand, simplify, and/or condense logs. This is best demonstrated with some examples.
    
    \example{Fully expand the expression $\log_2(4x^2(y+1)^3) - \log_2\left(\dfrac{x^3+4x^2-5x}{x^2-4}\right)$}{%
        Recall that we can only apply log rules when the argument is completely factored, so the very first step is to factor the arguments. Once we have done so, we have the following;
        \[
            \log_2\left(4x^2(y+1)^3\right) - \log_2\left(\frac{x(x-1)(x+5)}{(x-2)(x+2)}\right)
        \]
        Next we want to fully expand the above expression using our log rules, this will also help us find cancellations. Doing so yields
        \[
            \log_2(4) + \log_2(x^2) + \log_2((y+1)^3) - 
                \left[ \log_2(x-1) + \log_2(x+5) + \log_2(x) - \log_2(x-2) - \log_2(x + 2) \right]
        \]
        
        Note that $\log_2(4) = 2$, and we will use the log rules to remove powers, to get;
        
        \[
            = 2 + 2\log_2(x) + 3\log_2(y+1) - \log_2(x-1) - \log_2(x+5) - \log_2(x) + \log_2(x-2) + \log_2(x+2)
        \]
        
        Finally, we can combine like terms.
        \[
            = 2 + \log_2(x) + 3\log_2(y+1) + \log_2(x-2) + \log_2(x+2) - \log_2(x-1) - \log_2(x+5)
        \]
        
    }% End of Example
    
    \example{Simplify the expression 
    \[
        \log\left( \dfrac{(30x^3 - 60x^2 + 30x)yz}{(y+1)(z-1)} \right) + \log\left( \dfrac{(y^2-1)(z^2-1)}{(x^3 - 3x^2 + 3x - 1)z} \right)
    \]
    }{%
        As before, we will begin by expanding each of the logs fully. Also like before, we must fully factor the argument of the log before we can apply the log rules to expand it. For the sake of clarity I will factor and then simplify the first log on it's own, and then do the same with the second log on it's own.\\
        \renewcommand{\arraystretch}{1.5}
        
        \begin{tabular}{ll}
            $\log\left( \dfrac{(30x^3 - 60x^2 + 30x)yz}{(y+1)(z-1)} \right)$ & $= \log\left( \dfrac{30x(x-1)^2yz}{(y+1)(z-1)} \right)$\\
            & $= \log(30) + \log(x) + \log((x-1)^2) + \log(y) + \log(z)$\\
            & \hspace*{2cm} $ - \log(y+1) - \log(z-1)$\\
            & $= 1 + \log(3) + \log(x) + 2\log(x-1) + \log(y) + \log(z)$\\
            & \hspace*{2cm} $ - \log(y+1) - \log(z-1)$
        \end{tabular}
        
        Next we do the same with the second log;
        
        \begin{tabular}{ll}
            $\log\left( \dfrac{(y^2-1)(z^2-1)}{(x^3 - 3x^2 + 3x - 1)z} \right)$ 
            & $= \log\left( \dfrac{(y-1)(y+1)(z+1)(z-1)}{(x-1)^3z} \right)$\\
            & $= \log(y-1) + \log(y+1) + \log(z+1) + \log(z-1)$\\
            & \hspace*{2cm} $  - \log((x-1)^3) - \log(z)$\\
            & $= \log(y-1) + \log(y+1) + \log(z+1) + \log(z-1)$\\
            & \hspace*{2cm} $  - 3\log(x-1) - \log(z)$
        \end{tabular}
        
        Finally we combine the two expanded expressions we got above;
        
        \begin{tabular}{ll}
            & $ \log\left( \dfrac{(30x^3 - 60x^2 + 30x)yz}{(y+1)(z-1)} \right) + \log\left( \dfrac{(y^2-1)(z^2-1)}{(x^3 - 3x^2 + 3x - 1)z} \right) $ \\
            & $= 1 + \log(3) + \log(x) + 2\log(x-1) + \log(y) + \log(z) - \log(y+1) - \log(z-1) + $\\
            & $  \log(y-1) + \log(y+1) + \log(z+1) + \log(z-1) - 3\log(x-1) - \log(z)$ \\
            & $= 1 + \log(3) + \log(x) - \log(x-1) + \log(y) + \log(y-1) + \log(z+1)$
        \end{tabular}
    }% End Example


    \example{Condense the following expression to a single log, $\ln(x^2) - 3\ln(y) + \frac{1}{2}\ln(x)$}{%
        In order to condense separated log terms we need to get them into the \textit{exactly} correct forms for the log rules, that is, either $\log($stuff$) + \log($stuff$)$ or $\log($stuff$) - \log($stuff$)$. In particular, we need to make sure there are \textit{\textbf{no} coefficients in front of the log term}. To accomplish this, we will use the fact that, for any coefficient $c$ we can transform $c\log_b(x)$ into $\log_b(x^b)$, ie we can move the coefficient to the argument as a power. Thus, for our expression we would start with,
        \[
            \ln(x^2) - 3\ln(y) + \frac{1}{2}\ln(x) = \ln(x^2) + \ln\left(y^{-3}\right) + \ln\left(x^{\frac{1}{2}}\right)
        \]
        Note that we could have left the negative sign and written $- \ln(y^3)$ instead of $\ln(y^{-3})$. There is nothing wrong with doing this, but I find that often the negative sign generates computational errors, so putting it into the power is a way to avoid that trouble.\\
        
        Next we will condense the logs using the property that $\log(xy) = \log(x) + \log(y)$ in `reverse', meaning we will go from the right hand side to the left hand side of the equality, which gives us;
        \[
            \ln(x^2) + \ln\left(y^{-3}\right) + \ln\left(x^{\frac{1}{2}}\right) = 
            \ln\left( x^2y^{-3}x^{\frac{1}{2}} \right) = 
            \ln\left(\dfrac{x^{\frac{5}{2}}}{y^3}\right)
        \]
        Now we are done. $\ln\left(\dfrac{x^{\frac{5}{2}}}{y^3}\right)$ is the condensed form of the log. 
    }% End of Example


% End of Topic 12.
    
    
    
    
