\Lecture{Applying Mathematical Reasoning to Modeling}
\injectBox{Terminology To Know}{
\newTerm{Information}{A collection of knowledge or observations used to problem solve.}
\newTerm{Data}{Quantified versions of information.}
\newTerm{Model}{A (collection of) mathematical expressions used to represent a specific situation.}
}
\injectBox{Questions To Answer}{
How do you tell what is important and what is extraneous information?\\
Why are Models useful?\\
How does this apply outside of a classroom?\\
}

\ifcompletedNotes
\injectBox{Goal for Lecture Content}{
Math is a "language of deduction". It is a way to bring precision to your reasoning and problem solving.\\
Recognizing extraneous vs important information.\\
How is mathematics like a language?\\
Translating between words ("information") and Models ("data").\\
}
\textbf{Estimated Lectures: 1}
\fi

\subsection{So what is a Model?}
Although we aren't quite ready to get into true ``models", what we discussed in the last topic might most conveniently be described as a ``numeric model."%
\footnote{Technically ``numeric model" refers to a product of a numeric approximation method that we won't be discussing in this class. In our case however we mean simply that the ``model" was entirely number-driven without using variables, abstracting or generalizing}
Another way of thinking about what we did in the last topic, is that we reasoned our way to a solution to the problem we were given. But how do you go from all those observations and statements, to something you can attach numbers to? How do you go from ``how much would it cost to build a patio" to ``I will need approximately 300 pavers at \$$2.50$ each"?

This is where it is important to recognize \textit{types} of information. Information comes in many different types. In this class we will cover some specific types of information that are best handled by the mathematical reasoning process%
\footnote{Note that this doesn't include all types. For example there are many unquantifiable forms of information that we will rarely discuss here, including most of what is studied in humanities disciplines like History or Philosophy, and even in some other scientific fields like psychology.}%
, but before we narrow our focus to that it is worthwhile to at least recognize a few other types of information.

At it's most basic level we could define ``information" as ``a collection of knowledge or observations used to problem solve". Even this definition is a little lackluster (as are most definitions when it comes to trying to define anything as nebulous as ``information"). Nonetheless, you want to think of information as the basic blocks you use to build some method of solving the problem in front of you. Just like anything you build however, it's important to know what the strengths and weaknesses of your building materials are, and information is no exception. Here are some key types of information for our purposes:
\begin{itemize}
\item{\textbf{Quantifiable Information:}} Information that can be numerically or algebraically represented (but may not be yet, see Data below); eg the design of the patio or the fact that bricks are available to buy at the local hardware store.
\item{\textbf{Data:}} Data is Quantifiable Information that has been quantified. This is a gray line; data is rarely distinguished from quantifiable information. An example might be that the ``quantifiable information" would be that a brick can be bought with money (ie there is some number that corresponds to it's cost) whereas the ``data" would be the specific cost of the brick (eg the brick costs \$$2.50$). In one case we have information that tells us there exists a piece of data (a number), but we don't have it yet, whereas in the other case we have the actual data (number) already.
\item{\textbf{Extraneous Information:}} Information that is not relevant to the problem solving process/method you are using. This is often (somewhat paradoxically) the most important type of information, in that recognizing that some piece of knowledge can be ignored is often the key to seeing how a solution can be achieved. In our example; if you keep spending all your time trying to quantify the relative costs of paint colors and you don't stop to ask if the patio is going to be painted, then you are wasting a lot of time if it turns out that paint isn't necessary.
\end{itemize}

\subsubsection*{Really though, what does any of this have to do with math?}
If you are wondering why I haven't given you anything to memorize yet, I would first refer you to earlier; memorizing is the antithesis of learning. But, if you are wondering what this has to do with ``math", especially since I've been somewhat vague as to what that is, that is a fair question. The answer is that this has \textit{everything} to do with math.

Remember, mathematics is not about formulas, numbers and variables. It is true that mathematics utilizes these things (and we will get to that part later). Perhaps a better question then is; "what does any of this have to do with what we've done so far (my claim, after all, is that we are doing math right now!). Mathematics is \textit{the language of deduction}, it is a very carefully developed language built with the intent of bringing precision and structure to what we casually refer to as ``thinking".%
\footnote{Don't worry I'm not a megalomaniacal proselytizer and I am not trying to say all thought is mathematics. Here I mean more that; when we say things like ``thinking through a problem" what we typically mean is trying to deductively work through a problem from some beginning set of information. \textit{This} specific type of ``thinking" is really what I'm referring to when I'm talking about math}

\subsubsection*{Why does ``thinking" need to be precise? What does that even mean?}
You may already believe that your thinking is precise. Or you may believe that thinking can't be made precise. The truth is, thinking itself is pretty nebulous, since it depends on the individual. What we really mean is that \textit{mathematics aims to bring precision to the process of communicating one's thoughts}. By way of example, think of some amount of money. Really, think of it. Now think about some amount of milk... yes really. Chances are, when you thought of the money, you probably imagined a number, say one million dollars. You could have pictured a pile of bills, but instead you had a number to represent the value. In contrast, when you thought of milk, you probably had a ``glass" of milk in mind, or maybe a container. But a ``glass" is nonspecific. Is the glass half full? filled to the point of surface tension? Is it a twelve ounce glass? sixteen? eight? Keep in mind that if I had said to think of a ``glass" of milk instead of ``some" milk, most would say I'm being more precise, but am I really?

One of the key aspects of mathematics is to bring specifics into play when you are trying to discuss these things. Instead of saying ``I have some milk" it is more helpful to say ``I have thirteen ounces of milk." Otherwise the person you are talking to may be picturing a vastly different glass than you are. For example; you could be imagining a travel mug of milk, whereas the other person may be considering one of those tiny Styrofoam cups or wax paper cones of milk... or if they are Canadian, a bag of milk.\footnote{Yes, Canadians buy milk by the bag. But in fairness so do a ton of other countries}

One may notice in our numeric models above that we had to go through a process of 'quantifying our information'. Another way to think of this is that, any problem we are given to solve will almost inevitably entail nonprecise information or goals. Thus, one of your jobs as a problem solver (ie modeler) is to bring mathematical precision to the ``thinking" of the one requesting a model. This is typically the hard part of the ``clarifying" steps and then the future ``quantifying" steps.


\injectBox{Important points about mathematics as a language}{
Viewing mathematics as a language is so appropriate that it is worth further explanation.%

Firstly this analogy to language is deliberately made as we will make it a point in this class to discuss how to ``properly convey" your ideas in mathematics. This is usually the part that drives students insane as it seems like professors are being arbitrarily nitpicky about what they require. The truth is that, like any language, whether or not mathematics is written correctly is a very subtle thing, but obvious to those that are fluent. For instance consider the two sentences: ``This things is written correctly" vs ``This thing is written correctly". The first one likely feels entirely wrong when you try to read it out loud (assuming you noticed it was wrong; your brain might have autocorrected it for you!) and immediately makes you cringe a little inside, whereas the second seems fine (if somewhat boring). Yet, to someone that is learning English, the difference is ``merely one letter", a difference they would probably think is nitpicky and if you corrected it and took points off in an essay they might say you are being unreasonable. This is typically the situation math teachers are in when they are deducting small amounts of credit in your work. When you say you are ``only off by a negative sign" that is the English equivalent of accidentally writing ``the cake is not a lie" instead of ``the cake is a lie"; you've completely reversed the meaning of the sentence to the polar opposite.

Another way this analogy is apt is that one minor flaw in a mathematical solution can upset the entire process and render the result meaningless. If we scramble the words of an essay, we would expect to get a poor grade. If someone has to debate a point in a discussion class and instead fumbles to put together a coherent string of words, we would probably infer that they had (at the very least) not properly prepared. Yet these are exactly the sort of analogs to having a computational mistake at the start of a problem that filters through the whole rest of it (especially if there are a horde of mistakes that somehow happen to yield a ``correct" answer). Math teachers will often say that it is ``more important how you got the answer than what the answer is". Yet this may not exactly convey the point.

If you are writing an essay on World War Two and handed in a bunch of gibberish, but had the concluding line of ``Hitler was wrong", would you expect to get a good grade because you had a ``good answer"? This is what your math teacher meant; the process in your worked-out solution is important because \textit{it's the process that has all the specifics and support in it}. The value you get at the end should be looked at as a useful footnote to the process you created to get that value; just like the essay concludes that Hitler was wrong, but that wasn't what gave the essay substance and meaning, the substance comes from what specifics you are utilizing to highlight and support your conclusion.

To bring it back to our example, the process we used (phase one through three in this topic) is the (abridged version of the) ``logical argument" we used to get to a conclusion. This it is exactly that logical structure/argument that is the ``mathematics" of the problem. This class will teach you a lot about what we can do with this language, like a poetry class might teach you how to parse a poem, build cadence, or structure various kinds of poetic forms, but ultimately the \textit{beauty} of the work reaches well beyond the mechanics. Thus, as we move forward to the ``\textit{what} can we do" in this class, don't forget the ``\textit{why} do we do this" and the ``\textit{how} do we do this" of what we are learning. Otherwise you are wasting your time memorizing all the useless facts of the course (eg memorizing a few specific poems) instead of learning the content of the course (eg learning how to write your own poem).}

\addToAppendix{Detailed Numeric Model Example Walkthrough}{% This is the very detailed full numeric walkthrough example
This is intended as a detailed walkthrough to give students an idea of what a \textit{complete and thorough} version of a numeric model is. Keep in mind this is the ideal, and in practice it is often the case that they are messier and/or lack some data. Moreover, because this is a numeric model (and a precalculus class) this model will be simplified in terms of the depth of detail we consider when we are modeling the given problem.

\subsubsection*{The Given Problem}
You are a master carpenter with several apprentices and you have been subcontracted by a large construction company on one of their projects; building apartments in a new apartment building. You need to have a proposal for what to charge the company to put kitchen cabinetry in a single apartment.

\subsubsection*{Phase One: Clarify and precisely state the problem.}
You know you need to come up with a number to charge for an apartment's kitchen cabinets, but more information is needed. How big is the apartment? How many cabinets are necessary? What kind/quality of wood is appropriate? (ie is it a high end luxury apartment, or a low end mass housing apartment for low income families?) You decide to sit down and brainstorm this initial list of possible things to consider and/or ask the construction company about to get a better idea of what you are expected to design.
\footnote{A master carpenter would more easily know which information to consider and which information is extraneous. You may consider trying to determine if there is any important information not considered here, or what information here might be extraneous and why?}
\begin{enumerate}
\item Wood quality (or specific type if available)
\item Solid wood construction, or paneled wood construction (ie particle board or equivalent with veneer facing attached)?
\item What cubic footage of storage (or other available dimensions) should be supplied?
\item How many occupants are expected in the apartment?
%\item Are there any required/requested finishes or paints for the cabinets?
\item Is the cabinetry required to be finished (eg painted and ready for use) or unfinished (eg only raw wood construction, to be painted and finished by someone else)?
\end{enumerate}

After further discussion with the construction company you determine you are expected to supply fully finished solid cherry construction cabinets with brass hardware, unpainted but sealed with appropriate chemical sealant. Moreover, they expect 2-3 occupants per apartment and request between 25 and 40 cubic feet of storage depending on your own professional recommendation for the space.

\subsubsection*{Phase Two: Quantifying the Information.}

Now that you have a more technical and precise idea of what is expected, you get to work determining the structure of the cabinets, the amount of materials required, and the corresponding costs. You know you need to consider the following things that contribute to cost;

\begin{itemize}
\item Labor: You expect to need 3 apprentice carpenters for three days, for a total of seventy two man-hours of work per apartment.
\item Travel: Travel to and from the work-site would account for \$100 per day in gas, insurance, and vehicle maintenance.
\item Raw Materials:
    \begin{itemize}
    \item Wood: Cherry wood is about \$9 a square foot
    \item Hardware: Hinges, screws, finishing nails, and handles come to about \$8 per cabinet.
    \item Finishing Supplies: An oil-based sealant, sanding supplies, and decorative carving come to about \$3 per cabinet (discounting labor).
    \end{itemize}
\end{itemize}

After looking through your blueprints and determining the build style and structure, you conclude that your design will have 37 cubic feet of storage and will need the following for raw materials;

\begin{itemize}
\item Wood: 30 square feet of wood for the top cabinets and 26 square feet for the bottom cabinets, for a total of 56 square feet of Cherry.
\item Hardware: There will be a total of 6 top cabinets and 10 bottom cabinets (for the purposes of hardware, drawers are similarly counted as cabinets), for a total of 16 `cabinets'.
\item Finishing Supplies: Again there are a total of 16 cabinets.
\end{itemize}

You will also need to account for your own salary, a profit margin, and ``slippage"
\footnote{In subcontracting and contracting work, the term slippage is often used to describe lost materials or labor due to mistakes in either planning or execution. For example, someone cutting a board too short by accident may mean that you lose a certain amount of wood that becomes entirely unusable, even for other aspects of your project.}

\subsubsection*{Phase Three: Developing your (numeric) answer}
Once you have written down all the specifics and quantified them, it is time to determine the final number to send to the contractors. First you compute the actual raw cost to you (the master carpenter) of the apartment cabinets, assuming everything goes perfectly correctly.\footnote{Spoiler: it never does.}

To determine this, we compute the following:
\begin{itemize}
\item Labor: 3 apprentice carpenters, each paid \$25/hr, for 72 man-hours comes to \$1800 total.
\item Travel: 3 days at \$100 a day comes to \$3000 total.
\item Raw Materials:
    \begin{itemize}
    \item Wood: 56 square feet of wood at \$9 per square foot comes to \$504 total.
    \item Per-Cabinet Costs: For finishing supplies and hardware it was 16 cabinets at \$11 per cabinet for \$176 total.
    \end{itemize}
\end{itemize}
Thus your total raw expenses are:
\[
(25\cdot 72) + (3\cdot 100) + (56\cdot 9) + (16\cdot 11) = \$1800 + \$300 + \$504 + \$176 = \$2780
\]
Thus your raw expenses as the subcontractor come to a total of \$2780 per apartment. A general rule of thumb for carpentry is to expect 10\%-15\% of raw expenses as `slippage`. You decide to play it a bit safe and round up, adding \$430 as a `slippage' factor. You also need to include your own salary which is billed at \$75 per hour, and you figure you had to contribute the planning and then oversight time for a total of about 40 hours of work, which is \$3000.

So you official charging price comes to:

\[
(\$2780 + \$430 + \$ 3000) = \$6210
\]
per apartment.

Finally, you always make sure to include some profit margin to guard against unexpected delays, changes in planning, and various other possible mitigating factors. So you take the final number and add an extra ten percent to be safe. Thus your final charging price (including the extra ten percent) comes to: \$6210.

\subsubsection*{Conclusion}

After accounting for all the material costs, mitigating charges (charges against possible complications in the project that you may be held accountable for), maintenance and transport fees, and labor, you finally come up with a number for your contractor of \$3850 per apartment for cabinets.

}% End of apprendix entry.




