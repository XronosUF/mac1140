\documentclass{ximeraXloud}

\title{Types of Information}
\begin{document}
\begin{abstract}
    This section aims to explore and explain different types of information.
\end{abstract}
\maketitle


Although we aren't quite ready to get into true ``models", what we discussed in the last topic might most conveniently be described as a ``numeric model."%
\footnote{Technically ``numeric model" refers to a product of a numeric approximation method that we won't be discussing in this class. In our case however we mean simply that the ``model" was entirely number-driven without using variables, abstracting or generalizing}
Another way of thinking about what we did in the last topic, is that we reasoned our way to a solution to the problem we were given. But how do you go from all those observations and statements, to something you can attach numbers to? How do you go from ``how much would it cost to build a patio" to ``I will need approximately 300 pavers at \$$2.50$ each"?

This is where it is important to recognize \textit{types} of information. Information comes in many different types. In this class we will cover some specific types of information that are best handled by the mathematical reasoning process%
\footnote{Note that this doesn't include all types. For example there are many unquantifiable forms of information that we will rarely discuss here, including most of what is studied in humanities disciplines like History or Philosophy, and even in some other scientific fields like psychology.}%
, but before we narrow our focus to that it is worthwhile to at least recognize a few other types of information.

At it's most basic level we could define ``information" as ``a collection of knowledge or observations used to problem solve". Even this definition is a little lackluster (as are most definitions when it comes to trying to define anything as nebulous as ``information"). Nonetheless, you want to think of information as the basic blocks you use to build some method of solving the problem in front of you. Just like anything you build however, it's important to know what the strengths and weaknesses of your building materials are, and information is no exception. Here are some key types of information for our purposes:
\begin{itemize}
    \item{\textbf{Quantifiable Information:}} Information that can be numerically or algebraically represented (but may not be yet, see Data below); eg the design of the patio or the fact that bricks are available to buy at the local hardware store.
    \item{\textbf{Data:}} Data is Quantifiable Information that has been quantified. This is a gray line; data is rarely distinguished from quantifiable information. An example might be that the ``quantifiable information" would be that a brick can be bought with money (ie there is some number that corresponds to it's cost) whereas the ``data" would be the specific cost of the brick (eg the brick costs \$$2.50$). In one case we have information that tells us there exists a piece of data (a number), but we don't have it yet, whereas in the other case we have the actual data (number) already.
    \item{\textbf{Extraneous Information:}} Information that is not relevant to the problem solving process/method you are using. This is often (somewhat paradoxically) the most important type of information, in that recognizing that some piece of knowledge can be ignored is often the key to seeing how a solution can be achieved. In our example; if you keep spending all your time trying to quantify the relative costs of paint colors and you don't stop to ask if the patio is going to be painted, then you are wasting a lot of time if it turns out that paint isn't necessary.
\end{itemize}




\end{document}