\documentclass{ximeraXloud}

\title{Math as a Language}
\begin{document}
\begin{abstract}
    This section contains important points about the analogy of mathematics as a language.
\end{abstract}
\maketitle

Viewing mathematics as a language is so appropriate that it is worth further explanation.%

Firstly this analogy to language is deliberately made as we will make it a point in this class to discuss how to ``properly convey" your ideas in mathematics. This is usually the part that drives students insane as it seems like professors are being arbitrarily nitpicky about what they require. The truth is that, like any language, whether or not mathematics is written correctly is a very subtle thing, but obvious to those that are fluent. For instance consider the two sentences: ``This things is written correctly" vs ``This thing is written correctly". The first one likely feels entirely wrong when you try to read it out loud (assuming you noticed it was wrong; your brain might have autocorrected it for you!) and immediately makes you cringe a little inside, whereas the second seems fine (if somewhat boring). Yet, to someone that is learning English, the difference is ``merely one letter", a difference they would probably think is nitpicky and if you corrected it and took points off in an essay they might say you are being unreasonable. This is typically the situation math teachers are in when they are deducting small amounts of credit in your work. When you say you are ``only off by a negative sign" that is the English equivalent of accidentally writing ``the cake is not a lie" instead of ``the cake is a lie"; you've completely reversed the meaning of the sentence to the polar opposite.

Another way this analogy is apt is that one minor flaw in a mathematical solution can upset the entire process and render the result meaningless. If we scramble the words of an essay, we would expect to get a poor grade. If someone has to debate a point in a discussion class and instead fumbles to put together a coherent string of words, we would probably infer that they had (at the very least) not properly prepared. Yet these are exactly the sort of analogs to having a computational mistake at the start of a problem that filters through the whole rest of it (especially if there are a horde of mistakes that somehow happen to yield a ``correct" answer). Math teachers will often say that it is ``more important how you got the answer than what the answer is". Yet this may not exactly convey the point.

If you are writing an essay on World War Two and handed in a bunch of gibberish, but had the concluding line of ``Hitler was wrong", would you expect to get a good grade because you had a ``good answer"? This is what your math teacher meant; the process in your worked-out solution is important because \textit{it's the process that has all the specifics and support in it}. The value you get at the end should be looked at as a useful footnote to the process you created to get that value; just like the essay concludes that Hitler was wrong, but that wasn't what gave the essay substance and meaning, the substance comes from what specifics you are utilizing to highlight and support your conclusion.

To bring it back to our example, the process we used (phase one through three in this topic) is the (abridged version of the) ``logical argument" we used to get to a conclusion. This it is exactly that logical structure/argument that is the ``mathematics" of the problem. This class will teach you a lot about what we can do with this language, like a poetry class might teach you how to parse a poem, build cadence, or structure various kinds of poetic forms, but ultimately the \textit{beauty} of the work reaches well beyond the mechanics. Thus, as we move forward to the ``\textit{what} can we do" in this class, don't forget the ``\textit{why} do we do this" and the ``\textit{how} do we do this" of what we are learning. Otherwise you are wasting your time memorizing all the useless facts of the course (eg memorizing a few specific poems) instead of learning the content of the course (eg learning how to write your own poem).


\end{document}