\documentclass{ximeraXloud}

\title{Logical Deduction}
\begin{document}
\begin{abstract}
    This section analyzes the previous example in detail to develop a three phase deductive process to develop a mathematical model.
\end{abstract}
\maketitle




Let's revisit the previous example and work through it to get a solution. This will highlight the reasoning process, and in the following lectures we will work through each of the reasoning steps in more detail.

\subsubsection*{Phase One: Statement and clarification of the problem}

    \begin{explanation}
        The first step in any problem, is the problem itself. This may seem obvious and trivial, but it's unfortunately the case that what someone \textit{asks} of you, and what they \textit{want} of you, are often not the same. So, even though it seems like the first phase is given for free (that is, that you have to be asked a problem in order to start problem solving) the truth is that often the first step is clarifying what someone has asked of you until you can concretely describe what they want. Let's look at our patio example.
        
        The original question was simply ``How much would it cost to build a patio?" However, this is entirely too vague. The desired answer is clearly quantifiable (they want cost, a number, as an answer after all), but the given information ``a patio" isn't quantifiable. Before we can give a quantified answer then, we need to clarify the question into something we can assign numbers to.
        
        How one goes about this can vary, but at the very least we should ask the most basic questions, such as:
        \begin{itemize}
            \item What is the patio made of?
            \item What \textit{is} a patio, or more specifically, what is a patio to the person asking the question? (Pro-tip: Never assume the person asking a question is using the same precise vocabulary you are).
            \item How big does the person want the patio? Specific dimensions (numbers!) are preferable.
        \end{itemize}
        
        After some back and forth questioning you determine that the patio is going to be made of cement paving stones, and encompass a flat area of between 15 and 20 feet on each side.
    \end{explanation}
    
\subsubsection*{Phase Two: Quantifying the situation, ie turning Information into Data}

    \begin{explanation}
        The next step is to start figuring out relevant numbers and their relationship. This is usually where you discover any other numbers that you need which are missing. Let's continue our patio example;
        
        We now know that the patio is made of cement paving stones, and that we want a surface between 15 and 20 feet on each side to be covered with them. Clearly, if we are trying to determine cost, we should know the cost of something. Hopefully it's clear that, since we are using cement paving stones to cover the surface, we need to know the cost of those paving stones, but that's not quite enough. We also need to know how many paving stones it will take to cover the given surface. After a quick trip to the local building supply store, you determine that cement paving stones are around \$$2.50$ each, and are about a foot and a half long and a half foot wide.
    \end{explanation}
    
\subsubsection*{Phase Three: Developing your (numeric) answer}

    \begin{example}
        Now we have all the information (data) that we need to determine an answer. Since the given question (and followup clarification) specified between 15 and 20 feet on a side (ie a range of possibilities), our answer should likewise be a range of possibilities. To get the minimum cost, it makes sense that we would build the smallest possible area, a fifteen by fifteen foot patio. Likewise, to get the highest cost we would build the largest patio, a twenty by twenty foot area.
        
        It would take ten pavers stacked end to end (the long way) to cover fifteen feet, and then it would take thirty of them stacked side by side to attain fifteen feet. Thus to completely cover a fifteen by fifteen foot patio with cement pavers, we would need $10 \cdot 30 = \answer{300}$ pavers at \$$2.50$ each, for a total cost of \$$\answer{750}$. Using similar calculations we find that we need approximately 520 cement pavers to cover a twenty by twenty foot patio, for a total cost of \$$1300$.
    \end{example}
    
\subsubsection*{Is that it?}

    The above is a basic example of using mathematical reasoning to answer a problem. But it can be used to do much more than that. To do so, we will introduce the idea of Modeling in the next lecture, and see how mathematical reasoning can be used to build a more general answer (after all; where did those equations come from in the example box?) 



\end{document}