\documentclass{ximeraXloud}

\title{Is this actually math?}
\begin{document}
\begin{abstract}
    This section aims to show how mathematical reasoning is different than `typical reasoning', as well as showing how what we are doing is mathematical.
\end{abstract}
\maketitle

\subsubsection*{Really though, what does any of this have to do with math?}
    \begin{exploration}
        If you are wondering why I haven't given you anything to memorize yet, I would first refer you to earlier; \wordChoice{\choice{learning}\choice[correct]{memorizing}} is the antithesis of \wordChoice{\choice[correct]{learning}\choice{memorizing}}. But, if you are wondering what this has to do with ``math", especially since I've been somewhat vague as to what that is, that is a fair question. The answer is... this has \textit{everything} to do with math.
        
        Remember, mathematics is not about formulas, numbers and variables. It is true that mathematics utilizes these things (and we will get to that part later). Perhaps a better question then is; ``what does any of this have to do with what we've done so far?" (my claim, after all, is that we are doing math right now!). Mathematics is \textit{the language of deduction}, it is a very carefully developed language built with the intent of bringing precision and structure to what we casually refer to as ``thinking".%
        \footnote{Don't worry I'm not a megalomaniacal proselytizer and I am not trying to say all thought is mathematics. Here I mean more that; when we say things like ``thinking through a problem" what we typically mean is trying to deductively work through a problem from some beginning set of information. \textit{This} specific type of ``thinking" is really what I'm referring to when I'm talking about math}
    \end{exploration}
    
\subsubsection*{Why does ``thinking" need to be precise? What does that even mean?}
    You may already believe that your thinking is precise. Or you may believe that thinking can't be made precise. The truth is, thinking itself is pretty nebulous, since it depends on the individual. What we really mean is that \textit{mathematics aims to bring precision to the process of communicating one's thoughts}. By way of example, think of some amount of money. Really, think of it. Now think about some amount of milk... yes really. Chances are, when you thought of the money, you probably imagined a number, say one million dollars. You could have pictured a pile of bills, but instead you had a number to represent the value. In contrast, when you thought of milk, you probably had a ``glass" of milk in mind, or maybe a container. But a ``glass" is nonspecific. Is the glass half full? filled to the point of surface tension? Is it a twelve ounce glass? sixteen? eight? Keep in mind that if I had said to think of a ``glass" of milk instead of ``some" milk, most would say I'm being more precise, but am I really?
    
    One of the key aspects of mathematics is to bring specifics into play when you are trying to discuss these things. 
    \begin{example}
        Instead of saying ``I have some milk", which of the following is more helpful in conveying exactly how much milk I have?
        \begin{multipleChoice} 
            \choice{I have a bunch of milk.}
            \choice[correct]{I have thirteen ounces of milk.}
            \choice{I have a container of mulk.}
            \choice{I have a full glass of milk.} 
        \end{multipleChoice}
    \end{example}
    
    Notice that the precision is useful because the person you are talking to may be picturing a vastly different glass than you are. For example; you could be imagining a travel mug of milk, whereas the other person may be considering one of those tiny Styrofoam cups or wax paper cones of milk... or if they are Canadian, a bag of milk.\footnote{Yes, Canadians buy milk by the bag. But in fairness so do a ton of other countries}
    
    \begin{exploration}
        One may notice in our numeric models above that we had to go through a process of 'quantifying our information'. Another way to think of this is that, any problem we are given to solve will almost inevitably entail nonprecise information or goals. Thus, one of your jobs as a problem solver (ie modeler) is to bring mathematical precision to the ``thinking" of the one requesting a model. This is typically the hard part of the ``clarifying" steps and then the future ``quantifying" steps, but it is necessary since otherwise your solution may \wordChoice{\choice{be dismissed as pointless}\choice{be only applicable to a small subset}\choice[correct]{be wholly inaccurate due to faulty assumptions}}.
    \end{exploration}

\end{document}