\Lecture{Extending To Models}
\injectBox{Terminology To Know}{
\newTerm{Generalize}{A process of going from a (somewhat) specific problem (eg a numeric model) to a methodology or (Generalized) Model to attain answers to (less specific) variations of the original problem.}
\newTerm{(Mathematical) Symbol}{A mathematical symbol (usually referred to simply as `symbol' when context is clear) is a non-numeric piece of writing (typically a letter) that holds the place of some (typically numeric) value, to be decided (or deduced) later.

    \emph{The conceptual definition (Not to be used as the actual definition)}: A symbol is a letter, in some alphabet, that is a temporary stand-in for a piece of information. It is typically either a variable or replaced by a concrete value when one forms a model.
    
    \textbf{Author's Note:} This is included as we use the term 'symbol' many times throughout this section and thus it should have a formal definition. \textbf{However this term will not be tested on in discussion quizzes or on exams.} It is necessarily abstract given it's usage, but the abstraction is so deep as to be beyond the scope of this course.}
\newTerm{Variable}{A symbol used to represent an unknown piece of data.}
\newTerm{Independant Variable}{A variable whose value is unspecified, but will be provided by the statement of the problem.}
\newTerm{Dependent Variable}{A variable whose value is unknown and is generated by the Model. In most cases dependent variables are determined (solved for) by the Model (See Arbitrary Constant Below).}
\newTerm{Arbitrary Constant}{A symbol that is unaffected by independent variables, and whose value is determined by the model and initial conditions of the system.}% Old/Bad def: dependent variable whose value is left unknown (but we know it is a specific fixed value). Often these are left as values ``to be determined" by whomever posed the original problem, but are irrelevant to the solution method.}
}
\injectBox{Questions To Answer}{
When should you extend to a model, and when should you not?\\
What is the virtue of generalizing?\\
Can you generalize a generalization?\\
}

\ifcompletedNotes
\injectBox{Goal for lecture Content}{
The role of Variables, Independent Variables, Dependent Variables, and Arbitrary Constants.\\
Using the end goal to generate a process to attain the end goal (problem solve by starting at the end and working backwards).\\
Breaking big problems into a lot of small problems. Divide and Conquer.\\
"Embrace Laziness": Do something once thoroughly, to have minimal effort put in to do it in the future.\\
Sometimes it is easier to solve a more general question than a super-specific one.
}
\fi

\subsection{Embracing Laziness; Using Generalized Models}

So far we have spent a large amount of time and effort learning how to solve very narrow (and occasionally non-specific) problems. It gets tiresome to go through this process for every single question you are posed however, and often you can take many questions and group them together under one single problem with minor variations. This is where generalized models will come in handy.

Let's consider our earlier patio example. We were asked to determine the cost of building a patio, presumably by an \textit{individual} wanting a patio. But what if you worked for a construction company? You'd have to go through a similar process many times a week, and doing so would get irritatingly repetitive. But, things that are repetitive are usually susceptible to being generalized, and thus you can streamline the process. So, rather than brute forcing the work for many patios, embrace laziness and generalize the process to come up with a ``shorthand" version that works for almost any desired patio!

\textbf{Author's note on the best way to read the following sections}: I will first describe the process in general and then present a short example. It would be ideal to read these things in parallel, but for the sake of continuity I will present one and then the other. I actually recommend reading this content in order first and not worrying too much about fully understanding the ``concept" portion on the first time through. Then read the example, and come back to reread each section of the ``concept" description with the corresponding section in the example. This may seem much more time consuming, but it has a much higher chance of helping you ``learn" the content (as oppose to ``memorize" the content).

\subsubsection*{First step to generalizing: Determine what can be generalized.}
The first step in generalizing a numeric solution may seem ``obvious" (in the same way that the first phase of solving the numeric model; ``clarifying the problem" seemed like it should be obvious), but it is again often deceptively important. A good way to start determining what can be generalized, is to consider what information you need and already know what (type of) units it would be supplied in.
\footnote{One may restate this to say that a good first step would be to look at what you have as \textit{data}, and not simply \textit{information}. Data is often easier to generalize as it is quantifiable, and that quantity is what you are generalizing. It is very important to remember that this is only a start however. Often, in industry, the most pivotal piece of information to generalize isn't found by doing this, but it at least gives you a place to start.}
In our patio example we needed to know what the patio was made of (which we later determined would be cement pavers) as well as what size the patio would be. This first piece of information (what the patio was made of) isn't in units that we can expect (ie it isn't going to be \textit{data}) and as such may not be a good candidate for generalizing. The size however is going to be some form of area (and thus will be \textit{data}), thus we can expect some kind of square units (eg square feet or square yards) and so that piece of information may be a good candidate to try and generalize.

\injectComment{A common difficulty}{A common error is to overgeneralize. Just because you may be able to generalize a piece of information, doesn't mean you will want to. The first step is to \emph{identify} which elements we \emph{are able} to generalize, but that doesn't mean we will generalize every piece of information we can. This is the art-form of modeling; there is no definite rule to follow to know what is the `right' amount of generalizing, but with practice one can develop a talent for determining the `perfect' level of generalizing.}

\subsubsection*{Next Step: Determine what \emph{should} be generalized... and how.}

The art of modeling (in the mathematical sense) comes into play when you are trying to decide which variables to generalize. On the one hand, the more you generalize, the more versatile and applicable your model becomes. On the other hand, generalizing takes time and effort, meaning you may spend so much time generalizing that you take far longer to do the same job... something your boss will undoubtedly not appreciate.

The broad rule of thumb is to ask yourself ``which of the things that I can generalize are likely to need to change from project to project." If a piece of information is likely to change between different variations of projects, then it's a good candidate for generalizing. For example, not everyone will want the same size or dimension patio... so that is likely to change from project to project, ie from patio to patio. Keep in mind that these things differ by situation; consider the possibility that you are in a business where all you make are twenty by twenty patios from a variety of materials, generalizing the size of the patio will be pointless (it's always the same size after all), but generalizing the materials becomes key; something that might prove very difficult given the previous comments about the difficulty of generalizing non-data information.

Once you have identified the information that you wish to generalize, the \textit{how} is ``straightforward"
\footnote{I put straightforward in quotes because, in practice, executing the generalizing step itself might be easy, but keeping track of it as you update/build your model can be very difficult. This is why the advice about keeping a written list of variables and what they mean is absolutely key, especially early in learning this process.}.
You generalize a piece of information by replacing it in your model with the correct type of \textit{variable}.
In order to understand what we mean when we say the ``correct type of variable", you should first understand the role of variables in the model, and what different types exist.

\subsection{Variables and their Roles.}
To best understand variables and their roles, let's revisit the patio situation and consider each of the pieces of data as ``potential generalized data". In our patio example we might consider the following pieces of data;
\begin{itemize}
\item The length and width of the (rectangular) patio.
\item The area of the patio.
\item The cost of the cement pavers.
\item The size (dimensions; length and width) of a single cement paver.
\item The number of necessary cement pavers to make the patio.
\item The cost of labor to build the patio.
\item The time to build the patio.
\end{itemize}

When trying to determine which piece of data should be represented by which type of variable, it is helpful to consider how each piece of data is related to the others. For example, the length and width of the patio could give you the area, so those things are related (but notice that, having the area doesn't give you enough information to determine both length and width). The area of the patio and size of an individual paver is similarly related to the number of cement pavers one needs to build the patio with the pavers. Moreover, with a minimal amount of effort (and some basic geometry and algebra) we could write down these relationships mathematically.
\footnote{These are mathematical relationships and functions, which we will cover in more details in the next topic. For now we simply state what the relationships are between the specific pieces of data we have.}
\subsubsection*{Relationships between variables... Man that's way too many words!}
Thus, we could write down the following relationships:
\begin{itemize}
\item The (Number of Pavers Needed) is approximately%
\footnote{We say approximately here as there is some difficulty in the situation where the pavers don't exactly line up with the dimensions of the patio; eg if the patio is fifteen and two thirds of a paver long, it would likely require more paver purchases than this estimate suggests.}
the (Area of the Patio) divided by the (Area of Pavers).
\item The (Area of Pavers) is elqual to the product of the (Width of Pavers) and the (Length of Pavers).
\item The (Area of Patio) is the product of the (Width of Patio) and the (Length of Patio).
\item The (Total Cost of Pavers) is the product of (Cost of a Paver) and (Number of Pavers Needed)
\item The (Total Cost of Labor) is the product of the (Hourly Cost of Labor) and the (Time to Build Patio (in hours)).
\item The (Total Cost of Patio) is the sum of the (Total Cost of Pavers) and the (Total Cost of Labor).
\end{itemize}

Look at the list above and try to read them out loud... really. How far did you get before you stopped to come back and read because it was obnoxious to read all those words for ``such obvious relationships"? Looking at that list of relationships, it should be clear that, although we have to write phrases like ``Area of Pavers" and ``Width of Patio", it would be a lot easier if we could encode this information in something faster and easier to read; after all, we know what we mean right?%
\footnote{This phrase is often used and almost \textit{always} regretted at some point.}
This is where variables come into play. We could build an encoding, a kind of ``quick reference sheet" for a shorthand to refer to these things. An example of such a thing might be the following;
\begin{itemize}
\item $x$ is the number of bricks needed to build our patio.
\item $A_p$ is the area of the patio (in square feet).
\item $A_b$ is the area of a paver brick.
\item $L_p$ and $W_p$ are the length and width of the patio respectively.
\item $L_b$ and $W_b$ are the length and the width of the paver bricks respectively.
\item $C_l$ is the cost of the labor (in dollars).
\item $C_b$ is the cost of the paver bricks (in dollars).
\item $C_T$ is the total cost of the patio project (in dollars).
\item $t_p$ is the total time to build the patio.
\end{itemize}
The above looks intimidating. That's an awful lot of letters, but there are a few things to keep in mind when looking at that list. 
\begin{enumerate}
\item The variables names I chose were not pulled out of a hat. I deliberately picked names that correspond in some nice/`obvious' way to what it represents. For example, notice that all the `Cost' based variables are $C_{\texttt{something}}$, and moreover, that \texttt{something} clues you into what the variable is the cost of; $p$ for patio, $l$ for labor, $b$ for (paver) brick. Thus by intelligently naming your various, you can make things more sensible.
\item We have generalized \textit{everything we possibly could} and as I mentioned earlier, this is almost always overkill. For now it's helpful to see what the \textit{possibilities} are for generalizing, then we will want to cut back to which \textit{specific} data would be \textit{helpful} to generalize.
\item Despite the first point above, the variable names may make sense \textit{in context}, but it will be easy to forget that context if we were to put this down and come back to it in six months. For this reason, it's \textit{always a good idea to explicitly write down all your variables and what they literally mean}. This means writing something like "$A_p$ = area of the patio in square feet" \textbf{not} $A_p$ = area of patio. Units are the easiest thing to forget, and typically the most likely area to cause errors... just ask NASA!%
\footnote{NASA crashed a probe into a planet at great rates of speed because one data team used metric units and the other used imperial units... and nobody bothered to check before they put them together; they just used the numbers without units. That was an expensive mistake.}.
\end{enumerate}

\subsubsection*{Ok that's great but... what does this have to do with variable types?}
The next part is formalizing the relationship between these variables. This is something we will cover much more in the next topic, but for now we could probably conclude the following relationship from the above variables;
\[
A_p = L_p \cdot W_p
\]
Which tells us that the (Area of the Patio) is equal to the (Length of the Patio) times the (Width of the Patio); the basic formula for the area of a rectangle. Observe that with any two of these pieces of data, we could get the third (eg with width and total area, we could calculate the length). So the question is, what does our model expect to have provided to it, and what do we want our model to tell us? Take the following two examples:

\example{You're given length and width.}{%
    Let's say you work for a construction company and you are asked to make a model to determine the price of building a patio for a customer. That customer has very specific dimensions that they want, and so you know you will be given the length and width of the patio, but your calculations require area. In this case you would use the original equation above;
    \[
    A_p = L_p \cdot W_p
    \]
    Here the relationship between these variables in your model expects to have you supply length and width of the patio (the given information from your customer) and it will in turn calculate the area of the patio.
}


\example{You're given Area and width}{%
    Now let's say a customer is planning an above-ground pool. They know they need a certain amount of minimum square footage to have a foundation for their pool, a grill, and a lounge area. Furthermore they know they want the patio to run the full width of their main porch area. So when building the model, you want to build it assuming you will know the area and the width, but not the length. Thus you would use the following relationship;
    \[
    L_p = \frac{A_p}{W_p}
    \]
}

In both these cases we have the same variables and the model has the same end goal (to calculate the cost to build a patio), but in one situation we expect to know the length and width, but need the area. In the other situation you expect to know the area and width, but need the length.

Although we call $A_p$, $L_p$, and $W_p$ `variables' in both cases, we have special names to denote this ``expectation" aspect that is, in some sense, equally important to include in a model. \textit{For variables that we expect to be provided to us, we call them \textbf{independent variables}}. They are called ``independent" because they are (suppose to be) supplied independent of the model, meaning that they are the data that is ``fed into" the model to get results. \textit{The variables that are calculated or deduced by the model are called \textbf{dependent variables}}. These variables are call dependent because their value depends on what is put into the model (ie the dependent variables may%
\footnote{Dependent variables are \textit{capable} of changing value based on independent variable values, but that is not the same as saying they \textit{must} change value. This is a subtle distinction, but it turns out it's \textit{incredibly} important as we'll see much later when we are discussing functions and especially inverse functions.}
change value for different independent variable values). In general, if you (for a moment) think of a model as a magic machine, then independent variables are `fed into' that machine, and dependent variables are `spit back out' as your ``answers".%
\footnote{``Answers" here is in quotes as dependent variables often occur in substeps of models. In our examples above, our real ``answer" would be the cost of the patio, whereas the dependent variables (the area in the first example and length in the second) are ``answers" to their respective equations but one would typically not call them an ``answer" to the model, which tends to be what we mean when we say ``The Answer". This is why using words like ``solution" or ``answer" can be dangerous, and \textit{one should always be clear as to what they are claiming their result is an ``answer" to specifically}.}
So, in Example One above, $L_p$ and $W_p$ are both independent variables and $A_p$ is a dependent variables, but in Example Two, $A_p$ and $W_p$ are both independent variables and $L_p$ is a dependent variable.%
\footnote{Often in math classes dependent and independent variables are described by rote, meaning they simply say ``$x$ is an independent variable" and ``$y$ is a dependent variable". This is \textit{often} true, but it's very important to notice that there was no comment made in our definition about a certain letter needing to be a certain type of variable. In fact we demonstrated that the same letter could be one or the other depending on the model we are building!}

There are other kinds of variables one could encounter as well; of specific importance in calculus is the \textit{arbitrary constant}. This is typically a result of some initial information used in your model, and is a byproduct of choices in your model, but \textit{they are unaffected by independent variables}. One can think of the arbitrary constant as being a sort of ``starting spot" for your model. That is to say, even though your ``starting spot" is typically of great importance to your outcome (your starting height when throwing a ball and measuring how far it goes for example) no matter what information you ``feed into" the model (eg throwing speed, throwing angle, etc), your starting spot doesn't change. Thus the arbitrary constant doesn't change based on any of these ``input" values. 

One might wonder what the difference is, then, between a constant and an arbitrary constant. It may be clear that it doesn't vary based on independent variables, so it seems like it is constant. The key point though is that you may not know the intended ``starting spot" (ie the height someone will throw from) of your model when you are designing it. 

Consider our patio example again. You are building a generic model to calculate the cost of building a patio for your company, and part of that is travel costs. The travel costs themselves will depend on the location (where the customer lives, how accessible the construction site is, etc). But once you have determined the cost for travel, it won't change depending on the size of the patio (the independent variable). Thus it will be a constant value, but one that depends on the customer's location, not the customer's specific project. This is cost would then be an arbitrary constant; something that varies from project to project (specific model to specific model) but is constant within the specific model.


\addToAppendix{Detailed Generalized Model Example Walkthrough}{% This is the very detailed full generalized model walkthrough example
This is intended as a detailed walkthrough to give students an idea of what a \textit{complete and thorough} process of generalizing a model looks like. Keep in mind this is the ideal, and in practice it is often the case that they are messier and/or lack some data. Moreover, because this is still early in the precalculus class, this model will be simplified in terms of the depth of detail we consider when we are building the generalized model.

\subsubsection*{The Given Problem}
You have recently been hired by a local company to streamline their billing and customer interaction process. The company specializes in building brick patios, and they supply you with the following information;

\begin{itemize}
\item They make primarily rectangular patios, so they wish to have some kind of model to provide the cost of any requested rectangular patio.
\item The bricks they currently use are 6 inches by 3 inches and cost \$$1.25$ per brick.
\item The company charges \$$100$ per hour for labor, and they can usually put down about fifty square feet worth of brick in that time.
\item The company doesn't charge for travel or terrain preparation (they only do local work).
\end{itemize}

\subsubsection*{Step One: Identifying data worth generalizing}
Armed with the information they provided, you decide to start by determining what individual pieces of data there are, and then determining which might be worthy of generalizing into variables. You come up with the following list of (possibly relevant) data;
\begin{itemize}
\item The dimensions of the bricks that are used.
\item The dimensions of the patio that they are going to build
\item The cost of the bricks that are used.
\item The amount and cost of labor that is necessary to build the patio.
\item the incidental costs such as; travel, stone dust, terrain preparation, lunches, etc.
\end{itemize}

Upon further reflection on your list, you determine that the incidental costs aren't relevant because the company's owner specifically told you that they don't charge for these things and that it is considered somewhere else in the price system that they gave you. Moreover, the dimensions and cost of the bricks are constant, and the cost of labor \textit{per hour} is constant, although the amount of hours necessary for a given job will vary according to the size of the patio. So, you need to at least generalize the dimensions of the patio that is being built (this is supplied by the customer after all). Moreover you will need to have some kind of unknown value for the amount of labor needed and thus the cost of labor. 

\subsubsection*{Step Two: Assigning and recording variables explicitly (and carefully)}
Using the above data, you create the following reference:

\begin{itemize}
\item The dimensions of bricks are 3 inches by 6 inches; which means you need 8 bricks to fill 1 square foot.
\item The dimensions of the patio are unknown and are specifically assigned as independent variables. We will choose:
    \begin{itemize}
    \item $L =$ length of patio
    \item $W =$ width of patio
    \item You determine that knowing the explicit area of the patio may be helpful, so you denote this with $A$.
    \end{itemize}
\item The cost of the bricks are known to be \$$1.25$ per brick. Since 1 square foot is 8 bricks, brick costs work out to \$$10$ per square foot.
    \begin{itemize}
    \item You decide to denote the total cost for bricks by $T_B$.
    \end{itemize}
\item The amount of labor is unknown, and is a dependent variable because it depends on how big the patio is.
    \begin{itemize}
    \item $H =$ hours of work necessary to build the patio.
    \end{itemize}
\item The cost per hour of labor is a set rate of \$$100$ an hour, so it is constant and doesn't need a variable.
    \begin{itemize}
    \item The total cost for labor will depend on how long they need to work, and will be denote $T_W$.
    \end{itemize}
\end{itemize}

\subsubsection*{Building relations between variables}
Now that you have an explicit reference sheet with variable names and the constants you may need, you assemble the following relationships between the variables;

\begin{itemize}
\item Area of the patio is Length times Width.
    \begin{itemize}
    \item We record this with the equation: $A = LW$.
    \end{itemize}
\item The cost of the bricks will be the cost per square foot of bricks, times the area (in square feet) of the patio.
    \begin{itemize}
    \item We record this with the equation: $T_B = 10 \cdot A$
    \end{itemize}
\item The amount of labor will be the total area of the patio divided by the 50 square feet per hour they can lay.
    \begin{itemize}
    \item We record this with the equation: $H = \dfrac{A}{50}$
    \end{itemize}
\item The cost of labor will be the hourly wage times the number of hours.
    \begin{itemize}
    \item We record this with the equation: $T_W = 100H$
    \end{itemize}
\item Finally, our end goal, the total cost for the patio is the total cost of bricks and total cost of labor; which we can write out and simplify using the above equations as follows:

\[
\begin{array}{ll | r}
T_P &= T_B + T_W & \text{ \textbf{Step explanation}}\\
&= (10A) + (100H) & \text{ Since }T_B = 10A and T_W = 100H\\
&= 10(LW) + 100\left(\dfrac{A}{50}\right) & \text{ Since }A = LW and H = \dfrac{A}{50}\\
&= 10(LW) + 100\left(\dfrac{LW}{50}\right) & \text{ Since }A = LW\\
&= 10LW + 2LW & \text{ Simplifying the fraction}\\
&= 12LW
\end{array}
\]

\end{itemize}


\subsubsection*{Conclusion}

So we have built a generalized model to calculate the cost of a patio that is $L$ feet long by $W$ feet wide, and come up with the final answer: $T_P = 9.5LW$. Notice that many of the variables we originally defined for convenience, all ended up being simplified out during the process of simplifying $T_P$ in the last part above, this is very common when you use generalized models. It is still helpful to have those variables `lurking' in the background however; in case, for instance, the company's owner wants to know how much of his bill is actually materials versus labor... all the information is already there in your model, it just isn't displayed by the `final calculation' that you supply at the end.


    }% End of Appendix Example


