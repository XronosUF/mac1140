\Lecture{Syllabus and Course Introduction}

\ifcompletedNotes
\injectBox{Goal for lecture Content}{
When to apply mathematical reasoning, and when not to.\\
Memorizing vs learning, and what you should learn from this class.\\
What is the Difference between ``Normal Reasoning" and ``Mathematical Reasoning"?\\
What is the role of deductive reasoning in problem solving?}
\textbf{Estimated Lectures to complete topic: 1-2}
\fi


\subsection{What are the goals of this course?}
Simply put, the goal of this course is to prepare one for Calculus (MAC2311) and further math courses, with the exception of trigonometry content. \\

More conceptually however, the real goal of this course is to teach how to problem solve and use mathematical reasoning in an outside setting. Often the process of modeling and reasoning are devalued or even sacrificed entirely in an effort to teach the mechanical skills of arithmetic. This is typically met with exasperation and confusion by students who ask "What's the point of learning this?", which is a valid question in any classroom. Unfortunately the point of learning mathematics is not the computation aspect, but the reasoning and modelings aspects... those very things that are often pushed out in an effort to improve the more easily assessed computational skills. So, a more simply put, if far less simple, answer, is "The goal of this course is to teach you to 'think'." 

\subsection{How is this class different than the math classes you've had previously?}
In most math courses in the past you have likely been taught mechanics of mathematics; the computational tools like adding, subtracting, multiplying and dividing. These are definitely important things, but they are not what math \textit{is}. To use an analogy, "learning math" by learning these mechanical skills, would be like trying to learn a foreign language by reading a translation dictionary. The vocabulary is important, but reading a dictionary is dry and dull and still doesn't teach you any of the important aspects of the foreign language (eg grammar, culture, history, etc).

Thus, this class is going to embrace the other aspects of mathematics you may not have had to learn in the past (or at least, not learn as deeply). This means that this class will often seem fundamentally at odds with what you think of when you think of a "math class". My hope is that you will find this class more engaging and helpful; but regardless of the outcome, the techniques that will be discussed will be very different than what you have experienced in the past. I would recommend that you try to come to this class with an open mind and not treat this as "just another math class". With any luck, that will be a good thing.

\subsection{What are the virtues of learning math this way?}
It usually surprises people to learn that most mathematicians have spectacularly terrible memories. In a setting where most of your "math" education is centered around endless memorizing of symbols and symbol manipulation, it may seem like you're suppose to memorize everything. In fact, this is the \textit{exactly opposite} of how math should be learned.

In reality, memorizing something is very different than learning it. Often you start with memorizing, but think about this; if you were to see a word, let's say the word ``toast", you know what that means. But, do you know what it means because you memorized it? Or do you know what it means because you've simply seen and used it so much that you ``just know it"? Chances are, when you read that word you just sort of ``knew" what it meant, and it might have even evoked an idea of tasty (or awful... if you're some kind of barbarian) breakfast food. In contrast, if you read the word ``$\dot{\alpha}\pi o \pi\upsilon\rho\dot{\iota}\alpha\varsigma$", chances are that not only would you not know that it's ancient greek for ``toast", but you \textit{would know} that it's not English. Is that because you've memorized every English word? Or have you memorized every Non-English word? That's pretty unlikely, rather you've probably \textit{learned} how to recognize if something is English or not (such as the funky alphabet).

This is the fundamental difference between memorization and learning. If you learn something then it becomes ``obvious" what contexts make sense for that thing, and what contexts don't. \textbf{Mathematics is the language of deductive reasoning}, and just like English, it should be obvious when what you've written makes sense and when it doesn't. Thus if you ever find yourself in the situation of writing something down because ``that's what I'm suppose to do" and can't give any better justification, chances are that you've memorized something, but haven't learned it yet. This also means you should talk to your TA or myself in order to transition from ``memorizing" to ``learning" the concepts of this class. It will save you vast amounts of brain space and effort (after all, consider how many words you know, and how relatively little effort you have to spend to maintain that knowledge).


%ἀποπυρίας

\subsection{How should you prepare for this class?}
Mathematics is something that needs to be practiced to learn and understand it. In many cases this means that students are given hordes of practice problems and told to complete them all for a grade; the hope being that you will gain understanding by repetition. Unfortunately in my experience, students often memorize the process, but don't stop to consider \textit{why} they are doing whatever they are doing (something often exasperated by not getting a satisfactory answer to the "why am I doing this" question). Although there is definitely value in repetitious practice, the value is usually gained by hoping that the repetition does one of two things;
\begin{enumerate}
\item More practice means it is less likely to make simple computational mistakes. Your teachers/professors are usually not much better at computation than you are, but they have vastly more experience and practice, meaning they make far fewer mistakes.
\item The ``why does this work" will suddenly ``click" into place. The hope is that by repeatedly practicing, the human instinct of pattern recognition will suddenly pick up on the underlying structure that is making whatever technique you are practicing suddenly clear as to why it works. Although this happens (and is often necessary, especially in higher level math) it can only happen if the student is looking for it. So the ``turn my brain off and get this done" approach is usually rather antithetical to the intent of the homework.
\end{enumerate}

This means that it is important, not only to practice the skills we discuss, but to constantly ask yourself the following questions;
\begin{itemize}
\item Why does this work?
\item What steps are necessary (and what steps are not necessary), for a given problem?
\item What else could I use this technique on? 
\item What does this technique really require?
\end{itemize}

My teaching philosophy is to put the impetus (responsibility) of learning on the student. I will answer any questions, and provide limitless practice in the form of reviews (see the syllabus for more information). Your TAs (and other TAs, and I) have office hours you can come and ask questions and get more information/help to understand techniques and content. But in the end it comes down to this: You get out of this course what you put into it. If you don't do any optional work, you will be lucky to pass the class, and you will \emph{definitely not} pass calculus 1.

\subsubsection{Final thoughts (TLDR)}
In summation:
\begin{itemize}
\item If you don't engage with this class and this material, you will be lucky to pass, and you will be among the (roughly) half of all calculus one students that fail on their first attempt. 
\item If you find that you are struggling and don't know why, or aren't sure what to do to do better, \emph{talk to your TA or myself!}. We are here as resources to help you learn and get better at the things we are teaching.
\item If you expect this class to be like previous math classes you have taken, you will be surprised. Hopefully this is a good thing for you, maybe it will be a bad thing for you, but either way this will be a very different kind of math class.
\item Practice does not make perfect. If you just go through the motions of doing endless problems without stopping to ask yourself any questions about what you are doing, you might as well not have done them in the first place. This is true of any practice in any skill, math or otherwise.
\end{itemize}


