\documentclass{ximeraXloud}

\title{Parent Functions}
\begin{document}
\begin{abstract}
    This section provides the specific parent functions you should know.
\end{abstract}
\maketitle


%% Necessary code to graph in the following sections:
\pgfplotsset{my style/.append style={axis x line=middle, axis y line=middle, xlabel={$x$}, ylabel={$y$} }}



\subsection*{Parent functions and notable features}
    This section will present each of the function types from the library of functions list, give it's parent function, and then any relevant information on that function.

    \subsubsection*{Polynomials}
        Polynomials are very generic and ubiquitous, so they require a general note prior to discussing the various specific types. Polynomials have been studied for more than 5000 years and unfortunately this means that there are a \textit{lot} of things to learn about them, even without going into calculus. Students should expect to spend nearly as much time studying and learning techniques in the section on polynomials as we do on almost all the other function types combined. Generally the study polynomials is split up into the study of linear polynomials (degree 1 or 0), quadratic polynomials (degree 2), cubic polynomials (degree 3), and `higher order polynomials' (degree 4 and higher). There are good reasons for this that we will discuss in the exploration on Polynomials section but for this section we also divide polynomials up in this way.
        
        \textbf{Linear Polynomials:} Linear polynomials are perhaps the single most well studied function in existence and simultaneously the most common function to occur in nature. For now we will merely note that the parent function is also one of the nicest; $f(x) = x$ which is also sometimes referred to as ``the identity function".%
        \footnote{
            This is for some really abstract reasons, most of which we won't cover in this class. This will come up tangentially when we discuss function composition in a future topic
            }
        Linear polynomials are typically used in situations where two quantities are just a constant multiple of one another (or off by a constant factor). Such multiples may represent things like price, speed, or mass.
        \begin{description}
            \item[Parent Function:] The parent function for a linear polynomial is $p(x) = x$
            
            \begin{minipage}{\textwidth}\item[Graph of Parent Function:]\hspace*{0pt} \\
                \begin{center}
                    \begin{tikzpicture}
                        \begin{axis}[my style, xtick={-3,-2,...,3}, ytick={-3,-2,...,3},
                            xmin=-3, xmax=3, ymin=-3, ymax=3]
                            \addplot[<->,domain=-3:3, samples=300]{x}
                                node [pos=0.8, above left] {$y=x$};
                        \end{axis}
                    \end{tikzpicture}
                \end{center}
            \end{minipage}
            
            \item[Notable Features of Graph:] The notable features are:
            \begin{itemize}
                \item There is a point of interest; (0,0) which is an important point for a variety of reasons we will discuss in the future. 
                \item The shape is a perfectly straight line.
            \end{itemize}
            
            \item[Example usage:] Examples where linear functions are used occur in many physical and natural applications. For example;
            \begin{itemize}
                \item Distance is a linear function of time and speed; specifically $d = rt$.
                \item Calculating cost given a number of items and price per item. For example $C = pn$ for $C = $ cost, $p = $ price, and $n = $ number of items.
                \item Weight is a linear function of mass, since your weight is the force\footnote{really acceleration but this isn't a physics class} of gravity times your mass. Specifically $W = Gm$.
                \item The states of spin for electrons can be represented by a linear function.
            \end{itemize}
        \end{description}
        
        \textbf{Quadratic Polynomials:} Quadratic polynomials, or polynomials of degree two, are also incredibly well studied as they also have very special roles in mathematical history. The parent function is the classic parabola; $f(x) = x^2$. 
        
        \begin{description}
            \item[Parent Function:] The parent function for a quadratic polynomial is $p(x) = x^2$\\
            \begin{minipage}{\textwidth}\item[Graph of Parent Function:]\hspace*{0pt} \\
                \begin{center}
                    \begin{tikzpicture}
                        \begin{axis}[my style, xtick={-4,-3,...,3,4}, ytick={-1,0,...,7},
                            xmin=-5, xmax=5, ymin=-2, ymax=8]
                            \addplot[<->,domain=-2.5:2.5, samples=300]{x^2}
                                node [pos=0.8, above right] {$y=x^2$};
                        \end{axis}
                    \end{tikzpicture}
                \end{center}
            \end{minipage}
            
            \item[Notable Features of Graph:] The notable features are:
            \begin{itemize}
                \item There is a point of interest; (0,0) which is often referred to as the `vertex' of the quadratic and represents (on the parent function specifically) the lowest point on the graph.
                \item The shape is called a parabola.%
                \item This is a smooth curve, as oppose to the absolute value function below; this fact is actually exceptionally important for calculus one.
            \end{itemize}
            
            \item[Example usage:] Examples where quadratic functions also occur in many physical and natural applications. For example;
            \begin{itemize}
                \item Ballistic arcs (such as throwing a ball through the air, or firing a bullet) following a (more or less) quadratic curve. 
                \item The effect of acceleration on location (such as how pushing the gas pedal in a car effects your distance traveled on the road) is a quadratic.
            \end{itemize}
        \end{description}
        
        \textbf{Cubic Polynomials:} Cubic polynomials, or polynomials of degree three are primarily important because they are the first non-linear polynomial of odd degree. We will discuss why we care about such things in the section on polynomials specifically.
        
        \begin{description}
            \item[Parent Function:] The parent function for a cubic polynomial is $p(x) = x^3$\\
            \begin{minipage}{\textwidth}\item[Graph of Parent Function:]\hspace*{0pt} \\
                \begin{center}
                    \begin{tikzpicture}
                        \begin{axis}[my style, xtick={-3,-2,-1,...,2,3}, ytick={-8,-7,...,8},
                            xmin=-4, xmax=4, ymin=-8.5, ymax=8.5]
                            \addplot[<->,domain=-2:2, samples=300]{x^3}
                                node [pos=0.8, above right] {$y=x^3$};
                        \end{axis}
                    \end{tikzpicture}
                \end{center}
            \end{minipage}
            
            \item[Notable Features of Graph:] The notable features are:
            \begin{itemize}
                \item There is a point of interest; (0,0) which is sometimes referred to as the `vertex', but more often referred to as a (the) point of (rotational) symmetry.
                \item It is (deceptively) important that this function is (strictly) increasing; meaning that, at any point, if you go to the right of that point on the graph, the graph will go up.
            \end{itemize}
            \item[Example usage:] Examples where cubic functions genuinely occur tend to be more rare as they are more often used as approximations of actual behavior, rather than true models of specific behavior. Nonetheless, some examples exist;
            
            \begin{itemize}
                \item Most impulse (sudden rapid changes of acceleration) effects are modeled by cubic polynomials.
                \item Efficiency of space usage in three dimensional space is modeled by cubic polynomials.
            \end{itemize}
        \end{description}
        
        \textbf{Higher (order) Polynomials:} We won't discuss higher order polynomials in this section other than to say that for any given degree, the parent function will be the function of $x$ raised to that degree. Thus for a fifth degree polynomial, the parent function is $p(x) = x^5$. We will go into this in greater detail in the polynomial exploration section.


    \subsubsection*{Radicals}
        \begin{description}
            \item[Parent Function:] The most common radical function we consider is the square root function, whose parent function is $p(x) = \sqrt{x}$.\\
            \begin{minipage}{\textwidth}\item[Graph of Parent Function:]\hspace*{0pt} \\
                \begin{center}
                    \begin{tikzpicture}
                        \begin{axis}[my style, xtick={-1,0,...,4}, ytick={-1,0,...,3},
                            xmin=0, xmax=5, ymin=-1, ymax=3]
                            \addplot[->,domain=0:4, samples=300]{sqrt(x)}
                                node [pos=0.8, above left] {$y=\sqrt{x}$};
                        \end{axis}
                    \end{tikzpicture}
                \end{center}
            \end{minipage}
            \item[Notable Features of Graph:] The notable features are:
            \begin{itemize}
                \item A point of special interest is the point (0,0) on the parent function. This is sometimes referred to as the `vertex' or `origin' of the function.
                \item The restricted (natural) domain of the parent function is also of special interest as it is the non-negative real numbers, ie the interval $[0,\infty)$.
                \item The slope of the graph near the origin is of interest as it seems to slope to an almost straight up and down curve as it gets really close to the origin.\footnote{This is something that is made precise in calculus one.}
            \end{itemize}
            \item[Example usage:]
            \begin{itemize}
                \item One of the chief roles of radical functions is to undo a polynomial term. That is to say that a radical is typically used to eliminate a power, such as when you square root both sides of $x^2 = 4$ to solve for $x$.
                \item Another usage is to find a geometric mean, which is a number that is in the middle of two numbers in a multiplicative, rather than additive, sense. This is something studied and utilized extensively in statistics and/or probability, but we won't be studying it much in this class or in calculus.
            \end{itemize}
        \end{description}




    \subsubsection*{Absolute Value}
        
        \begin{description}
            \item[Parent Function:] The parent function is $p(x) = |x|$\\
            \begin{minipage}{\textwidth}\item[Graph of Parent Function:]\hspace*{0pt} \\
                \begin{center}
                    \begin{tikzpicture}
                        \begin{axis}[axis x line=middle, axis y line=middle, xlabel={$x$}, ylabel={$y$}, axis equal, xtick={-4,-3,...,3,4}, ytick={-1,0,...,4},
                            xmin=-5, xmax=5, ymin=-2, ymax=5]
                            \addplot[<->,domain=-4:4, samples=300]{abs(x)}
                                node [pos=0.8, right] {$y=|x|$};
                        \end{axis}
                    \end{tikzpicture}
                \end{center}
            \end{minipage}
            \item[Notable Features of Graph:] The notable features are:
            \begin{itemize}
                \item A point of interest (on the parent function) is the point (0,0), which is sometimes referred to as the `vertex' or `reflection' point.
                \item The sharpness of the change in slope at the reflection point is worth noting, this is referred to as a `corner' and is something that is studied closely in calculus 1.
                \item Absolute values are used in calculus primarily for the sake of the effect of the corner at the reflection point, and the nature of its graph.
            \end{itemize}
            \item[Example usage:] Whenever we wish to ensure a value is positive, an absolute value is a useful tool.
            \begin{itemize}
                \item Distances are usually represented by absolute value. In fact, one interpretation of the absolute value of a number\footnote{given to us by the ancient greeks, who viewed all numbers in this way} is the ``distance from zero"
                \item Accuracy is another area where absolute value is often used; specifically if we want to know which of several numbers are `closer' to the correct value, we can interpret `closer' meaning a type of distance, and use the absolute value.
            \end{itemize}
        \end{description}
        
    
    
    
    
    \subsubsection*{Exponentials}
    
        \begin{description}
            \item[Parent Function:] The most often used exponential function is the one with the `natural' base; whose parent function is $e^x$.\\
            \begin{minipage}{\textwidth}\item[Graph of Parent Function:]\hspace*{0pt} \\
                \begin{center}
                    \begin{tikzpicture}
                        \begin{axis}[my style, xtick={-3,-2,...,1,2}, ytick={-1,0,...,8},
                            xmin=-4, xmax=3, ymin=-1, ymax=9]
                            \addplot[<->,domain=-3:2, samples=300]{e^x}
                                node [pos=0.8, right] {$y=e^x$};
                        \end{axis}
                    \end{tikzpicture}
                \end{center}
            \end{minipage}
            \item[Notable Features of Graph:] The notable features are:
            \begin{itemize}
                \item A point of interest is the point (0,1) for a variety of reasons which we will discuss more in the section on exploring exponential functions.
                \item The graph has a horizontal asymptote at $y=0$ to the left.
                \item The graph is strictly increasing, like the cubic polynomial.
                \item The growth (sharpness) of the increase to the right is greater than any other function on this list.
            \end{itemize}
            \item[Example usage:] Exponential functions should be used whenever you are discussing a situation where the change in something over time depends on the value that is changing at each moment. This is somewhat vague, but specific examples are easy to come by, such as;
            \begin{itemize}
                \item The most common place to see exponential functions used are exponential growth functions such as earning compound interest, or population growth. In these cases, the more of the thing you have (the more money, or more population) the more you get each growth iteration. 
                \item Another common place to see exponential functions used is the spread of disease or other contagion. The more people infected, the more people there are to infect others, and so it's (initial) progress tends to be exponential.
            \end{itemize}
        \end{description}
    
    
    
    
    \subsubsection*{Logarithms}
        Logarithms are classically defined as the inverse function to the exponential function. In some sense they are the most `artificial' function in the library of functions because it doesn't really arise on it's own in very many contexts.
        
        \begin{description}
            \item[Parent Function:] The most common log to consider is the so-called `natural log', whose parent function is $p(x) = \ln(x)$.\footnote{In industry, most software will actually denote the natural log with $\log$ since nobody actually uses the so-called `common log' or log base 10 anymore. For historical reasons however most math classes still use $\ln$ for the natural log and $\log$ for log base ten and we conform to that convention here.}\\
            \begin{minipage}{\textwidth}\item[Graph of Parent Function:]\hspace*{0pt} \\
                \begin{center}
                    \begin{tikzpicture}
                        \begin{axis}[my style, xtick={-1,0,...,9}, ytick={-3,-2,...,3},
                            xmin=-1, xmax=10, ymin=-4, ymax=4]
                            \addplot[<->,domain=.001:9, samples=300]{ln(x)}
                                node [pos=0.8, above left] {$y=\ln(x)$};
                        \end{axis}
                    \end{tikzpicture}
                \end{center}
            \end{minipage}
            
            \item[Notable Features of Graph:] The notable features are:
            \begin{itemize}
                \item A point of interest is the point (1,0) as it is the point that corresponds to the point (0,1) of the inverted exponential function.
                \item It is also important to notice that the log function has a restricted domain of $(0,\infty)$, but it's range is all real numbers.
                \item The log function is an increasing function like we mentioned for the cubic polynomial and exponential function.
            \end{itemize}
            
            \item[Example usage:] The log function is most commonly used to invert the process of an exponential function, and so natural occurrences of the log function are uncommon. Nonetheless such examples exist;
            \begin{itemize}
                \item The scale upon which we measure the magnitude of earthquakes is logarithmic with a base of 10. This means that a magnitude 5 earthquake is ten times stronger than a magnitude 4 earthquake and a thousand times more `powerful' than a magnitude 2 earthquake.
                \item The half-life decay rate of radioactivity is logarithmic. In essence this is because half-life decay is exponential growth in reverse, instead of getting `twice as big' every $t$ years, it gets `half as big' every $t$ years. This could be done exponentially, but it is more generally asked; ``when will it be safe again?" which is equivalent to asking ``When will it be below a radioactivity level of $X$?", and to determine that we need to use a logarithm.
            \end{itemize}
        \end{description}

    \subsubsection*{Piecewise Functions and Rational Functions}
        In the case of piecewise and rational functions, these are more ways to combine other functions rather than their own fundamental functions. For this reason they do not have parent functions in the same way as the other functions, although we will discuss how to deal with this issue in their respective sections.

\end{document}