\documentclass{ximeraXloud}

\title{How to Access Xronos}
\begin{document}
\begin{abstract}
A Tutorial on accessing Xronos and how grades work.
\end{abstract}
\maketitle

There are a few very important points about accessing Xronos. Chances are good, if you are here, you have passed the first hurtle, the ``500 error" that says you don't have access to Xronos. Nonetheless in case you find yourself having this problem in the future, we will cover this error first.

If you are having a 500 error, it is almost certainly because you are on either the `ufguest' or the `eduroam' networks. In many cases the `eduroam' network should work fine, but the `ufguest' network absolute \textit{does not work}. If you are on the `eduroam' network and getting a 500 error, you should contact the Xronos Team (either directly or by contacting your lecturer or TA) as they need some information to prevent the eduroam network from having this problem in the future. Either way, you can resolve the 500 error by logging onto the `uf' wifi network directly.

The network problems only apply to the UF wifi networks mentioned above. If you are living somewhere off-campus and wish to access Xronos, it should work on any off-campus wifi that has normal internet access.

\begin{problem}
    Which of the following wifi networks can you use to ensure that you \textit{definitely} will have access to Xronos?
    \begin{selectAll}
        \choice{ufguest}
        \choice[correct]{uf}
        \choice{eduroam}
        \choice[correct]{any off-campus network with regular internet access}
    \end{selectAll}
    \begin{problem}
        What error would you get if you use the `ufguest' wifi, or (under somewhat rare conditions) the `eduroam' wifi?
        \begin{multipleChoice}
            \choice{Gremlins will pop out of your screen and burn the world!}
            \choice[correct]{You will see a ``500 error"}
            \choice{You will see a ``404 error"}
            \choice{Your homework will magically do itself and you never need to worry about doing it.}
        \end{multipleChoice}
    \end{problem}
\end{problem}

The other important part about accessing Xronos is how grades are determined. In order to get a grade in Xronos you MUST LOG INTO XRONOS THROUGH THE CANVAS LINK. This is absolutely necessary because the process of clicking the button that loads Xronos from within Canvas forms an ``LTI" link to the Canvas gradebook which allows Xronos to update your grade as you do work. This means that if you access Xronos through any other means, such as a bookmark or typing in the URL, that LTI link may not (and probably won't) be able to form, which means any work you do in Xronos WILL NOT GO BACK TO CANVAS. Thus despite getting work done inside Xronos, you won't receive any credit for it, which is probably not what you are hoping for. Also keep in mind that gradesync is typically quite quick, but it can lag behind at times depending on the level of activity the Xronos server is experiencing, so you may need to wait 15-20 minutes to see if the grades in the Canvas gradebook have updated.

\begin{problem}
    When accessing Xronos you need to do so by...
    \begin{selectAll}
        \choice{Getting the url from a friend.}
        \choice{Typing in the url from Canvas.}
        \choice[correct]{Clicking on the provided link in Canvas.}
        \choice{Bookmarking the url from Canvas, and use the bookmark for future access.}
    \end{selectAll}
    \begin{problem}
        When accessing Xronos throughout the semester you can...
        \begin{selectAll}
            \choice[correct]{Only use the button in Canvas.}
            \choice{Use browser history to get the url.}
            \choice{Use a bookmark to go back to assignments.}
            \choice{Get the url from someone else in the course.}
        \end{selectAll}
    \end{problem}
\end{problem}

\begin{problem}
    If you are doing work in Xronos and answering things correctly, but the grade isn't updating in Canvas then which of the following are likely the problem?
    \begin{selectAll}
        \choice[correct]{The LTI link has not been formed, probably due to not accessing Xronos by clicking the assignment in Canvas.}
        \choice{Xronos has caught fire and is busy consuming the souls of IT.}
        \choice{Xronos is down for maintenance, so grade sync is broken.}
        \choice[correct]{There may be a slight delay to update the grades, so you can wait 15-20 minutes and see if the grade has updated.}
    \end{selectAll}
\end{problem}



\end{document}