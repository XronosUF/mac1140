\documentclass{ximeraXloud}
\title{Functions Practice 1}
\begin{document}
\input{Useful-Sage-Macros}

\begin{sagesilent}
p1pwr = RandInt(1,3)
p1funcvec = [x^p1pwr, e^x, log(abs(x)), sqrt(x)]
p1c1 = NonZeroInt(-5,5)
p1c2 = RandInt(-5,5)
p1c3 = NonZeroInt(-5,5)
p1c4 = RandInt(-5,5)

p1choice1 = RandInt(0,3)
p1choice2 = NonZeroInt(0,3,[3-p1choice1])

p1func1 = p1c1*p1funcvec[p1choice1](x + p1c2)
p1func2 = p1c3*p1funcvec[p1choice2](x + p1c4)

p1ans1 = p1func1(p1func2(x))
p1ans2 = p1func2(p1func1(x))


p2pwr = RandInt(1,3)
p2funcvec = [x^p2pwr, e^x, log(abs(x)), sqrt(x)]
p2c1 = NonZeroInt(-5,5)
p2c2 = RandInt(-5,5)
p2c3 = NonZeroInt(-5,5)
p2c4 = RandInt(-5,5)

p2choice1 = RandInt(0,3)
p2choice2 = NonZeroInt(0,3,[3-p2choice1])

p2func1 = p2c1*p2funcvec[p2choice1](x + p2c2)
p2func2 = p2c3*p2funcvec[p2choice2](x + p2c4)

p2ans1 = p2func1(p2func2(x))
p2ans2 = p2func2(p2func1(x))


p3pwr = RandInt(1,3)
p3funcvec = [x^p3pwr, e^x, log(abs(x)), sqrt(x)]
p3c1 = NonZeroInt(-5,5)
p3c2 = RandInt(-5,5)
p3c3 = NonZeroInt(-5,5)
p3c4 = RandInt(-5,5)

p3choice1 = RandInt(0,3)
p3choice2 = NonZeroInt(0,3,[3-p3choice1])

p3func1 = p3c1*p3funcvec[p3choice1](x + p3c2)
p3func2 = p3c3*p3funcvec[p3choice2](x + p3c4)

p3ans1 = p3func1(p3func2(x))
p3ans2 = p3func2(p3func1(x))


p4pwr = RandInt(1,3)
p4funcvec = [x^p4pwr, e^x, log(abs(x)), sqrt(x)]
p4c1 = NonZeroInt(-5,5)
p4c2 = RandInt(-5,5)
p4c3 = NonZeroInt(-5,5)
p4c4 = RandInt(-5,5)

p4choice1 = RandInt(0,3)
p4choice2 = NonZeroInt(0,3,[3-p4choice1])

p4func1 = p4c1*p4funcvec[p4choice1](x + p4c2)
p4func2 = p4c3*p4funcvec[p4choice2](x + p4c4)

p4ans1 = p4func1(p4func2(x))
p4ans2 = p4func2(p4func1(x))




\end{sagesilent}

\begin{problem}
Consider the following functions: $f(x) = \sage{p1func1}$ and $g(x) = \sage{p1func2(x)}$. Compute:

$f(g(x)) = \answer{\sage{p1ans1}}$

$g(f(x)) = \answer{\sage{p1ans2}}$

$(f\circ g)(x) = \answer{\sage{p1ans1}}$

$((g\circ f)(x) = \answer{\sage{p1ans2}}$
\end{problem}


\begin{problem}
Consider the following functions: $f(x) = \sage{p2func1}$ and $g(x) = \sage{p2func2(x)}$. Compute:

$f(g(x)) = \answer{\sage{p2ans1}}$

$g(f(x)) = \answer{\sage{p2ans2}}$

$(f\circ g)(x) = \answer{\sage{p2ans1}}$

$((g\circ f)(x) = \answer{\sage{p2ans2}}$
\end{problem}


\begin{problem}
Consider the following functions: $f(x) = \sage{p3func1}$ and $g(x) = \sage{p3func2(x)}$. Compute:

$f(g(x)) = \answer{\sage{p3ans1}}$

$g(f(x)) = \answer{\sage{p3ans2}}$

$(f\circ g)(x) = \answer{\sage{p3ans1}}$

$((g\circ f)(x) = \answer{\sage{p3ans2}}$
\end{problem}


\begin{problem}
Consider the following functions: $f(x) = \sage{p4func1}$ and $g(x) = \sage{p4func2(x)}$. Compute:

$f(g(x)) = \answer{\sage{p4ans1}}$

$g(f(x)) = \answer{\sage{p4ans2}}$

$(f\circ g)(x) = \answer{\sage{p4ans1}}$

$((g\circ f)(x) = \answer{\sage{p4ans2}}$
\end{problem}



\end{document}