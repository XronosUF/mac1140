\documentclass{ximeraXloud}

\title{Virtues of CBT}
\begin{document}
\begin{abstract}
Contextual Based Learning (CBT) has many virtues, knowing why we are learning how we are will help your studying and learning process.
\end{abstract}
\maketitle

It usually surprises people to learn that most mathematicians have spectacularly terrible memories. In a setting where most of your ``math" education is centered around endless memorizing of symbols and symbol manipulation, it may seem like you're suppose to memorize everything. In fact, this is the \textit{exactly opposite} of how math should be learned.

In reality, memorizing something is very different than learning it. Often you start with memorizing, but think about this; if you were to see a word, let's say the word ``toast", you know what that means. But, do you know what it means because you memorized it? Or do you know what it means because you've simply seen and used it so much that you ``just know it"? Chances are, when you read that word you just sort of ``knew" what it meant, and it might have even evoked an idea of tasty (or awful... if you're some kind of barbarian) breakfast food. In contrast, if you read the word ``$\dot{\alpha}\pi o \pi\upsilon\rho\dot{\iota}\alpha\varsigma$", chances are that not only would you not know that it's ancient greek for ``toast", but you \textit{would know} that it's not English. Is that because you've memorized every English word? Or have you memorized every Non-English word? That's pretty unlikely, rather you've probably \textit{learned} how to recognize if something is English or not (such as the funky alphabet).

This is the fundamental difference between memorization and learning. If you learn something then it becomes ``obvious" what contexts make sense for that thing, and what contexts don't. \textbf{Mathematics is the language of deductive reasoning}, and just like English, it should be obvious when what you've written makes sense and when it doesn't. Thus if you ever find yourself in the situation of writing something down because ``that's what I'm suppose to do" and can't give any better justification, chances are that you've memorized something, but haven't learned it yet. This also means you should talk to your TA or myself in order to transition from ``memorizing" to ``learning" the concepts of this class. It will save you vast amounts of brain space and effort (after all, consider how many words you know, and how relatively little effort you have to spend to maintain that knowledge).


\begin{question}
    What is the point of CBT? (Select all that apply)
    \begin{selectAll}
        \choice[correct]{To put content in context to improve retention.}
        \choice{To give a technique that is better for memorizing information.}
        \choice[correct]{To aid in learning, rather than memorizing, information.}
        \choice[correct]{To help build intuition and understanding of content, allowing one to connect new content to already understood content.}
    \end{selectAll}
\end{question}

\end{document}