\documentclass{ximeraXloud}
\usepackage{longdivision}
\usepackage{polynom}
\usepackage{float}% Use `H' as the figure optional argument to force it's vertical placement to conform to source.
%\usepackage{caption}% Allows us to describe the figures without having "figure 1:" in it. :: Apparently Caption isn't supported.
%    \captionsetup{labelformat=empty}% Actually does the figure configuration stated above.
\usetikzlibrary{arrows.meta,arrows}% Allow nicer arrow heads for tikz.
\usepackage{gensymb, pgfplots}
\usepackage{tabularx}
\usepackage{arydshln}



\graphicspath{
  {./}
  {./explorePolynomials/}
  {./exploreRadicals/}
  {./graphing/}
}

%% Default style for tikZ
\pgfplotsset{my style/.append style={axis x line=middle, axis y line=
middle, xlabel={$x$}, ylabel={$y$}, axis equal }}


%% Because log being natural log is too hard for people.
\let\logOld\log% Keep the old \log definition, just in case we need it.
\renewcommand{\log}{\ln}


%%% Changes in polynom to show the zero coefficient terms
\makeatletter
\def\pld@CF@loop#1+{%
    \ifx\relax#1\else
        \begingroup
          \pld@AccuSetX11%
          \def\pld@frac{{}{}}\let\pld@symbols\@empty\let\pld@vars\@empty
          \pld@false
          #1%
          \let\pld@temp\@empty
          \pld@AccuIfOne{}{\pld@AccuGet\pld@temp
                            \edef\pld@temp{\noexpand\pld@R\pld@temp}}%
           \pld@if \pld@Extend\pld@temp{\expandafter\pld@F\pld@frac}\fi
           \expandafter\pld@CF@loop@\pld@symbols\relax\@empty
           \expandafter\pld@CF@loop@\pld@vars\relax\@empty
           \ifx\@empty\pld@temp
               \def\pld@temp{\pld@R11}%
           \fi
          \global\let\@gtempa\pld@temp
        \endgroup
        \ifx\@empty\@gtempa\else
            \pld@ExtendPoly\pld@tempoly\@gtempa
        \fi
        \expandafter\pld@CF@loop
    \fi}
\def\pld@CMAddToTempoly{%
    \pld@AccuGet\pld@temp\edef\pld@temp{\noexpand\pld@R\pld@temp}%
    \pld@CondenseMonomials\pld@false\pld@symbols
    \ifx\pld@symbols\@empty \else
        \pld@ExtendPoly\pld@temp\pld@symbols
    \fi
    \ifx\pld@temp\@empty \else
        \pld@if
            \expandafter\pld@IfSum\expandafter{\pld@temp}%
                {\expandafter\def\expandafter\pld@temp\expandafter
                    {\expandafter\pld@F\expandafter{\pld@temp}{}}}%
                {}%
        \fi
        \pld@ExtendPoly\pld@tempoly\pld@temp
        \pld@Extend\pld@tempoly{\pld@monom}%
    \fi}
\makeatother




%%%%% Code for making prime factor trees for numbers, taken from user Qrrbrbirlbel at: https://tex.stackexchange.com/questions/131689/how-to-automatically-draw-tree-diagram-of-prime-factorization-with-latex

\usepackage{forest,mathtools,siunitx}
\makeatletter
\def\ifNum#1{\ifnum#1\relax
  \expandafter\pgfutil@firstoftwo\else
  \expandafter\pgfutil@secondoftwo\fi}
\forestset{
  num content/.style={
    delay={
      content/.expanded={\noexpand\num{\forestoption{content}}}}},
  pt@prime/.style={draw, circle},
  pt@start/.style={},
  pt@normal/.style={},
  start primeTree/.style={%
    /utils/exec=%
      % \pt@start holds the current minimum factor, we'll start with 2
      \def\pt@start{2}%
      % \pt@result will hold the to-be-typeset factorization, we'll start with
      % \pgfutil@gobble since we don't want a initial \times
      \let\pt@result\pgfutil@gobble
      % \pt@start@cnt holds the number of ^factors for the current factor
      \def\pt@start@cnt{0}%
      % \pt@lStart will later hold "l"ast factor used
      \let\pt@lStart\pgfutil@empty,
    alias=pt-start,
    pt@start/.try,
    delay={content/.expanded={$\noexpand\num{\forestove{content}}
                            \noexpand\mathrlap{{}= \noexpand\pt@result}$}},
    primeTree},
  primeTree/.code=%
    % take the content of the node and save it in the count
    \c@pgf@counta\forestove{content}\relax
    % if it's 2 we're already finished with the factorization
    \ifNum{\c@pgf@counta=2}{%
      % add the factor
      \pt@addfactor{2}%
      % finalize the factorization of the result
      \pt@addfactor{}%
      % and set the style to the prime style
      \forestset{pt@prime/.try}%
    }{%
      % this simply calculates content/2 and saves it in \pt@end
      % this is later used for an early break of the recursion since no factor
      % can be greater then content/2 (for integers of course)
      \edef\pt@content{\the\c@pgf@counta}%
      \divide\c@pgf@counta2\relax
      \advance\c@pgf@counta1\relax % to be on the safe side
      \edef\pt@end{\the\c@pgf@counta}%
      \pt@do}}

%%% our main "function"
\def\pt@do{%
  % let's test if the current factor is already greather then the max factor
  \ifNum{\pt@end<\pt@start}{%
    % great, we're finished, the same as above
    \expandafter\pt@addfactor\expandafter{\pt@content}%
    \pt@addfactor{}%
    \def\pt@next{\forestset{pt@prime/.try}}%
  }{%
    % this calculates int(content/factor)*factor
    % if factor is a factor of content (without remainder), the result will
    % equal content. The int(content/factor) is saved in \pgf@temp.
    \c@pgf@counta\pt@content\relax
    \divide\c@pgf@counta\pt@start\relax
    \edef\pgf@temp{\the\c@pgf@counta}%
    \multiply\c@pgf@counta\pt@start\relax
    \ifNum{\the\c@pgf@counta=\pt@content}{%
      % yeah, we found a factor, add it to the result and ...
      \expandafter\pt@addfactor\expandafter{\pt@start}%
      % ... add the factor as the first child with style pt@prime
      % and the result of int(content/factor) as another child.
      \edef\pt@next{\noexpand\forestset{%
        append={[\pt@start, pt@prime/.try]},
        append={[\pgf@temp, pt@normal/.try]},
        % forest is complex, this makes sure that for the second child, the
        % primeTree style is not executed too early (there must be a better way).
        delay={
          for descendants={
            delay={if n'=1{primeTree, num content}{}}}}}}%
    }{%
      % Alright this is not a factor, let's get the next factor
      \ifNum{\pt@start=2}{%
        % if the previous factor was 2, the next one will be 3
        \def\pt@start{3}%
      }{%
        % hmm, the previos factor was not 2,
        % let's add 2, maybe we'll hit the next prime number
        % and maybe a factor
        \c@pgf@counta\pt@start
        \advance\c@pgf@counta2\relax
        \edef\pt@start{\the\c@pgf@counta}%
      }%
      % let's do that again
      \let\pt@next\pt@do
    }%
  }%
  \pt@next
}

%%% this builds the \pt@result macro with the factors
\def\pt@addfactor#1{%
  \def\pgf@tempa{#1}%
  % is it the same factor as the previous one
  \ifx\pgf@tempa\pt@lStart
    % add 1 to the counter
    \c@pgf@counta\pt@start@cnt\relax
    \advance\c@pgf@counta1\relax
    \edef\pt@start@cnt{\the\c@pgf@counta}%
  \else
    % a new factor! Add the previous one to the product of factors
    \ifx\pt@lStart\pgfutil@empty\else
      % as long as there actually is one, the \ifnum makes sure we do not add ^1
      \edef\pgf@tempa{\noexpand\num{\pt@lStart}\ifnum\pt@start@cnt>1 
                                           ^{\noexpand\num{\pt@start@cnt}}\fi}%
      \expandafter\pt@addfactor@\expandafter{\pgf@tempa}%
    \fi
    % setup the macros for the next round
    \def\pt@lStart{#1}% <- current (new) factor
    \def\pt@start@cnt{1}% <- first time
  \fi
}
%%% This simply appends "\times #1" to \pt@result, with etoolbox this would be
%%% \appto\pt@result{\times#1}
\def\pt@addfactor@#1{%
  \expandafter\def\expandafter\pt@result\expandafter{\pt@result \times #1}}

%%% Our main macro:
%%% #1 = possible optional argument for forest (can be tikz too)
%%% #2 = the number to factorize
\newcommand*{\PrimeTree}[2][]{%
  \begin{forest}%
    % as the result is set via \mathrlap it doesn't update the bounding box
    % let's fix this:
    tikz={execute at end scope={\pgfmathparse{width("${}=\pt@result$")}%
                         \path ([xshift=\pgfmathresult pt]pt-start.east);}},
    % other optional arguments
    #1
    % And go!
    [#2, start primeTree]
  \end{forest}}
\makeatother


\providecommand\tabitem{\makebox[1em][r]{\textbullet~}}
\providecommand{\letterPlus}{\makebox[0pt][l]{$+$}}
\providecommand{\letterMinus}{\makebox[0pt][l]{$-$}}



\title{Rational Root Theorem}
\begin{document}
\begin{abstract}
Find factors via rational root theorem
\end{abstract}
\maketitle

The rational root theorem is one of the most powerful, but least efficient, mechanisms for finding roots of a polynomial. You may have noticed in our discussion of factoring quadratics by factoring their coefficients above, we had talked about how zeros of a polynomial are of the form $\frac{b}{a}$ where $a$ divided the leading coefficient, and $b$ divided the constant in the polynomial. In our specific example above we were looking at the factored form $(3x + 4)(2x - 6) = 6x^2 - 10x - 24$, and then we had noticed that the leading coefficient $6$ comes from the $(3x) \cdot (2x)$ to get $6x^2$ and the $-24$ came from the $4 \cdot (-6)$... which is to say, if a quadratic of the form $a_2x^2 + a_1x + a_0$ has a root of the form $(ax - b)$, then $a$ should divide evenly into $a_2$ and $b$ should divide evenly into $a_0$.

\subsection*{Man, I had almost managed to forget that insanity... and now my head is hurting}
    Well... sorry? The good news is, that we can (hopefully) see that this has nothing specifically to do with quadratics. If we chain a bunch of these roots together, the leading coefficient will always be formed by taking the product of all the $x$-terms in each of the roots, and the constant coefficient (of the polynomial) will be formed by taking the product of all the constant coefficients in the roots. This is the rational root theorem.
    
    \begin{theorem}
        Let $p(x) = a_nx^n + a_{n-1}x^{n-1} + \dots + a_1x + a_0$ be a general polynomial with integer coefficients (ie $a_n, a_{n-1}, ..., a_1, a_0$ are all integers). If $(ax + b)$ is a root ($a$, $b$ integers), then  $a$ divides into $a_n$ evenly and $b$ divides into $a_0$ evenly.
        
        This has a number of equivalent forms but the following is often the most useful:
        Every zero of $p(x)$ that is a rational number, is of the form: $\dfrac{\text{Factor of }a_0}{\text{Factor of }a_n}$.
    \end{theorem}
    
    Before your head splits in half, this is much easier to see with an example.
    
    Consider $p(x) = 10x^5 - 13x^3 + 22x^2 - 3x + 14$. Let's say we want to know: ``what are all possible zeros of $p(x)$ and what are all possible roots of the form $(ax + b)$ for integers $a$ and $b$?"%
        \footnote{Technically there are still infinitely many such roots, so we should include a line like `up to a constant multiple' to avoid the stupid case where one could have a root of, say, $(3x + 3)$ and $3(x + 1)$ which are arguably different roots.}
    
        Notice that the question says ``possible" zeros and roots, not the \textit{actual} zeros or roots. This seems like an impossible task, after all there are infinitely many integers we could randomly plug in... but this is where using the rational root theorem is helpful. To do this then, we need to find what the factors of 14 are and what the factors of 10 are. Writing these out we have the factors of 14 are 1, 2, 7, 14 and the factors of 10 are 1, 2, 5, 10. Notice however, that if a positive number divides evenly, then so does the negative version, so we are actually going to need to list the positive and negative versions of every zero.
    
        According to the rational root theorem then, we can list the possible zeros of $p(x)$ by taking every combination of a factor of the constant coefficient (ie 14) and divided by factors of the leading coefficient (ie 10). This gives us the following list;
        \[
            \pm \left(\frac{1}{1}, \frac{2}{1}, \frac{7}{1}, \frac{14}{1},
            \frac{1}{2}, \frac{2}{2}, \frac{7}{2}, \frac{14}{2},
            \frac{1}{5}, \frac{2}{5}, \frac{7}{5}, \frac{14}{5},
            \frac{1}{10}, \frac{2}{10}, \frac{7}{10}, \frac{14}{10}\right)
        \]
        Which, after we clean it up a little and get rid of duplicates, gets us the following list:
        \[
            \pm \left( 1, 2, 7, 14,
            \frac{1}{2}, \frac{7}{2},
            \frac{1}{5}, \frac{2}{5}, \frac{7}{5}, \frac{14}{5}
            \frac{1}{10}, \frac{7}{10}\right)
        \]
    
        Next we wanted to write out the corresponding roots. Notice that zeros of the root $(ax + b)$ are $-\frac{b}{a}$. Thus, working backward, we can take each of the above zeros and form the corresponding root by taking the denominator as the `$a$' and the numerator as the `$b$' in the form $(ax + b)$. Since we need to have the positive and negative version of each zero we use the form $(ax \pm b)$, which is just saying `both the roots $(ax + b)$ and $(ax - b)$', to denote both at the same time.  %
        So, we get the following list of possible roots:
        \begin{align*}
            (x &\pm 1),     &(x &\pm 2),    &(x &\pm 7),    &(x &\pm 14),   &(2x &\pm 1),   &(2x &\pm 7), \\
            (5x &\pm 1),    &(5x &\pm 2),   &(5x &\pm 7),   &(5x &\pm 14),  &(10x &\pm 1),  &(10x &\pm 7)
        \end{align*}

You can also watch a video of using the rational root theorem to fully factor a polynomial!

\youtube{3uoh4Kpsrq4}


\end{document}