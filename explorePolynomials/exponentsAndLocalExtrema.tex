\documentclass{ximeraXloud}
\input{../preamble}
\title{Exponents and Extrema 2: Local Extrema}
\begin{document}
\begin{abstract}
    This section contains information on how exponents effect local extrema
\end{abstract}
\maketitle


\section*{Local extrema}

Although we lack the analytic methods to tackle local extrema properly%
\footnote{Indeed, this is a major area of study in the first calculus class, comprising about a third of the course}
there is some information we can gather just from the equation itself. Consider the following three graphs;

\begin{description}
    \item The graph of $y = x$;\\
\begin{tikzpicture}
    \begin{axis}[
        axis x line=middle,
        axis y line=middle,
        minor tick num=1,
        x label style={at={(axis description cs:1,0.5)},anchor=south},
        y label style={at={(axis description cs:0.5,1)},anchor=west},
        xlabel={$x$},
        ylabel={$y$},
        xmin=-5,
        xmax=5,
        ymin=-5,
        ymax=5
        ]
    \addplot[<->,domain=-4.5:4.5, samples=300]{x};
    \end{axis}
\end{tikzpicture}
\newpage
\item The graph of $y = x^2$;\\
\begin{tikzpicture}
    \begin{axis}[
        axis x line=middle,
        axis y line=middle,
        minor tick num=1,
        x label style={at={(axis description cs:1,0.1)},anchor=south},
        y label style={at={(axis description cs:0.5,1)},anchor=west},
        xlabel={$x$},
        ylabel={$y$},
        xmin=-4,
        xmax=4,
        ymin=-1,
        ymax=10
        ]
    \addplot[<->,domain=-3:3, samples=300]{x^2};
    \end{axis}
\end{tikzpicture}
\item The graph of $y = \frac{1}{3}x^3 - 3x$; \footnote{Note: This one seems a little strange, but the numbers are chosen just to emphasize the curve and aren't too important, the important thing is that it's a cubic function.}\\
\begin{tikzpicture}
    \begin{axis}[
        axis x line=middle,
        axis y line=middle,
        minor tick num=1,
        x label style={at={(axis description cs:1,0.5)},anchor=south},
        y label style={at={(axis description cs:0.5,1)},anchor=west},
        xlabel={$x$},
        ylabel={$y$},
        xmin=-5,
        xmax=5,
        ymin=-5,
        ymax=5
        ]
    \addplot[<->,domain=-3.5:3.5, samples=300]{1/3*x*(x-3)*(x+3)};
    \end{axis}
\end{tikzpicture}
\end{description}

As we can see, each time we increase the exponent we get another `bend' in the graph. But this need not always happen, after all the graph of $x^3$ doesn't have the same number of bends as the graph of $\frac{1}{3}x^3 - 3x$ that we showed above. Nonetheless, we know that a parabola will always have one bend, and a line will never have any. Continuing this pattern we can eventually come to the following observation; the number of bends in the graph of a polynomial will be, at most, one less than the degree of the polynomial. But notice that each bend is really a local extrema, so a more precise (or mathematical) way to state this observation is;

\begin{lemma}
if $p(x) = a_nx^n + a_{n-1}x^{n-1} + \dots + a_1 x + a_0$ is a polynomial of degree $n$, then there are \textbf{at most} $n-1$ local extrema on the graph of $p(x)$.
\end{lemma}

For example, based on our lemma, if we have a fifth degree polynomial, we know that it will have at most four local extrema, even if we don't know what the graph actually looks like. 

\begin{problem}
    Consider the polynomial $p(x) = 3x^2 -12 x^4 + x^3 - 2x + 1$. What can be said about the extrema of $p(x)$? (Select all that apply)
    \begin{selectAll}
        \choice[correct]{$p(x)$ has an absolute extrema.}
        \choice{$p(x)$ has no absolute extrema.}
        \choice{$p(x)$ has at most one local extrema.}
        \choice{$p(x)$ has at most four local extrema.}
        \choice[correct]{$p(x)$ has, at most, three local extrema.}
        \choice{$p(x)$ has no local extrema.}
        \choice{$p(x)$ has exactly one local extrema.}
    \end{selectAll}
\end{problem}





\end{document}