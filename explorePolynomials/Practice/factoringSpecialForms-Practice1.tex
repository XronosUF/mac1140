\documentclass{ximeraXloud}
\title{Factor Coefficients Method Practice 1}



\begin{document}
\input{Useful-Sage-Macros}

\begin{sagesilent}
###### Problem p1
p1c1 = RandInt(-10,10)
p1c2 = NonZeroInt(-5,5)
p1pwr1 = RandInt(1,3)

p1f1 = (p1c2*x^p1pwr1)^2 - p1c1^2
p1f2 = (p1c2*x^p1pwr1)^3 - p1c1^3
p1tog = RandInt(0,1)
p1f3 = p1f1^p1tog*p1f2^(1-p1tog)

p1f4 = expand(p1f3)


###### Problem p2
p2c1 = NonZeroInt(-5,5)
p2c2 = RandInt(-5,5)
p2pwr1 = RandInt(1,3)
p2pwr2 = RandInt(2,3)
p2f1 = (p2c1*x^p2pwr1 - p2c2)^p2pwr2

p2f2 = expand(p2f1)


###### Problem p3
p3c1 = RandInt(-10,10)
p3c2 = NonZeroInt(-5,5)
p3pwr1 = RandInt(1,3)

p3f1 = (p3c2*x^p3pwr1)^2 - p3c1^2
p3f2 = (p3c2*x^p3pwr1)^3 - p3c1^3
p3tog = RandInt(0,1)
p3f3 = p3f1^p3tog*p3f2^(1-p3tog)

p3f4 = expand(p3f3)


###### Problem p4
p4c1 = NonZeroInt(-5,5)
p4c2 = RandInt(-5,5)
p4pwr1 = RandInt(1,3)
p4pwr2 = RandInt(2,3)
p4f1 = (p4c1*x^p4pwr1 - p4c2)^p4pwr2

p4f2 = expand(p4f1)


###### Problem p5
p5c1 = RandInt(-10,10)
p5c2 = NonZeroInt(-5,5)
p5pwr1 = RandInt(1,3)

p5f1 = (p5c2*x^p5pwr1)^2 - p5c1^2
p5f2 = (p5c2*x^p5pwr1)^3 - p5c1^3
p5tog = RandInt(0,1)
p5f3 = p5f1^p5tog*p5f2^(1-p5tog)

p5f4 = expand(p5f3)


###### Problem p6
p6c1 = NonZeroInt(-5,5)
p6c2 = RandInt(-5,5)
p6pwr1 = RandInt(1,3)
p6pwr2 = RandInt(2,3)
p6f1 = (p6c1*x^p6pwr1 - p6c2)^p6pwr2

p6f2 = expand(p6f1)


\end{sagesilent}

%\begin{javascript}
%linearFactoring = function(correctAns,f1,f2,f3) {
%    var i;
%    var Truth;
%    var Prod;
%    var Temp;
%    Truth=1;
%    Prod=1;
%    if (isNaN(f1.derivative('x'))){Truth = Truth * 0;}else{Truth = Truth * 1;}
%    if (isNaN(f2.derivative('x'))){Truth = Truth * 0;}else{Truth = Truth * 1;}
%    if (isNaN(f3.derivative('x'))){Truth = Truth * 0;}else{Truth = Truth * 1;}
%    parser.evaluate('f(x) = f1(x)*f2(x)*f3(x));
%    if (f('x').equals(correctAns('x'))){Truth = Truth*1;}else{Truth=Truth*0;}
%    if (Truth === 0){return !1}else{return !0}
%    };
%\end{javascript}

\begin{problem}
Factor the following polynomial:
    \[
        \sage{p1f4} = \answer{\sage{p1f3}}
    \]

\end{problem}

\begin{problem}
Factor the following polynomial:
    \[
        \sage{p2f2} = \answer{\sage{p2f1}}
    \]

\end{problem}



\begin{problem}
Factor the following polynomial:
    \[
        \sage{p3f4} = \answer{\sage{p3f3}}
    \]

\end{problem}

\begin{problem}
Factor the following polynomial:
    \[
        \sage{p4f2} = \answer{\sage{p4f1}}
    \]

\end{problem}



\begin{problem}
Factor the following polynomial:
    \[
        \sage{p5f4} = \answer{\sage{p5f3}}
    \]

\end{problem}

\begin{problem}
Factor the following polynomial:
    \[
        \sage{p6f2} = \answer{\sage{p6f1}}
    \]

\end{problem}



\end{document}