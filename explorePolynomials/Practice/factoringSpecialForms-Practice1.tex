\documentclass{ximeraXloud}
\title{Factor Coefficients Method Practice 1}



\begin{document}
\begin{sagesilent}

######  Define a function to convert a sage number into a saved counter number.

#####Define default Sage variables.
#Default function variables
var('x,y,z,X,Y,Z')
#Default function names
var('f,g,h,dx,dy,dz,dh,df')
#Default Wild cards
w0 = SR.wild(0)

def DispSign(b):
    """ Returns the string of the 'signed' version of `b`, e.g. 3 -> "+3", -3 -> "-3", 0 -> "".
    """
    if b == 0:
        return ""
    elif b > 0:
        return "+" + str(b)
    elif b < 0:
        return str(b)
    else:
        # If we're here, then something has gone wrong.
        raise ValueError

def ISP(b):
    return DispSign(b)

def NoEval(f, c):
    # TODO
    """ Returns a non-evaluted version of the result f(c).
    """
    cStr = str(c)
    # fLatex = latex(f)
    fString = latex(f)
    fStrList = list(fString)
    length = len(fStrList)
    fStrList2 = range(length)
    for i in range(0, length):
        if fStrList[i] == "x":
            fStrList2[i] = "("+cstr+")"
        else:
            fStrList2[i] = fStrList[i]
    f2 = join(fStrList2,"")
    return LatexExpr(f2)

def HyperSimp(f):
    """ Returns the expression `f` without hyperbolic expressions.
    """
    subsDict = {
        sinh(w0) : (exp(w0) - exp(-w0))/2,
        cosh(w0) : (exp(w0) + exp(-w0))/2,
        tanh(w0) : (exp(w0) - exp(-w0))/(exp(w0) + exp(-w0)),
        sech(w0) : 2/(exp(w0) + exp(-w0)),                      # This seems to work, but Nowell said it didn't at one point.
        csch(w0) : 2/(exp(w0) - exp(-w0)),                      # This seems to work, but Nowell said it didn't at one point.
        coth(w0) : (exp(w0) + exp(-w0))/(exp(w0) - exp(-w0)),   # This seems to work, but Nowell said it didn't at one point.
        arcsinh(w0) :       ln( w0 + sqrt((w0)^2 + 1) ),
        arccosh(w0) :       ln( w0 + sqrt((w0)^2 - 1) ),
        arctanh(w0) : 1/2 * ln( (1 + w0) / (1 - w0) ),
        arccsch(w0) :       ln( (1 + sqrt((w0)^2 + 1))/w0 ),
        arcsech(w0) :       ln( (1 + sqrt(1 - (w0)^2))/w0 ),
        arccoth(w0) : 1/2 * ln( (1 + w0) / (w0 - 1) )
    }
    g = f.substitute(subsDict)
    return simplify(g)

def RandInt(a,b):
    """ Returns a random integer in [`a`,`b`]. Note that `a` and `b` should be integers themselves to avoid unexpected behavior.
    """
    return QQ(randint(int(a),int(b)))
    # return choice(range(a,b+1))

def NonZeroInt(b,c, avoid = [0]):
    """ Returns a random integer in [`b`,`c`] which is not in `av`. 
        If `av` is not specified, defaults to a non-zero integer.
    """
    while True:
        a = RandInt(b,c)
        if a not in avoid:
            return a

def RandVector(b, c, avoid=[], rep=1):
    """ Returns essentially a multiset permutation of ([b,c]-av) * rep.
        That is, a vector which contains each integer in [`b`,`c`] which is not in `av` a total of `rep` number of times.
        Example:
        sage: RandVector(1,3, [2], 2)
        [3, 1, 1, 3]
    """
    oneVec = [val for val in range(b,c+1) if val not in avoid]
    vec = oneVec * rep
    shuffle(vec)
    return vec


\end{sagesilent}

\begin{sagesilent}
###### Problem p1
p1c1 = RandInt(-10,10)
p1c2 = NonZeroInt(-5,5)
p1pwr1 = RandInt(1,3)

p1f1 = (p1c2*x^p1pwr1)^2 - p1c1^2
p1f2 = (p1c2*x^p1pwr1)^3 - p1c1^3
p1tog = RandInt(0,1)
p1f3 = p1f1^p1tog*p1f2^(1-p1tog)

p1f4 = expand(p1f3)


###### Problem p2
p2c1 = NonZeroInt(-5,5)
p2c2 = RandInt(-5,5)
p2pwr1 = RandInt(1,3)
p2pwr2 = RandInt(2,3)
p2f1 = (p2c1*x^p2pwr1 - p2c2)^p2pwr2

p2f2 = expand(p2f1)


###### Problem p3
p3c1 = RandInt(-10,10)
p3c2 = NonZeroInt(-5,5)
p3pwr1 = RandInt(1,3)

p3f1 = (p3c2*x^p3pwr1)^2 - p3c1^2
p3f2 = (p3c2*x^p3pwr1)^3 - p3c1^3
p3tog = RandInt(0,1)
p3f3 = p3f1^p3tog*p3f2^(1-p3tog)

p3f4 = expand(p3f3)


###### Problem p4
p4c1 = NonZeroInt(-5,5)
p4c2 = RandInt(-5,5)
p4pwr1 = RandInt(1,3)
p4pwr2 = RandInt(2,3)
p4f1 = (p4c1*x^p4pwr1 - p4c2)^p4pwr2

p4f2 = expand(p4f1)


###### Problem p5
p5c1 = RandInt(-10,10)
p5c2 = NonZeroInt(-5,5)
p5pwr1 = RandInt(1,3)

p5f1 = (p5c2*x^p5pwr1)^2 - p5c1^2
p5f2 = (p5c2*x^p5pwr1)^3 - p5c1^3
p5tog = RandInt(0,1)
p5f3 = p5f1^p5tog*p5f2^(1-p5tog)

p5f4 = expand(p5f3)


###### Problem p6
p6c1 = NonZeroInt(-5,5)
p6c2 = RandInt(-5,5)
p6pwr1 = RandInt(1,3)
p6pwr2 = RandInt(2,3)
p6f1 = (p6c1*x^p6pwr1 - p6c2)^p6pwr2

p6f2 = expand(p6f1)


\end{sagesilent}

%\begin{javascript}
%linearFactoring = function(correctAns,f1,f2,f3) {
%    var i;
%    var Truth;
%    var Prod;
%    var Temp;
%    Truth=1;
%    Prod=1;
%    if (isNaN(f1.derivative('x'))){Truth = Truth * 0;}else{Truth = Truth * 1;}
%    if (isNaN(f2.derivative('x'))){Truth = Truth * 0;}else{Truth = Truth * 1;}
%    if (isNaN(f3.derivative('x'))){Truth = Truth * 0;}else{Truth = Truth * 1;}
%    parser.evaluate('f(x) = f1(x)*f2(x)*f3(x));
%    if (f('x').equals(correctAns('x'))){Truth = Truth*1;}else{Truth=Truth*0;}
%    if (Truth === 0){return !1}else{return !0}
%    };
%\end{javascript}

\begin{problem}
Factor the following polynomial:
    \[
        \sage{p1f4} = \answer{\sage{p1f3}}
    \]

\end{problem}

\begin{problem}
Factor the following polynomial:
    \[
        \sage{p2f2} = \answer{\sage{p2f1}}
    \]

\end{problem}



\begin{problem}
Factor the following polynomial:
    \[
        \sage{p3f4} = \answer{\sage{p3f3}}
    \]

\end{problem}

\begin{problem}
Factor the following polynomial:
    \[
        \sage{p4f2} = \answer{\sage{p4f1}}
    \]

\end{problem}



\begin{problem}
Factor the following polynomial:
    \[
        \sage{p5f4} = \answer{\sage{p5f3}}
    \]

\end{problem}

\begin{problem}
Factor the following polynomial:
    \[
        \sage{p6f2} = \answer{\sage{p6f1}}
    \]

\end{problem}



\end{document}