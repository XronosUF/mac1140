\documentclass{ximeraXloud}
\input{../preamble}
\title{Geometric Vs Analytic Viewpoints}
\begin{document}
\begin{abstract}
    We discuss what Geometric and Analytic views of mathematics are and why one might be better for `learning' whereas the other may be better for `using'
\end{abstract}
\maketitle

\subsection*{The Hope...}

    There are two primary views one can take when learning almost any math; the geometric perspective and the analytic perspective. This is true regardless of mathematical level and, in fact, you have already experienced this many times throughout your math education. Unfortunately, the virtues (and flaws) of the two perspectives are often hidden from students as some sort of ``deep pedagogical viewpoint"%
    \footnote{%
        Pedagogical viewpoint is the teaching philosophy and perspective of how something is being taught. For example, some teachers may believe it is better to lecture about material, to ensure that they are able to cover large portions of material quickly and correctly. Others believe it is better to have students explore concepts and learn/discover them on their own to help with retention and understanding. These would be considered two different ``pedagogical perspectives".%
        }
    that is considered to be important for the teacher, but not for the student. In reality I have found that, the more transparent a teacher is in their methodology, the more willing the students are to embrace it, and the more effective it becomes.

\subsection*{The False Hierarchy: Analytic is (not!) ``better" than Geometric.}

    Mathematicians will often talk about ``seeing" the math. Contrary to popular belief, this does not mean we just visualize a bunch of complex equations and bizarre symbols floating around.
    
    In fact, what mathematicians mean by this is more about seeing a picture that represents what is happening. Kind of like when you were young and your teachers told you to use your fingers to add numbers, or manipulate groups of blocks to represent multiplication and division.
    
    As you continued your math education however, there was a switch to the ``analytic perspective" (which we will discuss more as we continue) which gives the false impression that thinking analytically (as oppose to geometrically) is somehow innately superior and that thinking geometrically is somehow childish or unhelpful.
    
    In reality, it is more often the case that the geometric and analytic perspectives complement each other. Geometric perspective is often used to gain an intuitive understanding of what is happening, and then the analytic perspective is often used to make that intuition precise.



\end{document}



