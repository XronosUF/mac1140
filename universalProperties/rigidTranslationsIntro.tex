\documentclass{ximeraXloud}

\title{Intro: Rigid Translations}
\begin{document}
\begin{abstract}
    An introduction to the ideas of rigid translations.
\end{abstract}
\maketitle

As we have seen, we can learn things utilizing both analytic and geometric perspectives. Often it is the case that the geometric perspective is more useful for initially learning a new concept, and then the analytic viewpoint is more useful to actually \textit{do} anything with the new concept. For better or worse, rigid translations are no exception to this dynamic. The analytic view of translations is somewhat unenlightening when trying to initially learn the idea.%
\footnote{%
    This is a dramatic understatement.%
    }
In truth, the very phrase `rigid translation' should be a clue that a better way to see what is happening during a rigid translation, is to... well... see it. Nonetheless, the analytic view tends to be more useful once you already understand what is happening. when you want to precisely state or describe a rigid translation quickly and concisely.

For these reasons we will start out our work on rigid translations with some geometric descriptions and depictions and then we will delve into the analytic mechanics in how to write and/or manipulate functions with rigid translations using algebra.

\begin{problem}
    We will start with a geometric view of rigid translations because
    \begin{multipleChoice}
        \choice{We had to start with something, might as well be that.}
        \choice{Geometry is a lot more fun and interesting than algebra.}
        \choice[correct]{It is hard to manipulate functions via rigid translations without understanding what they are, and it is hard to understand what they are without literally seeing what is happening.}
        \choice{We aren't starting with geometric view, we're starting with the analytic view.}
    \end{multipleChoice}
\end{problem}
\end{document}