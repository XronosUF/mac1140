\documentclass{ximeraXloud}

\title{Rigid Translations}
\begin{document}
\begin{abstract}
    This section describes what makes a Rigid Translation. We aim to learn how to use the function notation to perform (and read) a rigid translation quickly and easily.
\end{abstract}
\maketitle

There are two primary views one can take when learning almost any math; the geometric perspective and the analytic perspective.

The analytic view of translations is somewhat unenlightening when trying to initially learn about rigid translations.%
\footnote{%
    This is a dramatic understatement.%
    }
In truth, the very phrase `rigid translation' should be a clue that a better way to see what is happening during a rigid translation, is to... well... see it. Nonetheless, the analytic view tends to be more useful once you already understand what is happening.

\subsection*{The geometric view}
    The result of a rigid translation is incredibly apparent when looking at the graph (before and after) of the function. The result is also more easily stated (at least in terms of what is happening to the graph) in geometric terms. In essence, a `rigid translation' changes the graph of the function in a very specific way; by ``picking up the function and moving it's location on the $x$-$y$ plane". The key observation%
    \footnote{%
        but easily overlooked due to how `obvious' it is%
        }
    is that the \textbf{only} thing that changes about the graph is its position on the $x$-$y$ plane. Take, for example, the following graph:
    
    \begin{minipage}{\textwidth}
        \begin{center}
            \begin{tikzpicture}
                \begin{axis}[
                            axis x line=middle,
                            axis y line=middle,
                            minor tick num=5,
                            x label style={at={(axis description cs:1,0.5)},anchor=south},
                            y label style={at={(axis description cs:0.5,1)},anchor=west},
                            xlabel={$x$},
                            ylabel={$y$},
                            xmin=-8,
                            xmax=8,
                            ymin=-17,
                            ymax=17
                            ]
                \addplot[<->,domain=-3.1:3.1, samples=300]{1/2*(x-3)*(x-2)*(x+1)*(x+3)};
                \end{axis}
            \end{tikzpicture}
        \end{center}
    \end{minipage}
    
    We wish to apply a rigid translation to this graph; specifically the rigid translation that moves the graph to the right by $4$ and down by $5$. That would change the above graph to the following:
    
    \begin{minipage}{\textwidth}
        \begin{center}
            \begin{tikzpicture}
                \begin{axis}[
                    axis x line=middle,
                    axis y line=middle,
                    minor tick num=5,
                    x label style={at={(axis description cs:1,0.5)},anchor=south},
                    y label style={at={(axis description cs:0.5,1)},anchor=west},
                    xlabel={$x$},
                    ylabel={$y$},
                    xmin=-8,
                    xmax=8,
                    ymin=-17,
                    ymax=17
                    ]
                \addplot[<->,domain=0.9:7.1, samples=300]{1/2*(x-3-4)*(x-2-4)*(x+1-4)*(x+3-4) - 5};
                \end{axis}
            \end{tikzpicture}
        \end{center}
    \end{minipage}
    
    As we can see, we appear to have the same curve (the same `graph') but it moved to a different location. This is how we know the transformation we applied is a \textit{rigid} transformation.%
    \footnote{%
        In fact, the name `rigid translation' is from the fact that the shape stays `rigid', like it's made of steel, while we pick it up and move (translate) it from one location to another.%
        }

    \subsubsection*{Ok, I see that, but didn't I read a whole topic a bit ago on how we don't like writing this many words?}
        Indeed you did, and don't fret, we have notation for this too! I know this is just all kinds of exciting for you to hear, and you are surely running away to go tell your friends. You'll be back right? ... right? ... hello?
        
        Really though, this is where the functional notation%
        \footnote{%
            See the functional notation appendix entry for a review if needed%
            }
        aka the `analytic view' helps immensely.


\subsection*{The analytic view}
    A rigid translation is one in which the relation that represents the function is essentially `changed universally' in some sense. That is to say that for any particular $x$ and $y$ with $f(x) = y$, whatever is being done to that pair of related points, the same thing is being done to \textit{every single pair of related points}. Moreover, this change must always be a matter of adding or subtracting (in a particular way which we will discuss below).%
    \footnote{%
        This is certainly not always true in general, but it will be considered true for this course, and throughout the calculus sequence. For those that might go on to do more abstract/theoretical mathematical work, like mathematics or theoretical physics, you will eventually get a more generalized definition of `rigid translation'.%
        }
    
    For example, let's say we have some function $f(x)$, and we know two pairs of related points, $(3,4)$ (ie $f(3) = 4$) and $(8,1)$ (ie $f(8)=1$). Now we apply some kind of rigid translation. We are told that the domain point $3$ is now (after the rigid translation) changed to the domain point $7$. Moreover, the point of the codomain, $f(3)$ (which was originally $4$) is now changed, by this rigid translation, to $-1$.
    
    Knowing that the rigid translation can only use addition and subtraction, we can determine what this translation is based on how it moved the point we were given. The domain value (the $x$) went from $3$ to $7$, so the $x$-value was increased by 4. The codomain value (aka the $y$ or $f(3)$) went from $4$ to $-1$, thus it was decreased by $5$. So, our rigid transformation is then ``add 4 to the $x$-value and subtract 5 from the $y$-value". 
    
    \begin{exploration}
        The rigid translation would have to do the \textit{same exact thing} to \textit{every} point, specifically it would have to \textbf{add} or \textbf{subtract} \textit{exactly the same} values to \textit{every} related $(x,y)$ pair. Knowing this, we can conclude that the previous point of $(8,1)$ would be sent to $(8+\answer{4}, 1+\answer{-5}) = (\answer{12},\answer{-4})$.
    \end{exploration}
    %%%%%%%
    
    Notice how, in our example, we talked about the translations that applied to the domain element, and translations that applied to the codomain element (that is to say, we separated the translation into the addition/subtraction parts that effected the $x$ and the addition/subtraction parts that effected the $y$). This separation is made even more clear when we write the translation in functional notation. Let's consider the $f(x)$ from the geometric view above, whose graph is;

    \begin{minipage}{\textwidth}
        \begin{center}
            \begin{tikzpicture}
                \begin{axis}[
                    axis x line=middle,
                    axis y line=middle,
                    minor tick num=5,
                    x label style={at={(axis description cs:1,0.5)},anchor=south},
                    y label style={at={(axis description cs:0.55,1)},anchor=west},
                    xlabel={$x$},
                    ylabel={$y$},
                    xmin=-8,
                    xmax=8,
                    ymin=-17,
                    ymax=17
                    ]
                \addplot[<->,domain=-3.1:3.1, samples=300]{1/2*(x-3)*(x-2)*(x+1)*(x+3)};
                \end{axis}
            \end{tikzpicture}
        \end{center}
    \end{minipage}
    
    Now, for our example of adding 4 to the $x$ and subtracting 5 from the $f(x)$ we can rewrite the new translation by first giving it a name, say $g$, and writing it as a modification of the original $f$. 
    
    Intuitively we might try to write our translation as follows; $g(x) = f(x+4) - 5$. After all, we wanted to add $4$ to the $x$ and subtract $5$ from the $y$, so this seems pretty reasonable. Let's see what the graph $f(x+4) - 5$ looks like.
    
    \begin{minipage}{\textwidth}
        \begin{center}
            \begin{tikzpicture}
                \begin{axis}[
                    axis x line=middle,
                    axis y line=middle,
                    minor tick num=5,
                    x label style={at={(axis description cs:1,0.5)},anchor=south},
                    y label style={at={(axis description cs:0.5,1)},anchor=west},
                    xlabel={$x$},
                    ylabel={$y$},
                    xmin=-8,
                    xmax=8,
                    ymin=-17,
                    ymax=17
                    ]
                \addplot[<->,domain=-7.1:-0.9, samples=300]{1/2*(x-3+4)*(x-2+4)*(x+1+4)*(x+3+4) - 5};
                \end{axis}
            \end{tikzpicture}
        \end{center}
    \end{minipage}
    
    A near miss! It appears that we got the vertical movement correct but it shifted the wrong direction horizontally, it shifted left instead of right! This is because of a deceptively important (and generally useful) technique called `change of variable' or `change of coordinate' that is being used in the background here. The specifics on this are outside the scope of this course, but are included in the appendix section. The actual change of variable technique isn't (explicitly) taught until calc 2.%
    \footnote{%
        sometimes in calc 1 depending on the instructor%
        }
    The short version in this specific context is that `anything that effects the $x$ is the opposite of what you'd expect'. Thus, since you would expect that we want to add 4 to $x$ (to move it to the right) what we \textit{actually} want to do (in order to move it to the right) is \textit{subtract} 4. Thus the $g(x)$ we actually want to use is $g(x) = f(x-4)-5$ whose graph is;
    
    \begin{minipage}{\textwidth}
        \begin{center}
            \begin{tikzpicture}
                \begin{axis}[
                    axis x line=middle,
                    axis y line=middle,
                    minor tick num=5,
                    x label style={at={(axis description cs:1,0.5)},anchor=south},
                    y label style={at={(axis description cs:0.5,1)},anchor=west},
                    xlabel={$x$},
                    ylabel={$y$},
                    xmin=-8,
                    xmax=8,
                    ymin=-17,
                    ymax=17
                    ]
                \addplot[<->,domain=0.9:7.1, samples=300]{1/2*(x-3-4)*(x-2-4)*(x+1-4)*(x+3-4) - 5};
                \end{axis}
            \end{tikzpicture}
        \end{center}
    \end{minipage}
    
    
    Writing this rigid translation using the $f(x-4)-5$ type notation is the `functional representation'. In general, a rigid translation of a function $f$ can be written by adding (or subtracting) values from the functional argument (the part inside the parentheses that follow $f$), or by adding or subtracting values to the overall result after applying $f$. So, if you want to add some number $c$ to the $x$ values and another number $k$ to the codomain value, you can write it as;
    \[
        f(x - c) + k
    \]
    Notice that the $c$ inside has a \textit{subtraction} sign in front of it. If you remember it in this form then you can write the $c$ as you would expect (since the $c$ is being subtracted which will take care of the 'opposite of what you'd except' part). Thus if you want to move the graph to the right 4 units, you can use $c=4$ (note that $c$ here is positive) but remember the subtraction \textit{in the form} so you'd write $f(x - c) = f(x - (4))$. In comparison if you wanted to move the graph to the left by 4 you would do the same process. Since it's \textit{to the left} you would think $c = -4$, so plugging that into the \textit{form above} we have $g(x) = f(x - c) = f(x - (-4) ) = f(x+4)$ which is correct.
    
    Play with the following graph to see what happens as you change the parameters. Be careful to observe the effect of the sign of the parameters in the example. Hopefully playing with this interactive graph will help make clear how changing the values (rigidly) translate the graph!
    
    \desmos{zf9bdgelpw}{}{500}
    

\end{document}