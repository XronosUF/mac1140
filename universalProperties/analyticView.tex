\documentclass{ximeraXloud}
\usepackage{longdivision}
\usepackage{polynom}
\usepackage{float}% Use `H' as the figure optional argument to force it's vertical placement to conform to source.
%\usepackage{caption}% Allows us to describe the figures without having "figure 1:" in it. :: Apparently Caption isn't supported.
%    \captionsetup{labelformat=empty}% Actually does the figure configuration stated above.
\usetikzlibrary{arrows.meta,arrows}% Allow nicer arrow heads for tikz.
\usepackage{gensymb, pgfplots}
\usepackage{tabularx}
\usepackage{arydshln}



\graphicspath{
  {./}
  {./explorePolynomials/}
  {./exploreRadicals/}
  {./graphing/}
}

%% Default style for tikZ
\pgfplotsset{my style/.append style={axis x line=middle, axis y line=
middle, xlabel={$x$}, ylabel={$y$}, axis equal }}


%% Because log being natural log is too hard for people.
\let\logOld\log% Keep the old \log definition, just in case we need it.
\renewcommand{\log}{\ln}


%%% Changes in polynom to show the zero coefficient terms
\makeatletter
\def\pld@CF@loop#1+{%
    \ifx\relax#1\else
        \begingroup
          \pld@AccuSetX11%
          \def\pld@frac{{}{}}\let\pld@symbols\@empty\let\pld@vars\@empty
          \pld@false
          #1%
          \let\pld@temp\@empty
          \pld@AccuIfOne{}{\pld@AccuGet\pld@temp
                            \edef\pld@temp{\noexpand\pld@R\pld@temp}}%
           \pld@if \pld@Extend\pld@temp{\expandafter\pld@F\pld@frac}\fi
           \expandafter\pld@CF@loop@\pld@symbols\relax\@empty
           \expandafter\pld@CF@loop@\pld@vars\relax\@empty
           \ifx\@empty\pld@temp
               \def\pld@temp{\pld@R11}%
           \fi
          \global\let\@gtempa\pld@temp
        \endgroup
        \ifx\@empty\@gtempa\else
            \pld@ExtendPoly\pld@tempoly\@gtempa
        \fi
        \expandafter\pld@CF@loop
    \fi}
\def\pld@CMAddToTempoly{%
    \pld@AccuGet\pld@temp\edef\pld@temp{\noexpand\pld@R\pld@temp}%
    \pld@CondenseMonomials\pld@false\pld@symbols
    \ifx\pld@symbols\@empty \else
        \pld@ExtendPoly\pld@temp\pld@symbols
    \fi
    \ifx\pld@temp\@empty \else
        \pld@if
            \expandafter\pld@IfSum\expandafter{\pld@temp}%
                {\expandafter\def\expandafter\pld@temp\expandafter
                    {\expandafter\pld@F\expandafter{\pld@temp}{}}}%
                {}%
        \fi
        \pld@ExtendPoly\pld@tempoly\pld@temp
        \pld@Extend\pld@tempoly{\pld@monom}%
    \fi}
\makeatother




%%%%% Code for making prime factor trees for numbers, taken from user Qrrbrbirlbel at: https://tex.stackexchange.com/questions/131689/how-to-automatically-draw-tree-diagram-of-prime-factorization-with-latex

\usepackage{forest,mathtools,siunitx}
\makeatletter
\def\ifNum#1{\ifnum#1\relax
  \expandafter\pgfutil@firstoftwo\else
  \expandafter\pgfutil@secondoftwo\fi}
\forestset{
  num content/.style={
    delay={
      content/.expanded={\noexpand\num{\forestoption{content}}}}},
  pt@prime/.style={draw, circle},
  pt@start/.style={},
  pt@normal/.style={},
  start primeTree/.style={%
    /utils/exec=%
      % \pt@start holds the current minimum factor, we'll start with 2
      \def\pt@start{2}%
      % \pt@result will hold the to-be-typeset factorization, we'll start with
      % \pgfutil@gobble since we don't want a initial \times
      \let\pt@result\pgfutil@gobble
      % \pt@start@cnt holds the number of ^factors for the current factor
      \def\pt@start@cnt{0}%
      % \pt@lStart will later hold "l"ast factor used
      \let\pt@lStart\pgfutil@empty,
    alias=pt-start,
    pt@start/.try,
    delay={content/.expanded={$\noexpand\num{\forestove{content}}
                            \noexpand\mathrlap{{}= \noexpand\pt@result}$}},
    primeTree},
  primeTree/.code=%
    % take the content of the node and save it in the count
    \c@pgf@counta\forestove{content}\relax
    % if it's 2 we're already finished with the factorization
    \ifNum{\c@pgf@counta=2}{%
      % add the factor
      \pt@addfactor{2}%
      % finalize the factorization of the result
      \pt@addfactor{}%
      % and set the style to the prime style
      \forestset{pt@prime/.try}%
    }{%
      % this simply calculates content/2 and saves it in \pt@end
      % this is later used for an early break of the recursion since no factor
      % can be greater then content/2 (for integers of course)
      \edef\pt@content{\the\c@pgf@counta}%
      \divide\c@pgf@counta2\relax
      \advance\c@pgf@counta1\relax % to be on the safe side
      \edef\pt@end{\the\c@pgf@counta}%
      \pt@do}}

%%% our main "function"
\def\pt@do{%
  % let's test if the current factor is already greather then the max factor
  \ifNum{\pt@end<\pt@start}{%
    % great, we're finished, the same as above
    \expandafter\pt@addfactor\expandafter{\pt@content}%
    \pt@addfactor{}%
    \def\pt@next{\forestset{pt@prime/.try}}%
  }{%
    % this calculates int(content/factor)*factor
    % if factor is a factor of content (without remainder), the result will
    % equal content. The int(content/factor) is saved in \pgf@temp.
    \c@pgf@counta\pt@content\relax
    \divide\c@pgf@counta\pt@start\relax
    \edef\pgf@temp{\the\c@pgf@counta}%
    \multiply\c@pgf@counta\pt@start\relax
    \ifNum{\the\c@pgf@counta=\pt@content}{%
      % yeah, we found a factor, add it to the result and ...
      \expandafter\pt@addfactor\expandafter{\pt@start}%
      % ... add the factor as the first child with style pt@prime
      % and the result of int(content/factor) as another child.
      \edef\pt@next{\noexpand\forestset{%
        append={[\pt@start, pt@prime/.try]},
        append={[\pgf@temp, pt@normal/.try]},
        % forest is complex, this makes sure that for the second child, the
        % primeTree style is not executed too early (there must be a better way).
        delay={
          for descendants={
            delay={if n'=1{primeTree, num content}{}}}}}}%
    }{%
      % Alright this is not a factor, let's get the next factor
      \ifNum{\pt@start=2}{%
        % if the previous factor was 2, the next one will be 3
        \def\pt@start{3}%
      }{%
        % hmm, the previos factor was not 2,
        % let's add 2, maybe we'll hit the next prime number
        % and maybe a factor
        \c@pgf@counta\pt@start
        \advance\c@pgf@counta2\relax
        \edef\pt@start{\the\c@pgf@counta}%
      }%
      % let's do that again
      \let\pt@next\pt@do
    }%
  }%
  \pt@next
}

%%% this builds the \pt@result macro with the factors
\def\pt@addfactor#1{%
  \def\pgf@tempa{#1}%
  % is it the same factor as the previous one
  \ifx\pgf@tempa\pt@lStart
    % add 1 to the counter
    \c@pgf@counta\pt@start@cnt\relax
    \advance\c@pgf@counta1\relax
    \edef\pt@start@cnt{\the\c@pgf@counta}%
  \else
    % a new factor! Add the previous one to the product of factors
    \ifx\pt@lStart\pgfutil@empty\else
      % as long as there actually is one, the \ifnum makes sure we do not add ^1
      \edef\pgf@tempa{\noexpand\num{\pt@lStart}\ifnum\pt@start@cnt>1 
                                           ^{\noexpand\num{\pt@start@cnt}}\fi}%
      \expandafter\pt@addfactor@\expandafter{\pgf@tempa}%
    \fi
    % setup the macros for the next round
    \def\pt@lStart{#1}% <- current (new) factor
    \def\pt@start@cnt{1}% <- first time
  \fi
}
%%% This simply appends "\times #1" to \pt@result, with etoolbox this would be
%%% \appto\pt@result{\times#1}
\def\pt@addfactor@#1{%
  \expandafter\def\expandafter\pt@result\expandafter{\pt@result \times #1}}

%%% Our main macro:
%%% #1 = possible optional argument for forest (can be tikz too)
%%% #2 = the number to factorize
\newcommand*{\PrimeTree}[2][]{%
  \begin{forest}%
    % as the result is set via \mathrlap it doesn't update the bounding box
    % let's fix this:
    tikz={execute at end scope={\pgfmathparse{width("${}=\pt@result$")}%
                         \path ([xshift=\pgfmathresult pt]pt-start.east);}},
    % other optional arguments
    #1
    % And go!
    [#2, start primeTree]
  \end{forest}}
\makeatother


\providecommand\tabitem{\makebox[1em][r]{\textbullet~}}
\providecommand{\letterPlus}{\makebox[0pt][l]{$+$}}
\providecommand{\letterMinus}{\makebox[0pt][l]{$-$}}



\title{Analytic Viewpoint}

\begin{document}
\begin{abstract}
    We discuss the analytic view of mathematics, and specifically when and where it is most useful or appropriate to use.
\end{abstract}
\maketitle

\subsection*{So, What is an ``Analytic Perspective"?}
    Analytic perspective is the ``down and gritty number crunching" that is often required of employees to support their conclusions. The bold sweeping declaration of the ability to do something impressive is often initially supported or argued with geometric information (e.g. ``as we can see by these trend lines" or ``this chart shows the income invested versus return for the past several years, and we can see that we have already reached our break-even point..."), but the support is almost always analytic. The graphs, charts, and snapshots of progress that are used to convince people in a board room aren't materialized out of thin air, and they are (usually) not made up information out of nowhere. Typically they are the product of hours and hours of effort determining what things to include, what things to exclude, and how the data supports (or doesn't) the conclusion you are drawing. All of these ``detail" processes tend to rely on analytic information.
    
    The typical mathematical demonstration of this is algebra. If you wanted to calculate something like a ``break even point" for a company, you wouldn't draw a graph and guess. Instead you would set the profit equal to zero (or equivalently set the cost equal to the revenue) and solve for time in the resulting equation; utilizing techniques from algebra. This will give you a very precise answer and perhaps more importantly, it is the same answer anyone else using the same data would yield (assuming no mistakes were made). Thus it is an \textit{objective} result, once you have established the initial framework of the problem.%
    \footnote{%
        This is a key point; often the differences between two ``analyses" of information is found, not in the data itself, but in what was considered as `important' data, and how one should incorporate it.%
        }
    
    Most of our exploration of functions in the coming sections of this course will center around trying to determine/find analytic solutions to geometric observations. Things like ``what are the zeros of this function" originate from geometric reasoning based upon the context of the question, but require analytic methods to arrive at a precise conclusion.
    
    For example;
    \begin{example}
        Below is a graphic representation of a population study on a certain breed of monkey in the wild from 1962 to 1982\\
        \[
            \graph[xmin=0,xmax=20,ymin=0,ymax=60]{f(t)=50/(1+50e^{-t/2})}
        \]
        
        We wish to know the population of monkeys in 1962, which would occur at $t = \answer{0}$.
        \begin{explanation}
            Since we know the context of the problem, we know that the population in 1962 was the ``starting value" which means that we want to look at the $t = 0$ case. But this doesn't mean there is anything magical about $t=0$, it's just that, due to the context of the problem, we knew that was the value of $t$ that we needed. This is an example of \wordChoice{\choice{Analytic Reasoning}\choice[correct]{Geometric Reasoning}\choice{why we hate math}\choice{Torture. Just torture}}.
        \end{explanation}
    
        \begin{problem}
            If we wished to know the population in, say, 1971, what would we need to do?
            
            \begin{multipleChoice}
                \choice{Look at $t=9$ and approximate the value of the graph.}
                \choice{Look at the sky and curse the Gods.}
                \choice[correct]{Plug in $t=9$ into the function that generated the graph to get a value.}
                \choice{Ask someone else.}
            \end{multipleChoice}

            \begin{explanation}
                Now that we want \textit{precise} data, we need to use a more \textit{precise} method. Thus we will need to use something analytic, which involves going beyond the graph. In this example, since the graph is generated by an underlying equation, we would need to plug in the correct value of $t$ (ie $9$) to get the \textit{precise} value we are looking for. This is an example of \wordChoice{\choice{Analytic Reasoning}\choice[correct]{Geometric Reasoning}\choice{why we hate math}\choice{Torture. Just torture}}.
            \end{explanation}            
            
        \end{problem}
        
    \end{example}
    
    


\end{document}



