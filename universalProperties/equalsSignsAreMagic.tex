\documentclass{ximeraXloud}

\title{Equals Signs are Magic!}
\begin{document}
\begin{abstract}
    This section describes the very special and often overlooked virtue of the `equals sign'. It also includes when and why you should ``set something equal to zero" which is often overused or used incorrectly.
\end{abstract}
\maketitle

Perhaps the most common and useful techniques in \textit{all} levels of mathematics can be boiled down to ``add zero or multiply by 1... \textit{cleverly}". This is because zero and one are incredibly special numbers in mathematics.%
\footnote{%
    In fact, they are so special that studying why they were seemingly unique led to the inception of vast areas of study in mathematics, including `abstract algebra' and `number theory', both of which are considered core disciplines of mathematical study for any mathematician.%
    }
Here we will give a very brief explanation of what we mean by being `clever' and a few things about what makes these numbers special.


\subsection*{An equals sign is magic... kind of like wishes... and curses.}

    Two things being equal in mathematics is an \textit{incredibly strong statement}. Again it's important to remember that math is all about \textit{precision}. Thus when we say something like ``$x = 4$", this is meant to be a \textit{precise} statement. This might seem trivial or silly to point out, but the human brain is impressive at building \textit{equivalence}, and in mathematics \textbf{there is a very important difference between equality and equivalence}.
    
    An equivalence is where you can convert one thing to another through some outside mechanism. For example, we could say that the letters of the alphabet and the numbers one through twenty six are \textit{equivalent} by assigning each letter the corresponding number in sequence (eg A = 1, B = 2, C = 3, ..., Z = 26). But we would \textit{not }say the letters of the alphabet and the numbers one through twenty six are \textit{equal}.
    
    In mathematics equality means they are absolutely, and in every way, the exact same thing. This is sort of like the difference between `congruence' and `similarity' in geometry. Things can be similar (equivalent) because they are somehow `basically the same thing', but being congruent means that two shapes are \textit{exactly} the same, which is what equality requires. This means that \textbf{in general it is very difficult to claim two things are equal}, but if you \textit{already} know that two things are equal, it allows you to do a lot of useful things with that knowledge.%
    \footnote{%
        Generally speaking it's very difficult to show things are equal, but math teachers often skip that part by merely providing the equality to the students. Although understandable, this often hides the incredibly rigorous requirements to meet to show that two things are equal, and thereby hides how powerful the equality sign really is from the students.%
        }
    In this class we will rarely deduce two things are equal (but it will happen), but we will often have circumstances where we know things are equal, or we \textit{define} something to be equal to something else (this is how variable substitution works).
    
    The litany of things we can do with an equality sign will be developed as we progress through the course. There is one common thing to mention here however as it does in fact apply to all functions, which is that often you have had to `set equations equal to zero' in the past.%
    \footnote{%
        Strictly speaking there has been nothing wrong with doing this, since the times you were taught to do this were appropriate times to do it. However, my experience with students is that this kind of blunt hammering of students with always setting various problems to zero is that they begin to view that as a sort of `default procedure' based on the function, not based on what you are looking for. As a result, they tend to do this even/especially in circumstances where it makes absolutely no sense to do so. This is what I am trying to address here.%
        }
    This is a common thing to consider in mathematics (again, at all levels) and we will do the same thing at many times throughout topics 9+. However, it's important to realize that by introducing an equal sign yourself you are making an \textit{incredibly powerful statement}. This means that you should \textit{never set something equal to zero without being able to explain why that specifically is useful}. More specifically `because that's how I get the solution' is \textit{not} a valid response when asked why you are setting something equal to zero. You should be able to explain why the equation or expression being zero represents the solution.
    
    \begin{explanation}[Object flying through the air]%
        Let's say we throw a ball through the air. It's vertical position (how high off the ground it is) at any time $t$ can be modeled by the following equation;
        \[
            h(t) = -4.9t^2 + 30t + 1
        \]
        How would we determine when the ball hits the ground?
    
        When we think about this question we might consider what it means to `hit the ground'. Specifically, what would the height of the ball be when it hit the ground? Since $h(t)$ is telling us how high off the ground the ball is, then when it hits the ground $h(t)$ would be zero. Thus we would want to calculate the answer to the equation;
        \[
            0 = h(t) = -4.9t^2 + 30t + 1
        \]
        Thus we appear to be `setting $h(t)$ to zero', but this isn't because it's a quadratic and we just `always do this with quadratics'. The zero didn't come from the nature of the function we are looking at, it came from the context of the model; we knew that the ball hits the ground when the height is zero, so we knew the height of the ball at the time of interest should be zero and thus `plugged it in'. Meaning that the value \textit{happened} to be zero, and that's why we `set it equal to zero', not because of the equation itself.
    \end{explanation}

\end{document}