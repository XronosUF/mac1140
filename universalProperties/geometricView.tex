\documentclass{ximeraXloud}
\input{../preamble}
\title{Geometric Viewpoint}
%\outcome{To do better.}
\begin{document}
\begin{abstract}
    We discuss what Geometric and Analytic views of mathematics are and why one might be better for `learning' whereas the other may be better for `using'
\end{abstract}
\maketitle

\subsection*{So, What is a ``Geometric Perspective"?}
    
    The previous example; counting using your fingers, is actually a great example of geometric reasoning or reasoning using a geometric perspective. Almost surely, as you continued in your math education, the next time this kind of reasoning occurred is graphing. 
    
    Most students loathe graphing%
    \footnote{%
        I was one of those students, I feel your pain...%
        }
    because it feels like a lot of work and it isn't clear why it's useful. Combine this with the classic approach of ``if you do it four thousand times, it will magically click and make perfect sense!" approach of most math classrooms%
    \footnote{%
        In fairness, practice is actually incredibly important in mathematics, and without actually doing something you will almost never really understand it in math. However, this approach has it's downsides as well, and graphing is just about as close to the perfect setting as you can get when it comes to murdering creativity and understanding with practice.%
        }
    and it can swiftly become an exercise in testing the endurance of students to academic torture.
    
    Luckily I'm not a sadist%
    \footnote{%
        Despite what some students might claim...%
        }
    and I have no interest in trying to make you graph things endlessly... at least without cause.%
    \footnote{%
        Queue evil villain laugh%
        }
    The goal with graphing and learning about graphing manipulations in the next several sections, is to build intuition on how certain actions effect the underlying relation of a given function. This is a remarkably vague statement, but graphing is used in a lot of contexts so we will be more specific as we introduce specific tools and usages of graphing. In general however, one should think of a geometric perspective of an idea as the ``big picture", an idea of how and why something is working, but lacking in the (important but fine) details to actually \textit{do} anything using the idea.
    
\subsection*{So... we can't do anything with graphing. Why are we learning this again?}
    
    Ok there killer, back up. I didn't say you can't do \textit{anything} with graphing. It is more accurate to say that graphing lacks the detail necessary to do any of the fine manipulations and detailed problem solving that you would actually need to do in practice with any given problem. For example, knowing that your company's profits are ``trending upward" is geometric, but your boss would probably like to know things like ``by how much exactly" and ``at what point will we hit a large enough profit to expand to another location?" These are the kinds of questions that graphing \textit{can't} typically answer.%
    \footnote{%
        At least, it can't answer it in precise terms without more information. As mentioned earlier, sometimes graphing can answer precise questions, but they require \textit{very} specific circumstances that rarely occur in reality.%
        }
    
    In a business setting, geometric information usually takes the form of ``presentation information", that is to say, the information that you would present to others to show things like a proof of concept, anticipated growth/profits, or data trends. What it isn't, is the endless hours you put in ``crunching numbers" to be able to boil everything down to the nice pages in the report with pretty pictures and ``at a glance" information that you hand out in the meetings. That would be the ``analytic perspective" which we will talk about next, and which usually comprises the bulk of that report you hand out that people skim over.

\begin{problem}
    Which of the following is the \textit{best} description of ``Geometric Perspective"?
    \begin{multipleChoice}
        \choice{Visualizing data points, rather than recording them in a spreadsheet.}
        \choice[correct]{To get a `wide lense view' of data; to have a visual summation of large-scale trends and important features of data.}
        \choice{To find specific data points that are of interest.}
        \choice{To demonstrate ideal data trends and features, as oppose to the actual data trends and features.}
    \end{multipleChoice}
\end{problem}

\end{document}



