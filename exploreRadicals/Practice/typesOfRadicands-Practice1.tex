\documentclass{ximeraXloud}
\title{Factor Coefficients Method Practice 1}



\begin{document}
\input{Useful-Sage-Macros}

\begin{sagesilent}
funcvec = [x^2, x, e^x, log(x)]

###### Problem p1
p1c1 = NonZeroInt(-10,10)
p1c2 = NonZeroInt(-10,10,[0,p1c1])
p1pick1 = RandInt(0,3)
p1pick2 = NonZeroInt(0,3,[p1pick1])

p1rad = p1c1*funcvec[p1pick1]+p1c2*funcvec[p1pick2]


###### Problem p2
p2c1 = NonZeroInt(-10,10)
p2c2 = NonZeroInt(-10,10,[0,p2c1])
p2pick1 = RandInt(0,3)
p2pick2 = NonZeroInt(0,3,[p2pick1])

p2rad = p2c1*funcvec[p2pick1]*p2c2*funcvec[p2pick2]


###### Problem p3
p3c1 = NonZeroInt(-10,10)
p3c2 = NonZeroInt(-10,10,[0,p3c1])
p3pick1 = RandInt(0,3)
p3pick2 = NonZeroInt(0,3,[p3pick1])

p3rad = p3c1*funcvec[p3pick1]+p3c2*funcvec[p3pick2]


###### Problem p4
p4c1 = NonZeroInt(-10,10)
p4c2 = NonZeroInt(-10,10,[0,p4c1])
p4pick1 = RandInt(0,3)
p4pick2 = NonZeroInt(0,3,[p4pick1])

p4rad = p4c1*funcvec[p4pick1]*p4c2*funcvec[p4pick2]


\end{sagesilent}

\begin{problem}
    Is the following radical a Type 1 or Type 2 Radical?
    \[
        \sqrt{\sage{p1rad}}
    \]
    
    \begin{multipleChoice}
        \choice{Type 1 (aka only one term within the radical)}
        \choice[correct]{Type 2 (aka more than one term within the radical)}
    \end{multipleChoice}
    
    \begin{problem}
        Can you simplify this radical as it is currently written?
        \begin{multipleChoice}
            \choice{Only if the numbers happen to be perfect squares}
            \choice{Potentially; depending on the values, since it is a Type 2 it is at least possible to simplify.}
            \choice[correct]{Not as it is written. Type 2 radicals cannot be simplified without manipulating them first into Type 1 radicals.}
            \choice{No, there is never anything we can do to simplify Type 2 radicals.}
        \end{multipleChoice}
    \end{problem}
\end{problem}

\begin{problem}
    Is the following radical a Type 1 or Type 2 Radical?
    \[
        \sqrt{\sage{p2rad}}
    \]
    
    \begin{multipleChoice}
        \choice[correct]{Type 1 (aka only one term within the radical)}
        \choice{Type 2 (aka more than one term within the radical)}
    \end{multipleChoice}
    
    \begin{problem}
        Can you simplify this radical as it is currently written?
        \begin{multipleChoice}
            \choice{Only if the numbers happen to be perfect squares}
            \choice[correct]{Potentially; depending on the values, since it is a Type 1 it is at least possible to simplify.}
            \choice{Not as it is written. Type 1 radicals cannot be simplified without manipulating them first into Type 2 radicals.}
            \choice{No, there is never anything we can do to simplify Type 1 radicals.}
        \end{multipleChoice}
    \end{problem}
\end{problem}

\begin{problem}
    Is the following radical a Type 1 or Type 2 Radical?
    \[
        \sqrt{\sage{p3rad}}
    \]
    
    \begin{multipleChoice}
        \choice{Type 1 (aka only one term within the radical)}
        \choice[correct]{Type 2 (aka more than one term within the radical)}
    \end{multipleChoice}
    
    \begin{problem}
        Can you simplify this radical as it is currently written?
        \begin{multipleChoice}
            \choice{Only if the numbers happen to be perfect squares}
            \choice{Potentially; depending on the values, since it is a Type 2 it is at least possible to simplify.}
            \choice[correct]{Not as it is written. Type 2 radicals cannot be simplified without manipulating them first into Type 1 radicals.}
            \choice{No, there is never anything we can do to simplify Type 2 radicals.}
        \end{multipleChoice}
    \end{problem}
\end{problem}

\begin{problem}
    Is the following radical a Type 1 or Type 2 Radical?
    \[
        \sqrt{\sage{p4rad}}
    \]
    
    \begin{multipleChoice}
        \choice[correct]{Type 1 (aka only one term within the radical)}
        \choice{Type 2 (aka more than one term within the radical)}
    \end{multipleChoice}
    
    \begin{problem}
        Can you simplify this radical as it is currently written?
        \begin{multipleChoice}
            \choice{Only if the numbers happen to be perfect squares}
            \choice[correct]{Potentially; depending on the values, since it is a Type 1 it is at least possible to simplify.}
            \choice{Not as it is written. Type 1 radicals cannot be simplified without manipulating them first into Type 2 radicals.}
            \choice{No, there is never anything we can do to simplify Type 1 radicals.}
        \end{multipleChoice}
    \end{problem}
\end{problem}


\end{document}