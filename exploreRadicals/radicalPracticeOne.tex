\documentclass{ximeraXloud}
\input{../preamble}
\title{Radical Practice}

\begin{document}
\begin{abstract}
    Unlimited Practice for Radicals.
\end{abstract}
\maketitle

\textbf{NOTE:} These are all randomized problems. As a result, it is entirely possible to get pretty awful numbers if you are suitably unlucky. Some of these may look bad until you start doing them, but if you see problems that look excessively awful, remember that you can always hit the `Another' button in the top (green refresh arrow) to get new numbers. If you find yourself doing this frequently, you may want to discuss it with your TA to see if you have a gap in your understanding, or to see if the problems are just really that bad (in which case the TA will forward the info to the content authors).

%\input{Useful-Sage-Macros}

\begin{sagesilent}
#####Define default Sage variables.
#Default function variables
var('x,y,z,X,Y,Z')
#Default function names
var('f,g,h,dx,dy,dz,dh,df')
#Default Wild cards
w0 = SR.wild(0)

def RandInt(a,b):
    """ Returns a random integer in [`a`,`b`]. Note that `a` and `b` should be integers themselves to avoid unexpected behavior.
    """
    return QQ(randint(int(a),int(b)))
    # return choice(range(a,b+1))

def NonZeroInt(b,c, avoid = [0]):
    """ Returns a random integer in [`b`,`c`] which is not in `av`. 
        If `av` is not specified, defaults to a non-zero integer.
    """
    while True:
        a = RandInt(b,c)
        if a not in avoid:
            return a

\end{sagesilent}
\begin{sagesilent}
### Problem p1

p1c1 = RandInt(1,5)*RandInt(1,5)
p1c2 = NonZeroInt(2,20,[4,8,9,16])
p1r1 = RandInt(2,4)
p1q1 = (p1c1^p1r1*p1c2)

p1c3 = RandInt(1,5)*RandInt(1,5)
p1c4 = NonZeroInt(2,20,[4,8,9,16])
p1r2 = RandInt(2,4)
p1q2 = (p1c3^p1r2*p1c4)

p1c5 = RandInt(1,5)*RandInt(1,5)
p1c6 = NonZeroInt(2,20,[4,8,9,16])
p1r3 = RandInt(2,4)
p1q3 = (p1c5^p1r3*p1c6)

p1c7 = RandInt(1,5)*RandInt(1,5)
p1c8 = NonZeroInt(2,20,[4,8,9,16])
p1r4 = RandInt(2,4)
p1q4 = (p1c7^p1r4*p1c8)


### Problem p2

p2c1 = RandInt(-15,15)
p2c2 = RandInt(-15,15)
p2c3 = RandInt(-15,15)
p2r1 = RandInt(2,5)
p2r2 = 2*p2r1

p2x1 = p2c1%p2r1
p2x1f = floor(p2c1/p2r1)
p2y1 = p2c2%p2r1
p2y1f = floor(p2c2/p2r1)
p2z1 = p2c3%p2r1
p2z1f = floor(p2c3/p2r1)

p2q1 = x^p2c1*y^p2c2*z^p2c3

if p2r1%2==0:
    if p2c1%p2r1==0 and p2c1%p2r2>0:
        p2xa1 = abs(x^p2x1f)
    else:
        p2xa1 = x^p2x1f
    if p2c2%p2r1==0 and p2c2%p2r2>0:
        p2ya1 = abs(y^p2y1f)
    else:
        p2ya1 = y^p2y1f
    if p2c3%p2r1==0 and p2c3%p2r2>0:
        p2za1 = abs(z^p2z1f)
    else:
        p2za1 = z^p2z1f
else:
    p2xa1 = x^p2x1f
    p2ya1 = y^p2y1f
    p2za1 = z^p2z1f

p2a1 = p2xa1*p2ya1*p2za1
p2a2 = x^p2x1*y^p2y1*z^p2z1


### Problem p3

p3c1 = RandInt(-15,15)
p3c2 = RandInt(-15,15)
p3c3 = RandInt(-15,15)
p3r1 = RandInt(2,5)
p3r2 = 2*p3r1

p3x1 = p3c1%p3r1
p3x1f = floor(p3c1/p3r1)
p3y1 = p3c2%p3r1
p3y1f = floor(p3c2/p3r1)
p3z1 = p3c3%p3r1
p3z1f = floor(p3c3/p3r1)

p3q1 = x^p3c1*y^p3c2*z^p3c3

if p3r1%2==0:
    if p3c1%p3r1==0 and p3c1%p3r2>0:
        p3xa1 = abs(x^p3x1f)
    else:
        p3xa1 = x^p3x1f
    if p3c2%p3r1==0 and p3c2%p3r2>0:
        p3ya1 = abs(y^p3y1f)
    else:
        p3ya1 = y^p3y1f
    if p3c3%p3r1==0 and p3c3%p3r2>0:
        p3za1 = abs(z^p3z1f)
    else:
        p3za1 = z^p3z1f
else:
    p3xa1 = x^p3x1f
    p3ya1 = y^p3y1f
    p3za1 = z^p3z1f

p3a1 = p3xa1*p3ya1*p3za1
p3a2 = x^p3x1*y^p3y1*z^p3z1


### Problem p4

p4c1 = RandInt(-15,15)
p4c2 = RandInt(-15,15)
p4c3 = RandInt(-15,15)
p4r1 = RandInt(2,5)
p4r2 = 2*p4r1

p4x1 = p4c1%p4r1
p4x1f = floor(p4c1/p4r1)
p4y1 = p4c2%p4r1
p4y1f = floor(p4c2/p4r1)
p4z1 = p4c3%p4r1
p4z1f = floor(p4c3/p4r1)

p4q1 = x^p4c1*y^p4c2*z^p4c3

if p4r1%2==0:
    if p4c1%p4r1==0 and p4c1%p4r2>0:
        p4xa1 = abs(x^p4x1f)
    else:
        p4xa1 = x^p4x1f
    if p4c2%p4r1==0 and p4c2%p4r2>0:
        p4ya1 = abs(y^p4y1f)
    else:
        p4ya1 = y^p4y1f
    if p4c3%p4r1==0 and p4c3%p4r2>0:
        p4za1 = abs(z^p4z1f)
    else:
        p4za1 = z^p4z1f
else:
    p4xa1 = x^p4x1f
    p4ya1 = y^p4y1f
    p4za1 = z^p4z1f

p4a1 = p4xa1*p4ya1*p4za1
p4a2 = x^p4x1*y^p4y1*z^p4z1


### Problem p5

p5c1 = RandInt(-15,15)
p5c2 = RandInt(-15,15)
p5c3 = RandInt(-15,15)
p5r1 = RandInt(2,5)
p5r2 = 2*p5r1

p5x1 = p5c1%p5r1
p5x1f = floor(p5c1/p5r1)
p5y1 = p5c2%p5r1
p5y1f = floor(p5c2/p5r1)
p5z1 = p5c3%p5r1
p5z1f = floor(p5c3/p5r1)

p5q1 = x^p5c1*y^p5c2*z^p5c3

if p5r1%2==0:
    if p5c1%p5r1==0 and p5c1%p5r2>0:
        p5xa1 = abs(x^p5x1f)
    else:
        p5xa1 = x^p5x1f
    if p5c2%p5r1==0 and p5c2%p5r2>0:
        p5ya1 = abs(y^p5y1f)
    else:
        p5ya1 = y^p5y1f
    if p5c3%p5r1==0 and p5c3%p5r2>0:
        p5za1 = abs(z^p5z1f)
    else:
        p5za1 = z^p5z1f
else:
    p5xa1 = x^p5x1f
    p5ya1 = y^p5y1f
    p5za1 = z^p5z1f

p5a1 = p5xa1*p5ya1*p5za1
p5a2 = x^p5x1*y^p5y1*z^p5z1


\end{sagesilent}

\begin{problem}% Problem p1
    Simplify the following numeric radicals
\begin{itemize}
\item $\sqrt[\sage{p1r1}]{\sage{p1q1}} = \answer{\sage{p1c1}}\sqrt[\sage{p1r1}]{\sage{p1c2}}$
\item $\sqrt[\sage{p1r2}]{\sage{p1q2}} = \answer{\sage{p1c3}}\sqrt[\sage{p1r2}]{\sage{p1c4}}$
\item $\sqrt[\sage{p1r3}]{\sage{p1q3}} = \answer{\sage{p1c5}}\sqrt[\sage{p1r3}]{\sage{p1c6}}$
\item $\sqrt[\sage{p1r4}]{\sage{p1q4}} = \answer{\sage{p1c7}}\sqrt[\sage{p1r4}]{\sage{p1c8}}$
\end{itemize}
\end{problem}

\begin{problem}
    Simplify the following radical. Make sure there are \textbf{no fractions} in the resulting radicand, and all exponents are \textbf{positive}. 
    \begin{itemize}
        \item $\sqrt[\sage{p2r1}]{\sage{p2q1}} = \answer{\sage{p2a1}}\sqrt[\sage{p2r1}]{\answer{\sage{p2a2}}}$
        \item $\sqrt[\sage{p3r1}]{\sage{p3q1}} = \answer{\sage{p3a1}}\sqrt[\sage{p3r1}]{\answer{\sage{p3a2}}}$
        \item $\sqrt[\sage{p4r1}]{\sage{p4q1}} = \answer{\sage{p4a1}}\sqrt[\sage{p4r1}]{\answer{\sage{p4a2}}}$
        \item $\sqrt[\sage{p5r1}]{\sage{p5q1}} = \answer{\sage{p5a1}}\sqrt[\sage{p5r1}]{\answer{\sage{p5a2}}}$
    \end{itemize}
\end{problem}




\end{document}