\documentclass{ximeraXloud}
\usepackage{longdivision}
\usepackage{polynom}
\usepackage{float}% Use `H' as the figure optional argument to force it's vertical placement to conform to source.
%\usepackage{caption}% Allows us to describe the figures without having "figure 1:" in it. :: Apparently Caption isn't supported.
%    \captionsetup{labelformat=empty}% Actually does the figure configuration stated above.
\usetikzlibrary{arrows.meta,arrows}% Allow nicer arrow heads for tikz.
\usepackage{gensymb, pgfplots}
\usepackage{tabularx}
\usepackage{arydshln}



\graphicspath{
  {./}
  {./explorePolynomials/}
  {./exploreRadicals/}
  {./graphing/}
}

%% Default style for tikZ
\pgfplotsset{my style/.append style={axis x line=middle, axis y line=
middle, xlabel={$x$}, ylabel={$y$}, axis equal }}


%% Because log being natural log is too hard for people.
\let\logOld\log% Keep the old \log definition, just in case we need it.
\renewcommand{\log}{\ln}


%%% Changes in polynom to show the zero coefficient terms
\makeatletter
\def\pld@CF@loop#1+{%
    \ifx\relax#1\else
        \begingroup
          \pld@AccuSetX11%
          \def\pld@frac{{}{}}\let\pld@symbols\@empty\let\pld@vars\@empty
          \pld@false
          #1%
          \let\pld@temp\@empty
          \pld@AccuIfOne{}{\pld@AccuGet\pld@temp
                            \edef\pld@temp{\noexpand\pld@R\pld@temp}}%
           \pld@if \pld@Extend\pld@temp{\expandafter\pld@F\pld@frac}\fi
           \expandafter\pld@CF@loop@\pld@symbols\relax\@empty
           \expandafter\pld@CF@loop@\pld@vars\relax\@empty
           \ifx\@empty\pld@temp
               \def\pld@temp{\pld@R11}%
           \fi
          \global\let\@gtempa\pld@temp
        \endgroup
        \ifx\@empty\@gtempa\else
            \pld@ExtendPoly\pld@tempoly\@gtempa
        \fi
        \expandafter\pld@CF@loop
    \fi}
\def\pld@CMAddToTempoly{%
    \pld@AccuGet\pld@temp\edef\pld@temp{\noexpand\pld@R\pld@temp}%
    \pld@CondenseMonomials\pld@false\pld@symbols
    \ifx\pld@symbols\@empty \else
        \pld@ExtendPoly\pld@temp\pld@symbols
    \fi
    \ifx\pld@temp\@empty \else
        \pld@if
            \expandafter\pld@IfSum\expandafter{\pld@temp}%
                {\expandafter\def\expandafter\pld@temp\expandafter
                    {\expandafter\pld@F\expandafter{\pld@temp}{}}}%
                {}%
        \fi
        \pld@ExtendPoly\pld@tempoly\pld@temp
        \pld@Extend\pld@tempoly{\pld@monom}%
    \fi}
\makeatother




%%%%% Code for making prime factor trees for numbers, taken from user Qrrbrbirlbel at: https://tex.stackexchange.com/questions/131689/how-to-automatically-draw-tree-diagram-of-prime-factorization-with-latex

\usepackage{forest,mathtools,siunitx}
\makeatletter
\def\ifNum#1{\ifnum#1\relax
  \expandafter\pgfutil@firstoftwo\else
  \expandafter\pgfutil@secondoftwo\fi}
\forestset{
  num content/.style={
    delay={
      content/.expanded={\noexpand\num{\forestoption{content}}}}},
  pt@prime/.style={draw, circle},
  pt@start/.style={},
  pt@normal/.style={},
  start primeTree/.style={%
    /utils/exec=%
      % \pt@start holds the current minimum factor, we'll start with 2
      \def\pt@start{2}%
      % \pt@result will hold the to-be-typeset factorization, we'll start with
      % \pgfutil@gobble since we don't want a initial \times
      \let\pt@result\pgfutil@gobble
      % \pt@start@cnt holds the number of ^factors for the current factor
      \def\pt@start@cnt{0}%
      % \pt@lStart will later hold "l"ast factor used
      \let\pt@lStart\pgfutil@empty,
    alias=pt-start,
    pt@start/.try,
    delay={content/.expanded={$\noexpand\num{\forestove{content}}
                            \noexpand\mathrlap{{}= \noexpand\pt@result}$}},
    primeTree},
  primeTree/.code=%
    % take the content of the node and save it in the count
    \c@pgf@counta\forestove{content}\relax
    % if it's 2 we're already finished with the factorization
    \ifNum{\c@pgf@counta=2}{%
      % add the factor
      \pt@addfactor{2}%
      % finalize the factorization of the result
      \pt@addfactor{}%
      % and set the style to the prime style
      \forestset{pt@prime/.try}%
    }{%
      % this simply calculates content/2 and saves it in \pt@end
      % this is later used for an early break of the recursion since no factor
      % can be greater then content/2 (for integers of course)
      \edef\pt@content{\the\c@pgf@counta}%
      \divide\c@pgf@counta2\relax
      \advance\c@pgf@counta1\relax % to be on the safe side
      \edef\pt@end{\the\c@pgf@counta}%
      \pt@do}}

%%% our main "function"
\def\pt@do{%
  % let's test if the current factor is already greather then the max factor
  \ifNum{\pt@end<\pt@start}{%
    % great, we're finished, the same as above
    \expandafter\pt@addfactor\expandafter{\pt@content}%
    \pt@addfactor{}%
    \def\pt@next{\forestset{pt@prime/.try}}%
  }{%
    % this calculates int(content/factor)*factor
    % if factor is a factor of content (without remainder), the result will
    % equal content. The int(content/factor) is saved in \pgf@temp.
    \c@pgf@counta\pt@content\relax
    \divide\c@pgf@counta\pt@start\relax
    \edef\pgf@temp{\the\c@pgf@counta}%
    \multiply\c@pgf@counta\pt@start\relax
    \ifNum{\the\c@pgf@counta=\pt@content}{%
      % yeah, we found a factor, add it to the result and ...
      \expandafter\pt@addfactor\expandafter{\pt@start}%
      % ... add the factor as the first child with style pt@prime
      % and the result of int(content/factor) as another child.
      \edef\pt@next{\noexpand\forestset{%
        append={[\pt@start, pt@prime/.try]},
        append={[\pgf@temp, pt@normal/.try]},
        % forest is complex, this makes sure that for the second child, the
        % primeTree style is not executed too early (there must be a better way).
        delay={
          for descendants={
            delay={if n'=1{primeTree, num content}{}}}}}}%
    }{%
      % Alright this is not a factor, let's get the next factor
      \ifNum{\pt@start=2}{%
        % if the previous factor was 2, the next one will be 3
        \def\pt@start{3}%
      }{%
        % hmm, the previos factor was not 2,
        % let's add 2, maybe we'll hit the next prime number
        % and maybe a factor
        \c@pgf@counta\pt@start
        \advance\c@pgf@counta2\relax
        \edef\pt@start{\the\c@pgf@counta}%
      }%
      % let's do that again
      \let\pt@next\pt@do
    }%
  }%
  \pt@next
}

%%% this builds the \pt@result macro with the factors
\def\pt@addfactor#1{%
  \def\pgf@tempa{#1}%
  % is it the same factor as the previous one
  \ifx\pgf@tempa\pt@lStart
    % add 1 to the counter
    \c@pgf@counta\pt@start@cnt\relax
    \advance\c@pgf@counta1\relax
    \edef\pt@start@cnt{\the\c@pgf@counta}%
  \else
    % a new factor! Add the previous one to the product of factors
    \ifx\pt@lStart\pgfutil@empty\else
      % as long as there actually is one, the \ifnum makes sure we do not add ^1
      \edef\pgf@tempa{\noexpand\num{\pt@lStart}\ifnum\pt@start@cnt>1 
                                           ^{\noexpand\num{\pt@start@cnt}}\fi}%
      \expandafter\pt@addfactor@\expandafter{\pgf@tempa}%
    \fi
    % setup the macros for the next round
    \def\pt@lStart{#1}% <- current (new) factor
    \def\pt@start@cnt{1}% <- first time
  \fi
}
%%% This simply appends "\times #1" to \pt@result, with etoolbox this would be
%%% \appto\pt@result{\times#1}
\def\pt@addfactor@#1{%
  \expandafter\def\expandafter\pt@result\expandafter{\pt@result \times #1}}

%%% Our main macro:
%%% #1 = possible optional argument for forest (can be tikz too)
%%% #2 = the number to factorize
\newcommand*{\PrimeTree}[2][]{%
  \begin{forest}%
    % as the result is set via \mathrlap it doesn't update the bounding box
    % let's fix this:
    tikz={execute at end scope={\pgfmathparse{width("${}=\pt@result$")}%
                         \path ([xshift=\pgfmathresult pt]pt-start.east);}},
    % other optional arguments
    #1
    % And go!
    [#2, start primeTree]
  \end{forest}}
\makeatother


\providecommand\tabitem{\makebox[1em][r]{\textbullet~}}
\providecommand{\letterPlus}{\makebox[0pt][l]{$+$}}
\providecommand{\letterMinus}{\makebox[0pt][l]{$-$}}



%\usepackage{sagetex}
\title{Exam One Review}
\begin{document}
\begin{abstract}
This is an unlimited practice exam for Exam 1. It DOES NOT COUNT TOWARD A GRADE. This practice is entirely optional!
\end{abstract}
\maketitle

\begin{sagesilent}
    def RandInt(a,b):
        """ Returns a random integer in [`a`,`b`]. Note that `a` and `b` should be integers themselves to avoid unexpected behavior.
        """
        return QQ(randint(int(a),int(b)))
        # return choice(range(a,b+1))
    
    def NonZeroInt(b,c, avoid = [0]):
        """ Returns a random integer in [`b`,`c`] which is not in `av`. 
            If `av` is not specified, defaults to a non-zero integer.
        """
        while True:
            a = RandInt(b,c)
            if a not in avoid:
                return a
    
    #### Problem 1
    p1c1 = RandInt(1,5)
    p1c2 = RandInt(1,5)
    p1c3 = RandInt(1,5)
    
    p1c4 = RandInt(1,5)
    p1c5 = RandInt(1,5)
    p1c6 = RandInt(1,5)
    
    p1ans1 = p1c1*p1c4 + p1c2*p1c5 + p1c3*p1c6
    
    #### Problem 2
    p2c1 = RandInt(-10,10)
    p2c2 = RandInt(-10,10)
    
    #### Problem 3
    p3vec = [x, x^2, x^3, log(x), e^x, sqrt(x)]
    p3c1 = NonZeroInt(-10,10)
    p3c2 = NonZeroInt(-10,10)
    p3c3 = RandInt(-10,10)
    p3c4 = RandInt(-10,10)
    p3f1 = choice(p3vec)
    p3f2 = p3c1*p3f1(p3c2*x+p3c3)+p3c4
    
    #### Problem 4
    p4vec = p3vec
    p4f1t = choice(p4vec)
    p4f2t = choice(p4vec)
    
    p4c1 = NonZeroInt(-10,10)
    p4c2 = NonZeroInt(-10,10)
    p4c3 = NonZeroInt(-10,10)
    p4c4 = NonZeroInt(-10,10)
    
    p4f1 = p4c1*p4f1t(x-p4c2)
    p4f2 = p4c3*p4f2t(x) + p4c4
    p4f3 = choice(p4vec)
    
    p4ans1 = p4f1 + p4f2
    p4ans2 = p4f1*p4f2
    p4ans3 = p4f1*p4f2 - p4f1 - p4f2
    p4ans4 = p4f1/p4f2
    p4ans5 = (p4f1 + p4f2)/(p4f1 + p4f3)
    
    #### Problem EXPERIMENTAL
    pEXf1 = piecewise( [ [(-2,1),x^2], [(1,2),x]] )
    pEXg1 = plot(pEXf1, (-2,2), figsize = [3,3])
    pEXg2 = plot(1/x,(-5,5), figsize = [3,3])
    
\end{sagesilent}

\begin{problem}
        You are walking through a flea market when you notice a table selling crystals. The crystals are broken down into a number of types, each with a cost, as follows;
        
        \begin{center}
            \begin{tabular}{| l | l |}\hline
            \textbf{Crystal Type} & \textbf{Cost} \\ \hline
            Colored Crystals & \$ $\sage{p1c1}$ \\ \hline
            Clear Crystals & \$ $\sage{p1c2}$ \\ \hline
            Patterned Crystals & \$ $\sage{p1c3}$ \\ \hline
            \end{tabular}
        \end{center}
        
        Consider the relationship that tells you all the possible individual crystals you can buy based on how much money you have. Is this relationship a function? Note: you are considering specific individual crystals, not how many `types' of crystals you can buy.
        
        \begin{multipleChoice}
            \choice[correct]{No, for one price you can buy many different crystals.}
            \choice{Yes, for one price you can buy many different crystals.}
            \choice{Yes, each crystal costs one specific amount.}
            \choice{No, each crystal costs one specific amount.}
        \end{multipleChoice}
    
        \begin{problem}
            You decide to purchase $\sage{p1c4}$ colored crystals, $\sage{p1c5}$ clear crystals, and $\sage{p1c6}$ patterned crystals. How much does it cost? $\answer{\sage{p1ans1}}$
        \end{problem}
\end{problem}

\begin{problem}
    What are the coordinates of a point that is best described as ``moved $\sage{p2c1}$ units horizontally and $\sage{p2c2}$ units vertically"? (Note: positive is right/up and negative is left/down)
    $(\answer{\sage{p2c1}},\answer{\sage{p2c2}})$
\end{problem}

\begin{problem}
    Consider the function $f(x) = \sage{p3f2(x)}$. What is the parent function of $f(x)$? $\answer{\sage{p3f1}}$.
\end{problem}

\begin{problem}
    Consider the functions $f(x) = \sage{p4f1(x)}$, $g(x) = \sage{p4f2(x)}$ and $h(x) = \sage{p4f3(x)}$.
    
    Compute $(f + g)(x) = \answer{\sage{p4ans1(x)}}$.
    \begin{problem}
        Compute $(fg)(x) = \answer{\sage{p4ans2(x)}}$.
        \begin{problem}
            Compute $(fg)(x) - f(x) - g(x) = \answer{\sage{p4ans3(x)}}$.
            \begin{problem}
                Compute $\left( \frac{f}{g} \right)(x) = \answer{\sage{p4ans4}}$
                \begin{problem}
                    Compute $\left( \frac{f + g}{f + h}\right)(x) = \answer{\sage{p4ans5}}$
                \end{problem}
            \end{problem}
        \end{problem}
    \end{problem}
\end{problem}

\begin{problem}
    This is an experimental problem to try and fix graphing and is probably entirely non-functional... so ignore it.
    
    $\sageplot{pEXg1}$.
\end{problem}

\end{document}