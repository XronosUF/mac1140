\documentclass{ximeraXloud}
\input{../preamble}
%\usepackage{sagetex}
\title{Exam One Review}
\begin{document}
\begin{abstract}
This is an unlimited practice exam for Exam 1. It DOES NOT COUNT TOWARD A GRADE. This practice is entirely optional!
\end{abstract}
\maketitle

\begin{sagesilent}
    def RandInt(a,b):
        """ Returns a random integer in [`a`,`b`]. Note that `a` and `b` should be integers themselves to avoid unexpected behavior.
        """
        return QQ(randint(int(a),int(b)))
        # return choice(range(a,b+1))
    
    def NonZeroInt(b,c, avoid = [0]):
        """ Returns a random integer in [`b`,`c`] which is not in `av`. 
            If `av` is not specified, defaults to a non-zero integer.
        """
        while True:
            a = RandInt(b,c)
            if a not in avoid:
                return a
    
    #### Problem 1
    p1c1 = RandInt(1,5)
    p1c2 = RandInt(1,5)
    p1c3 = RandInt(1,5)
    
    p1c4 = RandInt(1,5)
    p1c5 = RandInt(1,5)
    p1c6 = RandInt(1,5)
    
    p1ans1 = p1c1*p1c4 + p1c2*p1c5 + p1c3*p1c6
    
    #### Problem 2
    p2c1 = RandInt(-10,10)
    p2c2 = RandInt(-10,10)
    
    #### Problem 3
    p3vec = [x, x^2, x^3, log(x), e^x, sqrt(x)]
    p3c1 = NonZeroInt(-10,10)
    p3c2 = NonZeroInt(-10,10)
    p3c3 = RandInt(-10,10)
    p3c4 = RandInt(-10,10)
    p3f1 = choice(p3vec)
    p3f2 = p3c1*p3f1(p3c2*x+p3c3)+p3c4
    
    #### Problem 4
    p4vec = p3vec
    p4f1t = choice(p4vec)
    p4f2t = choice(p4vec)
    
    p4c1 = NonZeroInt(-10,10)
    p4c2 = NonZeroInt(-10,10)
    p4c3 = NonZeroInt(-10,10)
    p4c4 = NonZeroInt(-10,10)
    
    p4f1 = p4c1*p4f1t(x-p4c2)
    p4f2 = p4c3*p4f2t(x) + p4c4
    p4f3 = choice(p4vec)
    
    p4ans1 = p4f1 + p4f2
    p4ans2 = p4f1*p4f2
    p4ans3 = p4f1*p4f2 - p4f1 - p4f2
    p4ans4 = p4f1/p4f2
    p4ans5 = (p4f1 + p4f2)/(p4f1 + p4f3)
    
    #### Problem EXPERIMENTAL
    pEXf1 = piecewise( [ [(-2,1),x^2], [(1,2),x]] )
    pEXg1 = plot(pEXf1, (-2,2), figsize = [3,3])
    pEXg2 = plot(1/x,(-5,5), figsize = [3,3])
    
\end{sagesilent}

\begin{problem}
        You are walking through a flea market when you notice a table selling crystals. The crystals are broken down into a number of types, each with a cost, as follows;
        
        \begin{center}
            \begin{tabular}{| l | l |}\hline
            \textbf{Crystal Type} & \textbf{Cost} \\ \hline
            Colored Crystals & \$ $\sage{p1c1}$ \\ \hline
            Clear Crystals & \$ $\sage{p1c2}$ \\ \hline
            Patterned Crystals & \$ $\sage{p1c3}$ \\ \hline
            \end{tabular}
        \end{center}
        
        Consider the relationship that tells you all the possible individual crystals you can buy based on how much money you have. Is this relationship a function? Note: you are considering specific individual crystals, not how many `types' of crystals you can buy.
        
        \begin{multipleChoice}
            \choice[correct]{No, for one price you can buy many different crystals.}
            \choice{Yes, for one price you can buy many different crystals.}
            \choice{Yes, each crystal costs one specific amount.}
            \choice{No, each crystal costs one specific amount.}
        \end{multipleChoice}
    
        \begin{problem}
            You decide to purchase $\sage{p1c4}$ colored crystals, $\sage{p1c5}$ clear crystals, and $\sage{p1c6}$ patterned crystals. How much does it cost? $\answer{\sage{p1ans1}}$
        \end{problem}
\end{problem}

\begin{problem}
    What are the coordinates of a point that is best described as ``moved $\sage{p2c1}$ units horizontally and $\sage{p2c2}$ units vertically"? (Note: positive is right/up and negative is left/down)
    $(\answer{\sage{p2c1}},\answer{\sage{p2c2}})$
\end{problem}

\begin{problem}
    Consider the function $f(x) = \sage{p3f2(x)}$. What is the parent function of $f(x)$? $\answer{\sage{p3f1}}$.
\end{problem}

\begin{problem}
    Consider the functions $f(x) = \sage{p4f1(x)}$, $g(x) = \sage{p4f2(x)}$ and $h(x) = \sage{p4f3(x)}$.
    
    Compute $(f + g)(x) = \answer{\sage{p4ans1(x)}}$.
    \begin{problem}
        Compute $(fg)(x) = \answer{\sage{p4ans2(x)}}$.
        \begin{problem}
            Compute $(fg)(x) - f(x) - g(x) = \answer{\sage{p4ans3(x)}}$.
            \begin{problem}
                Compute $\left( \frac{f}{g} \right)(x) = \answer{\sage{p4ans4}}$
                \begin{problem}
                    Compute $\left( \frac{f + g}{f + h}\right)(x) = \answer{\sage{p4ans5}}$
                \end{problem}
            \end{problem}
        \end{problem}
    \end{problem}
\end{problem}

\begin{problem}
    This is an experimental problem to try and fix graphing and is probably entirely non-functional... so ignore it.
    
    $\sageplot{pEXg1}$.
\end{problem}

\end{document}