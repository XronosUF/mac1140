\documentclass{ximeraXloud}

\title{Terminology To Know}
\begin{document}
\begin{abstract}
These are important terms and notations for this section.
\end{abstract}
\maketitle

\begin{definition}[Exponent]
    The exponent of a term is the power that the term is being raised to. For example, the exponent of $x^13$ is $13$.
\end{definition}

\begin{definition}[Exponentiation]
    Exponentiation is the process of raising a term to a power. For example, if we have $x = 13$ and ``square'' both sides, we would say we ``exponentiated both sides by two''. This is primarily of use/interest in industry as it is a much more common phrase than saying ``we raised both sides to the power of two.''
\end{definition}

\begin{definition}[Exponential Growth/Decay]
    Exponential Growth (or decay) is any growth (or decay respectively) whose rate depends upon the current population value of the growing (or decaying) substance. For example, a population of deer will have more babies if there are more deer to bread, so it's (unrestricted) growth would be an example of exponential growth. In contrast, if a student has 2 assignments assigned every week, due at the end of the semester, the workload growth is \textit{not} an example of exponential growth, as it grows at the same rate regardless of how many assignments the student currently has. 
\end{definition}

\begin{definition}[Growth/Decay Multiplier]
    The growth/decay multiplier is the ratio of the size of the population after one growth/decay cycle to it's starting population. Thus if you have a population that goes from 100 to 200 in its first growth cycle, then it's growth multiplier is $\frac{200}{100} = 2$.
\end{definition}



\end{document}